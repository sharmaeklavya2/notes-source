\documentclass[a4paper, 12pt, fleqn]{article}

\usepackage{amsmath}
\usepackage{amsthm}
\usepackage{amssymb}
\usepackage[margin=1in]{geometry}
\usepackage{xcolor}
\usepackage{float}
\usepackage{url}
\usepackage{array}
\usepackage{comment}
\usepackage[bookmarksnumbered=true,hypertexnames=false]{hyperref}
\usepackage{algorithm, algpseudocode}
\usepackage[capitalize,sort]{cleveref}

\hypersetup{
    colorlinks,
    linkcolor={red!50!black},
    citecolor={red!50!black},
    urlcolor={blue!50!black}
}

% THEOREMS

\newtheorem{theorem}{Theorem}
\newtheorem{definition}{Definition}
\newtheorem{example}{Example}
\newtheorem{corollary}{Corollary}[theorem]
\newtheorem{lemma}[theorem]{Lemma}

% SHORTHANDS

\newcommand*{\Th}{^{\textrm{th}}}
\newcommand*{\WLoG}{Without loss of generality}
\newcommand*{\wLoG}{without loss of generality}
\newcommand*{\wrt}{with respect to}

% SYMBOLS

\let\eps\epsilon
\newcommand*{\defeq}{:=}

\newcommand*{\Ahat}{\widehat{A}}
\newcommand*{\Bhat}{\widehat{B}}
\newcommand*{\Chat}{\widehat{C}}
\newcommand*{\Jhat}{\widehat{J}}
\newcommand*{\Shat}{\widehat{S}}
\newcommand*{\Yhat}{\widehat{Y}}
\newcommand*{\bhat}{\widehat{b}}
\newcommand*{\what}{\widehat{w}}
\newcommand*{\xhat}{\widehat{x}}
\newcommand*{\yhat}{\widehat{y}}
\newcommand*{\zhat}{\widehat{z}}

\newcommand*{\Btild}{\widetilde{B}}
\newcommand*{\Ctild}{\widetilde{C}}
\newcommand*{\Jtild}{\widetilde{J}}
\newcommand*{\Stild}{\widetilde{S}}
\newcommand*{\Ytild}{\widetilde{Y}}
\newcommand*{\btild}{\widetilde{b}}
\newcommand*{\wtild}{\widetilde{w}}
\newcommand*{\xtild}{\widetilde{x}}
\newcommand*{\ytild}{\widetilde{y}}
\newcommand*{\ztild}{\widetilde{z}}

\newcommand*{\Fcal}{\mathcal{F}}
\newcommand*{\Tcal}{\mathcal{T}}

% MATH

\newcommand*{\floor}[1]{\left\lfloor #1 \right\rfloor}
\newcommand*{\smallfloor}[1]{\lfloor #1 \rfloor}
\newcommand*{\ceil}[1]{\left\lceil #1 \right\rceil}
\newcommand*{\smallceil}[1]{\lceil #1 \rceil}
\newcommand*{\abs}[1]{\left\lvert #1 \right\rvert}
\newcommand*{\smallabs}[1]{\lvert #1 \rvert}
\newcommand*{\norm}[1]{\left\lVert #1 \right\rVert}
\newcommand*{\smallnorm}[1]{\lVert #1 \rVert}
\newcommand*{\Z}{\mathbb{Z}}
\DeclareMathOperator*{\E}{E}
\DeclareMathOperator*{\Var}{Var}
\DeclareMathOperator*{\argmin}{argmin}
\DeclareMathOperator*{\argmax}{argmax}
\DeclareMathOperator{\poly}{poly}
\DeclareMathOperator{\opt}{opt}
\DeclareMathOperator{\support}{support}
\newcommand*{\OPT}{\mathrm{OPT}}


% INITIALIZATIONS

\newcommand{\initMinimal}{
\setlength{\parindent}{0pt}
\setlength{\parskip}{0.5em}
}
\newcommand{\initFromContents}{
\tableofcontents
\newpage
\initMinimal{}
}
\newcommand{\initAfterBeginDocument}{
\maketitle
\initFromContents{}
}
\newcommand{\addMyBib}{
\bibliographystyle{plainurl}
\bibliography{bibdb}
}

\author{Eklavya Sharma}
\date{\empty}


\title{2 -- Perfectly Secret Encryption}

\newcommand*{\StdEncScheme}[1]{(\mathcal{M}, \mathcal{K}, \mathcal{C}, \textsf{Gen}, e, d)}

\begin{document}

\maketitle
\initMinimal{}

\textbf{Random number generation}: Requires high-entropy data and a process
for creating unbiased independent bits from it.

\tableofcontents

\section{Formal definitions of security}

There are multiple definitions of security.

Let $\Pi = \StdEncScheme{}$
be the encryption scheme under consideration.
We will consider ciphertext-only attack by an adversary with unbounded computational power.
Let $M \in \mathcal{M}, K \in \mathcal{K}, C \in \mathcal{C}$
be the random variables corresponding to the message, key and ciphertext.

\begin{definition}
$\Pi$ is secure iff
\[ \forall m \in \mathcal{M}, \forall c \in \mathcal{C}, (P[C=c] > 0
\implies P[M=m|C=c] = P[M=m]) \]
\end{definition}

\begin{definition}
$\Pi$ is secure iff
\[ \forall m \in \mathcal{M}, \forall m' \in \mathcal{M}, \forall c \in \mathcal{C},
P[C=c|M=m] = P[C=c|M=m'] \]
\end{definition}

\begin{definition}[Perfect indistinguishability]
The adversarial indistinguishability experiment $\textrm{PrivK}^{\textrm{eav}}_{A,\Pi}$:
\begin{itemize}
\item The adversary $A$ outputs a pair of messages $m_0, m_1 \in \mathcal{M}$.
\item A key $k$ is generated using $\textsf{Gen}$, and a uniform bit $b \in \{0, 1\}$ is chosen.
Ciphertext $c = e_k(m_b)$, called the challenge ciphertext, is computed and given to $A$.
\item $A$ outputs a bit $b'$.
\item The output of the experiment is defined as
\[ \textrm{PrivK}^{\mathrm{eav}}_{A, \Pi}
= \begin{cases}1 & \textrm{if } b' = b \\ 0 & \textrm{if } b' = b\end{cases} \]
\end{itemize}
$\Pi$ is perfectly indistinguishable iff $P[\textrm{PrivK}^{\textrm{eav}}_{A,\Pi} = 1] = \frac{1}{2}$.
\end{definition}

\begin{theorem}All of the above definitions of security are equivalent.\end{theorem}

\section{One-time Pad}

The one-time pad is the encryption scheme where:
\begin{itemize}
\item $\mathcal{M} = \mathcal{K} = \mathcal{C} = \{0, 1\}^l$.
\item $\textsf{Gen}$ chooses a key uniformly randomly from $\mathcal{K}$.
\item $e_k(x) = d_k(x) = x \oplus k$.
\end{itemize}

\begin{theorem}The one-time pad is a perfectly secure encryption scheme.\end{theorem}

\section{Limitations of perfect security}

\begin{theorem}$\StdEncScheme{}$
is perfectly secure $\implies |\mathcal{K}| \ge |\mathcal{M}|$.\end{theorem}
\begin{proof}[Hint]Consider $\mathcal{M}(c) = \{d_k(c): k \in \mathcal{K}\}$.
Show that $|\mathcal{M} - \mathcal{M}(c)| > 0$.\end{proof}

\section{Shannon's theorem}

\begin{theorem}
Let $\Pi = \StdEncScheme{}$ and $|\mathcal{M}| = |\mathcal{K}| = |\mathcal{C}|$.
$\Pi$ is a perfectly secure encryption scheme iff both of the following are true:
\begin{itemize}
\item $\textsf{Gen}$ chooses every key with equal probability $1/|\mathcal{K}|$.
\item $\forall m \in \mathcal{M}, \forall c \in \mathcal{C}, |\{k: e_k(m) = c\}| = 1$.
\end{itemize}
\end{theorem}

\end{document}

\documentclass[a4paper, 12pt, fleqn]{article}

\usepackage{amsmath}
\usepackage{amsthm}
\usepackage{amssymb}
\usepackage[margin=1in]{geometry}
\usepackage{xcolor}
\usepackage{float}
\usepackage{url}
\usepackage{array}
\usepackage{comment}
\usepackage[bookmarksnumbered=true,hypertexnames=false]{hyperref}
\usepackage{algorithm, algpseudocode}
\usepackage[capitalize,sort]{cleveref}

\hypersetup{
    colorlinks,
    linkcolor={red!50!black},
    citecolor={red!50!black},
    urlcolor={blue!50!black}
}

% THEOREMS

\newtheorem{theorem}{Theorem}
\newtheorem{definition}{Definition}
\newtheorem{example}{Example}
\newtheorem{corollary}{Corollary}[theorem]
\newtheorem{lemma}[theorem]{Lemma}

% SHORTHANDS

\newcommand*{\Th}{^{\textrm{th}}}
\newcommand*{\WLoG}{Without loss of generality}
\newcommand*{\wLoG}{without loss of generality}
\newcommand*{\wrt}{with respect to}

% SYMBOLS

\let\eps\epsilon
\newcommand*{\defeq}{:=}

\newcommand*{\Ahat}{\widehat{A}}
\newcommand*{\Bhat}{\widehat{B}}
\newcommand*{\Chat}{\widehat{C}}
\newcommand*{\Jhat}{\widehat{J}}
\newcommand*{\Shat}{\widehat{S}}
\newcommand*{\Yhat}{\widehat{Y}}
\newcommand*{\bhat}{\widehat{b}}
\newcommand*{\what}{\widehat{w}}
\newcommand*{\xhat}{\widehat{x}}
\newcommand*{\yhat}{\widehat{y}}
\newcommand*{\zhat}{\widehat{z}}

\newcommand*{\Btild}{\widetilde{B}}
\newcommand*{\Ctild}{\widetilde{C}}
\newcommand*{\Jtild}{\widetilde{J}}
\newcommand*{\Stild}{\widetilde{S}}
\newcommand*{\Ytild}{\widetilde{Y}}
\newcommand*{\btild}{\widetilde{b}}
\newcommand*{\wtild}{\widetilde{w}}
\newcommand*{\xtild}{\widetilde{x}}
\newcommand*{\ytild}{\widetilde{y}}
\newcommand*{\ztild}{\widetilde{z}}

\newcommand*{\Fcal}{\mathcal{F}}
\newcommand*{\Tcal}{\mathcal{T}}

% MATH

\newcommand*{\floor}[1]{\left\lfloor #1 \right\rfloor}
\newcommand*{\smallfloor}[1]{\lfloor #1 \rfloor}
\newcommand*{\ceil}[1]{\left\lceil #1 \right\rceil}
\newcommand*{\smallceil}[1]{\lceil #1 \rceil}
\newcommand*{\abs}[1]{\left\lvert #1 \right\rvert}
\newcommand*{\smallabs}[1]{\lvert #1 \rvert}
\newcommand*{\norm}[1]{\left\lVert #1 \right\rVert}
\newcommand*{\smallnorm}[1]{\lVert #1 \rVert}
\newcommand*{\Z}{\mathbb{Z}}
\DeclareMathOperator*{\E}{E}
\DeclareMathOperator*{\Var}{Var}
\DeclareMathOperator*{\argmin}{argmin}
\DeclareMathOperator*{\argmax}{argmax}
\DeclareMathOperator{\poly}{poly}
\DeclareMathOperator{\opt}{opt}
\DeclareMathOperator{\support}{support}
\newcommand*{\OPT}{\mathrm{OPT}}


% INITIALIZATIONS

\newcommand{\initMinimal}{
\setlength{\parindent}{0pt}
\setlength{\parskip}{0.5em}
}
\newcommand{\initFromContents}{
\tableofcontents
\newpage
\initMinimal{}
}
\newcommand{\initAfterBeginDocument}{
\maketitle
\initFromContents{}
}
\newcommand{\addMyBib}{
\bibliographystyle{plainurl}
\bibliography{bibdb}
}

\author{Eklavya Sharma}
\date{\empty}


\title{Set Cover using Linear Programming}

\begin{document}

\maketitle

Let $[n] = \{1, 2, \ldots, n\}$.
In the set cover problem, we are given $m$ sets: $S_i \subseteq [n]$.
$I$ is a feasible solution to the set cover problem iff $\bigcup_{j \in I} S_j = [n]$.
We are also given $m$ positive weights $w \in \mathbb{R}^m$.
$I$ is an optimal solution solution iff $I$ is a feasible solution and $\sum_{i \in I} w_i$ is minimum.

This article looks at this problem in a way similar to the book
`The Design of Approximation Algorithms' by Williamson and Shmoys,
but with different terminology/notation.

\tableofcontents
\initMinimal{}

\section{Mathematical Preliminaries}

\subsection{Element-wise relational operators on vectors}

\begin{definition}
For a vector $z \in \{0, 1\}^n$,
$\operatorname{all}(x) \iff (\forall i \in [n], x_i = 1)$.
\end{definition}

\begin{definition}
Let $x, y \in \mathbb{R}^n$. Then $x < y \in \{0, 1\}^n$, where $(x < y)_i = x_i < y_i$.
Other relational operators $(>, \le, \ge, ==, \neq)$ are defined analogously.
\end{definition}

\begin{definition}
Let $\alpha \in \mathbb{R}$ and $x \in \mathbb{R}^n$.
Then $x < \alpha \in \{0, 1\}^n$, where $(x < \alpha)_i = x_i < \alpha$.
Other relational operators $(>, \le, \ge, ==, \neq)$ are defined analogously.
\end{definition}

\begin{theorem}[Exercise]\label{rel-vec-thm-1}
Let $x \in \mathbb{R}^n$ and $\alpha \in \mathbb{R}$.
\[ (\operatorname{all}(x \ge 0) \wedge (p = (x \ge \alpha))) \implies \operatorname{all}(x \ge \alpha p) \]
\end{theorem}

\begin{theorem}[Exercise]\label{rel-vec-thm-2}
\[ x^T(x==y) = y^T(x==y) \]
\end{theorem}

\begin{theorem}[Exercise]\label{rel-vec-thm-3}
\[ (\operatorname{all}(x \ge 0) \wedge \operatorname{all}(a \le b)) \implies a^Tx \le b^Tx \]
\end{theorem}

\begin{corollary}[Exercise]\label{rel-vec-thm-4}
\[ (\operatorname{all}(A \ge 0) \wedge \operatorname{all}(x \le y)) \implies Ax \le Ay \]
\end{corollary}

\subsection{Linear Programming}

An optimization problem is a 4-tuple (direction, variable, objective, constraint)
where direction is either min or max.

A linear program is an optimization problem of one of these forms:
\begin{itemize}
\item $(\textrm{min}, x, c^Tx, \operatorname{all}(x \ge 0) \wedge \operatorname{all}(Ax \ge b))$
\item $(\textrm{max}, x, c^Tx, \operatorname{all}(x \ge 0) \wedge \operatorname{all}(Ax \le b))$
\end{itemize}
Here $x, b, c$ are vectors and $A$ is a matrix.
If entries of $x$ are restricted to be integers, then the program is an integer linear program.

$\operatorname{argopt}(\textrm{LP})$ is the value of the variable
which optimizes the objective of the linear program LP.
$\operatorname{opt}(\textrm{LP})$ is the optimal objective value of the linear program LP.

\begin{definition}
These 2 programs are duals of each other:
\begin{itemize}
\item $(\textrm{min}, x, c^Tx, \operatorname{all}(x \ge 0) \wedge \operatorname{all}(Ax \ge b))$
\item $(\textrm{max}, x, b^Ty, \operatorname{all}(y \ge 0) \wedge \operatorname{all}(A^Ty \le c))$
\end{itemize}
\end{definition}

\begin{theorem}[Weak duality]
If $x$ is a feasible solution to
$(\textrm{min}, x, c^Tx, \operatorname{all}(x \ge 0) \wedge \operatorname{all}(Ax \ge b))$
and $y$ is a feasible solution to its dual, then $b^Ty \le y^TAx \le c^Tx$.
\end{theorem}
\begin{proof}
(Using theorem \ref{rel-vec-thm-3})
\[ b^Ty = y^Tb \le y^T(Ax) = (A^Ty)^Tx \le c^Tx \]
\end{proof}

\begin{theorem}[Strong duality]
If LP is the linear program
$(\textrm{min}, x, c^Tx, \operatorname{all}(x \ge 0) \wedge \operatorname{all}(Ax \ge b))$
and DLP is its dual, then $\operatorname{opt}(\textrm{LP}) = \operatorname{opt}(\textrm{DLP})$.
\end{theorem}
\begin{proof} Look it up in a textbook. It's too long to write here. \end{proof}

\section{Formulation as ILP}

Denote a solution $I$ by the vector $z \in \{0, 1\}^m$, where
$z_j = \begin{cases} 1 & j \in I \\ 0 & j \not\in I \end{cases}$.

Let $A$ be an $n$ by $m$ matrix where $A[i, j] = i \in S_j$.

Number of sets in $I$ which cover $i$
\[ = |\{ j: i \in S_j \wedge j \in I \}| = |\{j: A[i, j]z_j = 1\}| = \sum_{j=1}^m A[i, j]z_j = (Az)_i \]
Therefore, $z$ is a feasible solution iff $\operatorname{all}(Az \ge 1)$.

The set cover problem is described by the ILP
$(\textrm{min}, z, w^Tz, \operatorname{all}(z > 0) \wedge \operatorname{all}(Az \ge 1))$.
We don't need to specify the constraint $z \ge 1$ because
the optimal solution cannot have an integer entry greater than 1.
This is because if an entry is greater than 1,
reducing it to 1 will continue to respect constraints and will decrease the objective.

Denote by ILP the integer programming problem above.
Let RLP be the relaxed version of the ILP (i.e. with the integral constraint removed).
Let DLP be the dual of RLP.
\[ (\mathbf{1}^T A)_j
= \sum_{i=1}^n A[i, j]
= |S_j| \]

Let $T_i = \{j: i \in S_j\}$, so $T_i$ are the indices of the sets which $i$ belongs to.
Therefore, $A[i, j] = 1 \iff j \in T_i$.
Let $\deg(i) = |T_i|$.
Let $f = \max_{i \in [n]} \deg(i)$.
\[ (\mathbf{1}^T A^T)_i
= \sum_{j=1}^m A[i, j]
= \deg(i) \]

\section{Approximation algorithms}

\subsection{Algorithm 1}

Let $z^* = \operatorname{argopt}(\textrm{ILP})$.
Let $x^* = \operatorname{argopt}(\textrm{RLP})$.

Since ILP is more constrained, $w^Tx^* \le w^Tz^*$.

Let $s = (x \ge \frac{1}{f})$.

\begin{theorem}
$s$ is a feasible solution to ILP.
\end{theorem}
\begin{proof}
\begin{align*}
& \operatorname{all}(Ax^* \ge 1)
\\ &\Rightarrow (Ax^*)_i = \sum_{j=1}^m A[i, j]x^*_j = \sum_{j \in T_i} x^*_j \ge 1
\\ &\Rightarrow \exists j \in T_i, x^*_j \ge \frac{1}{|T_i|} \ge \frac{1}{f}
\\ &\Rightarrow \exists j \in T_i, s_j = 1
\\ &\Rightarrow \exists j \in T_i, s_j = 1
\\ &\Rightarrow \sum_{j \in T_i} s_j \ge 1
\\ &\Rightarrow \sum_{j=1}^m A[i, j]s_j \ge 1
\\ &\Rightarrow (As)_i \ge 1
\\ &\Rightarrow \operatorname{all}(As \ge 1)
\end{align*}
Since $s$ satisfies the ILP constraints, it is a feasible solution to ILP.
\end{proof}
Since $s$ is a feasible solution to ILP but not necessarily optimal, $w^Tz^* \le w^Ts$.

\begin{theorem}
$s$ is an $f$-approximate solution to ILP.
\end{theorem}
\begin{proof}
By theorem \ref{rel-vec-thm-1}, we get $\operatorname{all}(s \le fx^*)$.
By theorem \ref{rel-vec-thm-3}, we get $w^Ts \le fw^Tx^* \le fw^Tz^*$.
\end{proof}

\subsection{Algorithm 2}

DLP = $(\textrm{max}, y, 1^Ty, \operatorname{all}(y \ge 0) \wedge \operatorname{all}(A^Ty \le w))$.

Let $y^* = \operatorname{argopt}(\textrm{DLP})$.
Let $t = (A^Ty == w)$.

\begin{theorem}
$t$ is a feasible solution to ILP.
\end{theorem}
\begin{proof}
To be done
\end{proof}

\begin{theorem}
$t$ is an $f$-approximate solution to ILP.
\end{theorem}
\begin{proof}
\begin{align*}
& w^Tt
\\ &= (A^Ty^* == w)^Tw
\\ &= (A^Ty^* == w)^T(A^Ty^*) \tag{theorem \ref{rel-vec-thm-2}}
\\ &= t^T(A^Ty^*)
\\ &\le 1^T(A^Ty^*) \tag{theorem \ref{rel-vec-thm-3}}
\\ &= (1^TA^T)y^*
\\ &= [\deg(i)]_{i=1}^n y^*
\\ &\le f 1^Ty^*
\\ &= f \operatorname{opt}(DLP)
\\ &\le f \operatorname{opt}(RLP) \tag{strong duality}
\\ &\le f \operatorname{opt}(ILP)
\end{align*}
\end{proof}

\end{document}

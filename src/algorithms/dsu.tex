\documentclass[a4paper, 12pt, fleqn]{article}

\usepackage{amsmath}
\usepackage{amsthm}
\usepackage{amssymb}
\usepackage[margin=1in]{geometry}
\usepackage{xcolor}
\usepackage{float}
\usepackage{url}
\usepackage{array}
\usepackage{comment}
\usepackage[bookmarksnumbered=true,hypertexnames=false]{hyperref}
\usepackage{algorithm, algpseudocode}
\usepackage[capitalize,sort]{cleveref}

\hypersetup{
    colorlinks,
    linkcolor={red!50!black},
    citecolor={red!50!black},
    urlcolor={blue!50!black}
}

% THEOREMS

\newtheorem{theorem}{Theorem}
\newtheorem{definition}{Definition}
\newtheorem{example}{Example}
\newtheorem{corollary}{Corollary}[theorem]
\newtheorem{lemma}[theorem]{Lemma}

% SHORTHANDS

\newcommand*{\Th}{^{\textrm{th}}}
\newcommand*{\WLoG}{Without loss of generality}
\newcommand*{\wLoG}{without loss of generality}
\newcommand*{\wrt}{with respect to}

% SYMBOLS

\let\eps\epsilon
\newcommand*{\defeq}{:=}

\newcommand*{\Ahat}{\widehat{A}}
\newcommand*{\Bhat}{\widehat{B}}
\newcommand*{\Chat}{\widehat{C}}
\newcommand*{\Jhat}{\widehat{J}}
\newcommand*{\Shat}{\widehat{S}}
\newcommand*{\Yhat}{\widehat{Y}}
\newcommand*{\bhat}{\widehat{b}}
\newcommand*{\what}{\widehat{w}}
\newcommand*{\xhat}{\widehat{x}}
\newcommand*{\yhat}{\widehat{y}}
\newcommand*{\zhat}{\widehat{z}}

\newcommand*{\Btild}{\widetilde{B}}
\newcommand*{\Ctild}{\widetilde{C}}
\newcommand*{\Jtild}{\widetilde{J}}
\newcommand*{\Stild}{\widetilde{S}}
\newcommand*{\Ytild}{\widetilde{Y}}
\newcommand*{\btild}{\widetilde{b}}
\newcommand*{\wtild}{\widetilde{w}}
\newcommand*{\xtild}{\widetilde{x}}
\newcommand*{\ytild}{\widetilde{y}}
\newcommand*{\ztild}{\widetilde{z}}

\newcommand*{\Fcal}{\mathcal{F}}
\newcommand*{\Tcal}{\mathcal{T}}

% MATH

\newcommand*{\floor}[1]{\left\lfloor #1 \right\rfloor}
\newcommand*{\smallfloor}[1]{\lfloor #1 \rfloor}
\newcommand*{\ceil}[1]{\left\lceil #1 \right\rceil}
\newcommand*{\smallceil}[1]{\lceil #1 \rceil}
\newcommand*{\abs}[1]{\left\lvert #1 \right\rvert}
\newcommand*{\smallabs}[1]{\lvert #1 \rvert}
\newcommand*{\norm}[1]{\left\lVert #1 \right\rVert}
\newcommand*{\smallnorm}[1]{\lVert #1 \rVert}
\newcommand*{\Z}{\mathbb{Z}}
\DeclareMathOperator*{\E}{E}
\DeclareMathOperator*{\Var}{Var}
\DeclareMathOperator*{\argmin}{argmin}
\DeclareMathOperator*{\argmax}{argmax}
\DeclareMathOperator{\poly}{poly}
\DeclareMathOperator{\opt}{opt}
\DeclareMathOperator{\support}{support}
\newcommand*{\OPT}{\mathrm{OPT}}


% INITIALIZATIONS

\newcommand{\initMinimal}{
\setlength{\parindent}{0pt}
\setlength{\parskip}{0.5em}
}
\newcommand{\initFromContents}{
\tableofcontents
\newpage
\initMinimal{}
}
\newcommand{\initAfterBeginDocument}{
\maketitle
\initFromContents{}
}
\newcommand{\addMyBib}{
\bibliographystyle{plainurl}
\bibliography{bibdb}
}

\author{Eklavya Sharma}
\date{\empty}


\usepackage{listings}
\lstset{
numbers=left,
numberstyle=\tiny,
basicstyle=\ttfamily,
showstringspaces=false,
columns=fullflexible,
keepspaces=true
}

\lstset{language=Python}

\title{Disjoint-set Union}

\begin{document}

\initAfterBeginDocument{}

All (pseudo-)code in this document is based on the python programming language.

\section{Problem}

In the Disjoint-set Union (DSU) problem, we are given a set $S$ of $n$ singleton sets,
i.e. $S = \{\{i\}: 0 \le i < n\}$.

We have to perform $m$ operations on $S$.
Each operation can modify $S$ while maintaining these 2 invariants:
\begin{enumerate}
\item All elements of $S$ are sets.
\item Every integer from 0 to n-1 lies in exactly one set in $S$.
\end{enumerate}

Also, for every set $X \in S$, one of the elements of $X$ will be known as the `representative of $X$',
denoted as $\operatorname{repr}(X)$.

\paragraph*{Types of operations allowed}
\begin{enumerate}
\item \texttt{find(x)}: If $x \in X$, return $\operatorname{repr}(X)$.
\item \texttt{union(x, y)}: Let $x \in X$ and $y \in Y$.
Then remove $X$ and $Y$ from $S$ and add $X \cup Y$ to $S$.
\end{enumerate}

\texttt{union(x, y)} is the only operation which can modify $S$.
It is easy to see that \texttt{union(x, y)} maintains the 2 invariants.

\section{Forest algorithm}

The `forest algorithm' for DSU maintains a forest $F = (V, E)$ of rooted trees
where $V = \{i \in \mathbb{Z}: 0 \le i < n\}$.
Each tree in $F$ corresponds to a set in $S$.
The representative of a set is the root of the corresponding tree.

The forest is stored by keeping track of the parent of each vertex
in an array \texttt{parent} of size $n$.
If a vertex $x$ has no parent, then \texttt{parent[x] = x}.
The algorithm (optionally) maintains 2 additional arrays \texttt{rank} and \texttt{size}.
\texttt{rank[i]} is an upper-bound on the height of vertex $i$
and \texttt{size[i]} is the size of the subtree rooted at vertex $i$.
Initially \texttt{parent[i] = i}, \texttt{rank[i] = 0} and \texttt{size[i] = 1} for all $0 \le i < n$.

This algorithm offers 2 hyperparameters.
These are optional optimizations for speeding up DSU.
\begin{enumerate}
\item \texttt{union\_by}: can be \texttt{None}, \texttt{rank} or \texttt{size}.
\item \texttt{compress\_path}: can be \texttt{False} or \texttt{True}.
\end{enumerate}

This is how \texttt{find} and \texttt{union} are implemented:
\begin{lstlisting}
def find(x):
    if parent[x] == x:
        return x
    else:
        r = find(parent[x])
        if compress_path:
            parent[x] = r
        return r

def link(x, y):
    parent[y] = x
    size[x] += size[y]
    rank[x] = max(rank[x], rank[y] + 1)

def union(x, y):
    x = find(x)
    y = find(y)
    if union_by is not None and union_by[x] < union_by[y]:
        x, y = y, x

    link(x, y)
    return x != y
\end{lstlisting}

\subsection{Performance with no optimizations}

Consider the following operations:
\begin{lstlisting}
for i in range(1, n):
    union(i, i-1)
for i in range(1, m - n):
    find(0)
\end{lstlisting}

When \texttt{union\_by} is \texttt{None}, \texttt{union(x, y)}
makes the tree of $y$ a subtree of $x$.
Therefore, after all the \texttt{union} operations, the forest will be a single chain from 0 to $n-1$.
If \texttt{compress\_path} is \texttt{False}, each \texttt{find(0)} operation will take $\Theta(n)$ time.
Each $\texttt{union}$ operation takes $\Theta(1)$ time.
Therefore, total time taken is $\Theta((m-n)n)$.

\subsection{\texttt{rank} upper-bounds height}

For a tree $T$, let $h(T)$ denote its height, $n(T)$ denote the number of nodes in it
and $r(T) = \operatorname{rank}(\operatorname{repr}(T))$.

\begin{theorem} $h(T) \le r(T)$ throughout the algorithm. \end{theorem}
\begin{proof}
Initially, $h(T) = r(T) = 0$ for every tree $T$.

In a \texttt{find} operation, the height of a tree can only reduce
(it can reduce if \texttt{compress\_path} is \texttt{True},
otherwise it doesn't change).

Suppose \texttt{link(x, y)} is called and $x \in X$ and $y \in Y$.
Then $Y$ is made a subtree of $X$. Let the resulting tree be $Z$.
Suppose $h(X) \le r(X)$ and $h(Y) \le r(Y)$.
\[ h(Z) = \max(h(X), h(Y) + 1) \le \max(r(X), r(Y) + 1) = r(Z) \]

Since $h(T) \le r(T)$ is initially true and remains true across
\texttt{find} and \texttt{union} operations, $h(T) \le r(T)$
is true for all trees across the entire DSU algorithm.
\end{proof}

\subsection{Performance when \texttt{union\_by} is not \texttt{None}}

\begin{theorem}
\textup{\texttt{union\_by}} $\neq$ \textup{\texttt{None}} $\implies \forall T, r(T) \le \lg n(T)$.
\end{theorem}
\begin{proof}
Initially, $\forall T, r(T) = 0 = \lg 1 = \lg n(T)$.

\texttt{find} operations affect neither $r$ nor $n$.

Suppose \texttt{link(x, y)} is called and $x \in X$ and $y \in Y$.
Then $Y$ is made a subtree of $X$. Let the resulting tree be $Z$.
Suppose $r(X) \le \lg n(X)$ and $r(Y) \le \lg n(Y)$.
$r(Z) = \max(r(X), 1 + r(Y))$ and $n(Z) = n(X) + n(Y)$.

\paragraph{Case 1: \texttt{union\_by = size}} \mbox{}\\
\texttt{union\_by = size} $\implies n(Y) \le n(X)$.
\begin{align*}
r(Z) &= \max(r(X), r(Y) + 1)
\\ &\le \max(\lg n(X), \lg n(Y) + 1)
\\ &\le \max(\lg n(X), \lg(2n(Y)))
\\ &\le \lg\max(n(X), 2n(Y))
\end{align*}
$n(X) \le n(X) + n(Y)$ and
$n(Y) \le n(X) \Rightarrow 2n(Y) \le n(X) + n(Y)$.
\[\implies r(Z) \le \lg\max(n(X), 2n(Y)) \le \lg(n(X) + n(Y)) = \lg n(Z) \]

\paragraph{Case 2: \texttt{union\_by = rank}} \mbox{}\\
\texttt{union\_by = rank} $\implies r(Y) \le r(X)$.

\textbf{Case 2a: $r(Y) < r(X)$}\\
\[ r(Z) = \max(r(X), 1 + r(Y)) = h(X) \]
\[ \le \lg n(X) \le \lg n(X) + n(Y) \le \lg n(Z) \]

\textbf{Case 2b: $r(Y) = r(X)$}
\begin{align*}
& r(Z) = \max(r(X), 1 + r(Y)) = 1 + r(Y) = 1 + r(X)
\\ &\Rightarrow r(Z) \le 1 + \lg n(Y) \wedge r(Z) \le 1 + \lg n(X)
\\ &\Rightarrow r(Z) \le 1 + \min(\lg n(Y), \lg n(X))
\\ &\Rightarrow r(Z) \le \lg(2\min(n(X), n(Y)))
\\ &\Rightarrow r(Z) \le \lg(n(X) + n(Y)) = \lg n(Z)
\end{align*}

For both cases 1 and 2, $r(Z) \le \lg n(Z)$.
Therefore, \texttt{union} preserves the invariant $\forall T, r(T) \le \lg n(T)$.
\end{proof}

This means that any tree can have height at most $\lg n$.
Therefore, \texttt{find} and \texttt{union} have a worst-case time complexity of $O(\lg n)$
and \texttt{link} has a worst-case time complexity of $O(1)$.

\subsection{Lower bound on time when \texttt{compress\_path} is \texttt{False}}

When there is no path compression, we can lower bound the worst-case time complexity of \texttt{find}.

Consider these union operations:
\begin{lstlisting}
for i in range(int(log2(n))):
    for j in range(0, n, 1 << (i+1)):
        union(j, j + (1 << i))
\end{lstlisting}
The body of the outer loop is called a round.
There are $\floor{\lg n}$ rounds.

Number of union operations:
\[ \sum_{i=1}^{\floor{\lg n}} \floor{\frac{n}{2^i}}
\le n \sum_{i=1}^{\floor{\lg n}} \frac{1}{2^i}
\le n \left( 1 - 2^{\floor{\lg n}} \right)
\le n - 1 \]

\begin{theorem} After $i$ rounds, there are $\floor{\frac{n}{2^i}}$ trees with height $i$ and size $2^i$. \end{theorem}
\begin{proof}[Proof by induction]

Initially there are $n$ trees of height 0 and size 1, so this is true for $i=0$.

Assume the theorem is true for some $i$ (induction hypothesis).
Just before the $(i+1)^{\textrm{th}}$ round, there are $\floor{\frac{n}{2^i}}$ trees
of height $i$ and size $2^i$.
We can pair them up (if there are odd number of trees, leave the last one unpaired).
When we union them, we get $\floor{\frac{n}{2^{i+1}}}$ trees with height $i+1$ and size $2^{i+1}$
(this doesn't depend on the value of \texttt{union\_by}).

Therefore, the theorem is true by mathematical induction.
\end{proof}

\begin{theorem} \[ \floor{\frac{n}{2^{\floor{\lg n}}}} = 1 \] \end{theorem}

Therefore, after $\floor{\lg n}$ rounds, there is one tree of height $\floor{\lg n}$.
Therefore, worst-case time complexity of \texttt{find} is $\Omega(\lg n)$.

\subsection{Both union-by-rank and path-compression}

\subsubsection{Alt-Ackermann function}

\begin{definition}
For $j \ge 0$ and $k \ge 0$,
\[ A_k(j) = \begin{cases} j + 1 & k = 0 \\ A_{k-1}^{(j+1)}(j) & k \ge 1 \end{cases} \]
Here $A_k^{(0)}(j) = j$ and $A_k^{(i)}(j) = A_k(A_k^{(i-1)}(j))$.
\end{definition}

\begin{theorem} $A_k(0) = 1$ \end{theorem}
\begin{theorem} $A_1(j) = 2j+1$ \end{theorem}
\begin{theorem} $A_2(j) = 2^{j+1}(j+1) - 1$ \end{theorem}
\begin{theorem} $A_3(1) = 2047$ \end{theorem}
\begin{theorem} $A_k(j)$ is a non-decreasing function of $k$ and $j$. \end{theorem}
\begin{theorem} $A_4(1)$ is way too large. \end{theorem}
\begin{proof}
\begin{align*}
& A_4(1)
\\ &= A_3(A_3(1))
\\ &= A_3(2047)
\\ &= A_2^{(2048)}(2047)
\\ &\ge A_2^{(2)}(2047)
\\ &= A_2(A_2(2047))
\\ &= A_2(2^{2048} \times 2048 - 1)
\\ &= 2^{\left( 2^{2059} - 1 \right)}\left( 2^{2059} \right) - 1
\\ &> 2^{2^{2059}}
\\ &> 16^{16^{514}}
\end{align*}
\end{proof}

\begin{definition} $\alpha(n) = \min(\{k: A_k(1) \ge n\})$ \end{definition}
\begin{theorem} $p < \alpha(n) \le q \iff A_p(1) < n \le A_q(1)$ \end{theorem}

\subsubsection{level and iter}

Let $F$ be a DSU forest with $n$ nodes.
For a node $x$, let $x.p$ be its parent and $x.rank$ be its rank.
\begin{theorem} $x \neq x.p \implies x.rank < x.p.rank$ \end{theorem}
\begin{theorem} $x.rank \le \floor{\lg n} \le n-1$ \end{theorem}

We can partition the set of nodes into 3 parts:
\begin{itemize}
\item root nodes: $\{x: x = x.p \}$.
\item leaf nodes: $\{x: x.rank = 0\}$.
\item internal nodes: non-root and non-leaf nodes.
\end{itemize}

level and iter are functions which map an internal node $x$ to an integer.

\begin{definition} $\operatorname{level}(x) = \max(\{k: A_k(x.rank) \le x.p.rank \})$ \end{definition}
\begin{theorem} $k \le \operatorname{level}(x) \iff A_k(x.rank) \le x.p.rank$ \end{theorem}
\begin{theorem} $0 \le \operatorname{level}(x) < \alpha(\floor{\lg n}+1) \le \alpha(n)$ \end{theorem}
\begin{definition} $\operatorname{iter}(x) = \max(\{i: A_{\operatorname{level}(x)}^{(i)}(x.rank) \le x.p.rank \})$ \end{definition}
\begin{theorem} $i \le \operatorname{iter}(x) \iff A_{\operatorname{level}(x)}^{(i)}(x.rank) \le x.p.rank \})$ \end{theorem}
\begin{theorem} $1 \le \operatorname{iter}(x) \le x.rank$ \end{theorem}

\subsubsection{Potential function}

\begin{definition}For a node $x$, the potential function $\phi(x)$ is given by
\[ \phi(x) = \begin{cases} \alpha(n) \cdot x.rank & x \textrm{ is a root or leaf node}
\\ (\alpha(n) - \operatorname{level}(x)) \cdot x.rank - \operatorname{iter}(x) & \textrm{otherwise}
\end{cases} \] \end{definition}

\begin{theorem} $x$ is an internal node $\implies 0 \le \phi(x) < \alpha(n) \cdot x.rank$. \end{theorem}

To be continued $\ldots$

\end{document}

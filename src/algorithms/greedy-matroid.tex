\documentclass[a4paper, 12pt, fleqn]{article}

\usepackage{amsmath}
\usepackage{amsthm}
\usepackage{amssymb}
\usepackage[margin=1in]{geometry}
\usepackage{xcolor}
\usepackage{float}
\usepackage{url}
\usepackage{array}
\usepackage{comment}
\usepackage[bookmarksnumbered=true,hypertexnames=false]{hyperref}
\usepackage{algorithm, algpseudocode}
\usepackage[capitalize,sort]{cleveref}

\hypersetup{
    colorlinks,
    linkcolor={red!50!black},
    citecolor={red!50!black},
    urlcolor={blue!50!black}
}

% THEOREMS

\newtheorem{theorem}{Theorem}
\newtheorem{definition}{Definition}
\newtheorem{example}{Example}
\newtheorem{corollary}{Corollary}[theorem]
\newtheorem{lemma}[theorem]{Lemma}

% SHORTHANDS

\newcommand*{\Th}{^{\textrm{th}}}
\newcommand*{\WLoG}{Without loss of generality}
\newcommand*{\wLoG}{without loss of generality}
\newcommand*{\wrt}{with respect to}

% SYMBOLS

\let\eps\epsilon
\newcommand*{\defeq}{:=}

\newcommand*{\Ahat}{\widehat{A}}
\newcommand*{\Bhat}{\widehat{B}}
\newcommand*{\Chat}{\widehat{C}}
\newcommand*{\Jhat}{\widehat{J}}
\newcommand*{\Shat}{\widehat{S}}
\newcommand*{\Yhat}{\widehat{Y}}
\newcommand*{\bhat}{\widehat{b}}
\newcommand*{\what}{\widehat{w}}
\newcommand*{\xhat}{\widehat{x}}
\newcommand*{\yhat}{\widehat{y}}
\newcommand*{\zhat}{\widehat{z}}

\newcommand*{\Btild}{\widetilde{B}}
\newcommand*{\Ctild}{\widetilde{C}}
\newcommand*{\Jtild}{\widetilde{J}}
\newcommand*{\Stild}{\widetilde{S}}
\newcommand*{\Ytild}{\widetilde{Y}}
\newcommand*{\btild}{\widetilde{b}}
\newcommand*{\wtild}{\widetilde{w}}
\newcommand*{\xtild}{\widetilde{x}}
\newcommand*{\ytild}{\widetilde{y}}
\newcommand*{\ztild}{\widetilde{z}}

\newcommand*{\Fcal}{\mathcal{F}}
\newcommand*{\Tcal}{\mathcal{T}}

% MATH

\newcommand*{\floor}[1]{\left\lfloor #1 \right\rfloor}
\newcommand*{\smallfloor}[1]{\lfloor #1 \rfloor}
\newcommand*{\ceil}[1]{\left\lceil #1 \right\rceil}
\newcommand*{\smallceil}[1]{\lceil #1 \rceil}
\newcommand*{\abs}[1]{\left\lvert #1 \right\rvert}
\newcommand*{\smallabs}[1]{\lvert #1 \rvert}
\newcommand*{\norm}[1]{\left\lVert #1 \right\rVert}
\newcommand*{\smallnorm}[1]{\lVert #1 \rVert}
\newcommand*{\Z}{\mathbb{Z}}
\DeclareMathOperator*{\E}{E}
\DeclareMathOperator*{\Var}{Var}
\DeclareMathOperator*{\argmin}{argmin}
\DeclareMathOperator*{\argmax}{argmax}
\DeclareMathOperator{\poly}{poly}
\DeclareMathOperator{\opt}{opt}
\DeclareMathOperator{\support}{support}
\newcommand*{\OPT}{\mathrm{OPT}}


% INITIALIZATIONS

\newcommand{\initMinimal}{
\setlength{\parindent}{0pt}
\setlength{\parskip}{0.5em}
}
\newcommand{\initFromContents}{
\tableofcontents
\newpage
\initMinimal{}
}
\newcommand{\initAfterBeginDocument}{
\maketitle
\initFromContents{}
}
\newcommand{\addMyBib}{
\bibliographystyle{plainurl}
\bibliography{bibdb}
}

\author{Eklavya Sharma}
\date{\empty}


\title{Greedy Algorithm for max-weight Independent Subset of Matroid}

\begin{document}

\maketitle
\initMinimal{}

\tableofcontents

\section{Preliminaries}

For $x \not\in X$, denote $X \cup \{x\}$ by $X + x$.

Let $M = (S, I)$ be a matroid.
For $X \subseteq S$, the basis of $X$ is a subset $B$ of $X$
such that $B \in I$ and $\forall x \in X, B + x \not\in I$.

All bases of $X$ have the same size (by the exchange axiom).
This size is called the rank of $X$, denoted by $r(X)$.

\begin{theorem}
If $A \subseteq X$ and $A \in I$, then $|A| \le r(X)$.
\end{theorem}

\begin{theorem}
For $e \not\in X$, $r(X) \le r(X+e) \le r(X) + 1$.
\end{theorem}
\begin{proof}
Let $A$ be a basis of $X$. So $A \in I$ and $|A| = r(X)$.
Since $A \subseteq X+e$ and $A \in I$, $|A| \le r(X+e)$.
Therefore, $r(X) \le r(X+e)$.

Let $B$ be a basis of $X+e$. So $B \in I$ and $r(X+e) = |B|$.

If $e \not\in B$, then $B \in I$ and $B \subseteq X$.
Therefore, $|B| \le r(X) \Rightarrow r(X+e) \le r(X) \Rightarrow r(X+e) = r(X) \le r(X) + 1$.

Suppose $e \in B$. $B \in I \Rightarrow B-e \in I$.
Since $B-e \in X$, $|B-e| \le r(X) \Rightarrow r(X+e) - 1 \le r(X)$.
\end{proof}

\begin{theorem}
\label{rank-p1-basis}
Let $e \not\in X$. If $r(X+e) = r(X) + 1$, then:
\begin{itemize}
\item Every basis of $X+e$ contains $e$.
\item $B$ is a basis of $X \iff B+e$ is a basis of $X+e$.
\end{itemize}
\end{theorem}
\begin{proof}
Let $B$ be a basis of $X+e$. Therefore, $r(X+e) = |B|$ and $r(X) = |B|-1$.
If $e \not\in B$, then $B \subseteq X$.
Therefore, $|B| \le r(X) \Rightarrow |B| \le |B|-1 \Rightarrow \bot$.
Therefore, $e \in B$. Therefore, every basis of $X+e$ contains $e$.

Let $A+e$ be a basis of $X+e$ and $B$ be a basis of $X$.
$|A+e| = r(X+e) = r(X) + 1 = |B| + 1$.
Since $A \in I$ and $A \subseteq X$ and $|A| = r(X)$, $A$ is a basis of $X$.

By exchange axiom, we can add an element from $A+e$ to $B$ and maintain independence.
Such an element cannot be from $A$, because $A \subseteq X$ and $B$ is maximal in $X$.
Therefore, $e$ can be added to $B$ while maintaining independence.
Therefore, $B+e \in I$ and $|B+e| = |B| + 1 = r(X+e)$.
Therefore, $B$ is a basis of $X+e$.
\end{proof}

\section{The Problem}

Let $M = (S, I)$ be a matroid.
Let $w: S \mapsto \mathbb{R}^+$ be a weight function.
Overload $w$ for $2^S \mapsto \mathbb{R}$: $w(X) = \sum_{x \in X} w(x)$.

We are given $X \subseteq S$.
We have to find
\[ g(X) = \operatorname{argmax}_{Y \subseteq X, Y \in I}(w(Y)) \]
Also define $f(X) = w(g(X))$.

\begin{theorem}
There is a greedy algorithm which can find $g(X)$ for all $X \subseteq S$.
The algorithm starts with $Y = \{\}$.
It iterates over elements of $X$ in decreasing order of $w$
and adds element $x$ to $Y$ iff $Y + x \in I$.
\end{theorem}

\section{Properties of f and g}

\begin{theorem} $X \in I \implies f(X) = w(X)$ \end{theorem}

\begin{theorem}[Monotonicity] $X \subseteq Y \implies f(X) \le f(Y)$ \end{theorem}
\begin{proof} $g(X) \subseteq X \subseteq Y$.
Since $f$ is the weight of the optimal subset, $f(X) \le f(Y)$ \end{proof}

\begin{theorem} $g(X) = B \implies B$ is a basis of $X$ \end{theorem}

\begin{theorem}
Let $e \not\in X$. $f(X) \le f(X+e) \le f(X) + w(e)$.
\end{theorem}
\begin{proof}
$f(X) \le f(X+e)$ by monotonicity.

Let $B = g(X+e)$. So $B \subseteq X+e$, $B \in I$ and $f(X+e) = w(B)$.
\begin{itemize}
\item \textbf{Case 1}: $e \not\in B$:

Therefore, $B \subseteq X \Rightarrow w(B) \le f(X) \Rightarrow f(X+e) \le f(X) \le f(X) + w(e)$.

\item \textbf{Case 2}: $e \in B$:

$B \in I \Rightarrow B-e \in I \Rightarrow f(B-e) = w(B-e) = w(B) - w(e) = f(X+e) - w(e)$

$B-e \subseteq X \Rightarrow f(B-e) \le f(X)$
$\Rightarrow f(X+e) - w(e) \le f(X)$
$\Rightarrow f(X+e) \le f(X) + w(e)$
\end{itemize}
\end{proof}

\section{Submodularity of f}

To prove that $f$ is submodular, we must prove that
\[ \forall X \subseteq S, \forall y,e \not\in X, f(X+y+e) - f(X+y) \le f(X+e) - f(X) \]

\subsection[Case 1: r(X+e) = r(X) + 1]{Case 1: $r(X+e) = r(X) + 1$}

Let $B = g(X)$. Then by theorem \ref{rank-p1-basis}, $B+e = g(X+e)$.
Therefore, $f(X+e) = w(B+e) = w(B) + w(e) = f(X) + w(e)$.

Since $f(X+y-e) \le f(X+y) + w(e)$, we get that
$f(X+y+e) - f(X+y) \le w(e) = f(X+e) - f(X)$.

\subsection[Case 2: r(X+y) = r(X) + 1]{Case 2: $r(X+y) = r(X) + 1$}

This can be proved similarly to case 1.

\subsection[Case 3: r(X+y) = r(X+e) = r(X) (To be done)]{Case 3: $r(X+y) = r(X+e) = r(X)$ (To be done)}

\begin{theorem} $r(X+y+e) = r(X)$. \end{theorem}
\begin{proof}[Proof by contradiction]
Assume $r(X+y+e) = r(X+y) + 1 = r(X+e) + 1$.
Let $B$ be a basis of $X$.
Since $r(X) = r(X+y)$, $B$ is also a basis of $X+y$.
By theorem \ref{rank-p1-basis}, $B+e$ is a basis of $X+y+e$ and $y \in B+e$.
But $y$ cannot be in $B+e$ if $B$ is a basis of $X$.
Therefore, our assumption is wrong and $r(X+y+e) = r(X)$.
\end{proof}

\end{document}

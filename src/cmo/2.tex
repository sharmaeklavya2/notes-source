\documentclass[a4paper, 12pt, fleqn]{article}

\usepackage{amsmath}
\usepackage{amsthm}
\usepackage{amssymb}
\usepackage[margin=1in]{geometry}
\usepackage{xcolor}
\usepackage{float}
\usepackage{url}
\usepackage{array}
\usepackage{comment}
\usepackage[bookmarksnumbered=true,hypertexnames=false]{hyperref}
\usepackage{algorithm, algpseudocode}
\usepackage[capitalize,sort]{cleveref}

\hypersetup{
    colorlinks,
    linkcolor={red!50!black},
    citecolor={red!50!black},
    urlcolor={blue!50!black}
}

% THEOREMS

\newtheorem{theorem}{Theorem}
\newtheorem{definition}{Definition}
\newtheorem{example}{Example}
\newtheorem{corollary}{Corollary}[theorem]
\newtheorem{lemma}[theorem]{Lemma}

% SHORTHANDS

\newcommand*{\Th}{^{\textrm{th}}}
\newcommand*{\WLoG}{Without loss of generality}
\newcommand*{\wLoG}{without loss of generality}
\newcommand*{\wrt}{with respect to}

% SYMBOLS

\let\eps\epsilon
\newcommand*{\defeq}{:=}

\newcommand*{\Ahat}{\widehat{A}}
\newcommand*{\Bhat}{\widehat{B}}
\newcommand*{\Chat}{\widehat{C}}
\newcommand*{\Jhat}{\widehat{J}}
\newcommand*{\Shat}{\widehat{S}}
\newcommand*{\Yhat}{\widehat{Y}}
\newcommand*{\bhat}{\widehat{b}}
\newcommand*{\what}{\widehat{w}}
\newcommand*{\xhat}{\widehat{x}}
\newcommand*{\yhat}{\widehat{y}}
\newcommand*{\zhat}{\widehat{z}}

\newcommand*{\Btild}{\widetilde{B}}
\newcommand*{\Ctild}{\widetilde{C}}
\newcommand*{\Jtild}{\widetilde{J}}
\newcommand*{\Stild}{\widetilde{S}}
\newcommand*{\Ytild}{\widetilde{Y}}
\newcommand*{\btild}{\widetilde{b}}
\newcommand*{\wtild}{\widetilde{w}}
\newcommand*{\xtild}{\widetilde{x}}
\newcommand*{\ytild}{\widetilde{y}}
\newcommand*{\ztild}{\widetilde{z}}

\newcommand*{\Fcal}{\mathcal{F}}
\newcommand*{\Tcal}{\mathcal{T}}

% MATH

\newcommand*{\floor}[1]{\left\lfloor #1 \right\rfloor}
\newcommand*{\smallfloor}[1]{\lfloor #1 \rfloor}
\newcommand*{\ceil}[1]{\left\lceil #1 \right\rceil}
\newcommand*{\smallceil}[1]{\lceil #1 \rceil}
\newcommand*{\abs}[1]{\left\lvert #1 \right\rvert}
\newcommand*{\smallabs}[1]{\lvert #1 \rvert}
\newcommand*{\norm}[1]{\left\lVert #1 \right\rVert}
\newcommand*{\smallnorm}[1]{\lVert #1 \rVert}
\newcommand*{\Z}{\mathbb{Z}}
\DeclareMathOperator*{\E}{E}
\DeclareMathOperator*{\Var}{Var}
\DeclareMathOperator*{\argmin}{argmin}
\DeclareMathOperator*{\argmax}{argmax}
\DeclareMathOperator{\poly}{poly}
\DeclareMathOperator{\opt}{opt}
\DeclareMathOperator{\support}{support}
\newcommand*{\OPT}{\mathrm{OPT}}


% INITIALIZATIONS

\newcommand{\initMinimal}{
\setlength{\parindent}{0pt}
\setlength{\parskip}{0.5em}
}
\newcommand{\initFromContents}{
\tableofcontents
\newpage
\initMinimal{}
}
\newcommand{\initAfterBeginDocument}{
\maketitle
\initFromContents{}
}
\newcommand{\addMyBib}{
\bibliographystyle{plainurl}
\bibliography{bibdb}
}

\author{Eklavya Sharma}
\date{\empty}


\title{CMO Lecture 2 notes}

\begin{document}

\maketitle
\initMinimal{}

\section{Continuity}

\begin{definition}
\[ \lim_{x\rightarrow p} f(x) = q \iff \forall \epsilon > 0, \exists \delta > 0,
\forall x \in N_{\delta}(p), f(x) \in N_{\epsilon}(q) \]
\end{definition}

\begin{definition}
$f$ is continuous at $x$ $\iff \lim_{x\rightarrow p} f(x) = f(p)$.
$f$ is continuous over $S \iff f$ is continuous at all points $x \in S$.
\end{definition}

\begin{theorem}
Let $S \subseteq \mathbb{R}^d$ be closed and bounded.
Let $f(S) = \{f(x): x \in S\}$.
Let $f$ be continuous over $S$.
Then $f(S)$ is closed and bounded.
\end{theorem}

For optimization problems, $x^*$ is guaranteed to exist iff
$f$ is continuous and $S$ is closed and bounded.
Henceforth, we will assume $S$ to be closed and bounded
and assume functions to be continuous.

Let $g(x) = f(x) - f(p)$. Then $g(p) = 0$.

\section{Asymptotics}

\[ a(x) \in o(b(x)) \iff \lim_{x \rightarrow x_0} \left| \frac{a(x)}{b(x)} \right| \]

For example, at $x=0$, $x^3 \in o(x^2)$.

If $f$ is continuous at $x=p$, $f(x) = f(p) + o(1)$.

\section{Taylor Series}

Let $f: [a,b] \mapsto \mathbb{R}$.
Let $x, y \in [a, b]$.

Suppose $f$ is differentiable only once.
Then $f(y) = f(x) + f'(z)(y-x)$, for some $z \in (x, y)$.

Suppose $f$ is differentiable $k$ times. Then for some $z \in (x, y)$,
\[ f(y) = \sum_{i=0}^{k-1}f^{(i)}(x)\frac{(y-x)^i}{i!} + f^{(k)}(z)\frac{(y-x)^k}{k!} \]

When $f^{(k)}$ is continuous,
\[ f(y) = \sum_{i=0}^k f^{(i)}(x)\frac{(y-x)^i}{i!} + o(1)\frac{(y-x)^k}{k!} \]

Therefore, we can ignore the last term if $x$ is close to $y$.

\end{document}

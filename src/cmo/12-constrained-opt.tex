\documentclass[a4paper, 12pt, fleqn]{article}

\usepackage{amsmath}
\usepackage{amsthm}
\usepackage{amssymb}
\usepackage[margin=1in]{geometry}
\usepackage{xcolor}
\usepackage{float}
\usepackage{url}
\usepackage{array}
\usepackage{comment}
\usepackage[bookmarksnumbered=true,hypertexnames=false]{hyperref}
\usepackage{algorithm, algpseudocode}
\usepackage[capitalize,sort]{cleveref}

\hypersetup{
    colorlinks,
    linkcolor={red!50!black},
    citecolor={red!50!black},
    urlcolor={blue!50!black}
}

% THEOREMS

\newtheorem{theorem}{Theorem}
\newtheorem{definition}{Definition}
\newtheorem{example}{Example}
\newtheorem{corollary}{Corollary}[theorem]
\newtheorem{lemma}[theorem]{Lemma}

% SHORTHANDS

\newcommand*{\Th}{^{\textrm{th}}}
\newcommand*{\WLoG}{Without loss of generality}
\newcommand*{\wLoG}{without loss of generality}
\newcommand*{\wrt}{with respect to}

% SYMBOLS

\let\eps\epsilon
\newcommand*{\defeq}{:=}

\newcommand*{\Ahat}{\widehat{A}}
\newcommand*{\Bhat}{\widehat{B}}
\newcommand*{\Chat}{\widehat{C}}
\newcommand*{\Jhat}{\widehat{J}}
\newcommand*{\Shat}{\widehat{S}}
\newcommand*{\Yhat}{\widehat{Y}}
\newcommand*{\bhat}{\widehat{b}}
\newcommand*{\what}{\widehat{w}}
\newcommand*{\xhat}{\widehat{x}}
\newcommand*{\yhat}{\widehat{y}}
\newcommand*{\zhat}{\widehat{z}}

\newcommand*{\Btild}{\widetilde{B}}
\newcommand*{\Ctild}{\widetilde{C}}
\newcommand*{\Jtild}{\widetilde{J}}
\newcommand*{\Stild}{\widetilde{S}}
\newcommand*{\Ytild}{\widetilde{Y}}
\newcommand*{\btild}{\widetilde{b}}
\newcommand*{\wtild}{\widetilde{w}}
\newcommand*{\xtild}{\widetilde{x}}
\newcommand*{\ytild}{\widetilde{y}}
\newcommand*{\ztild}{\widetilde{z}}

\newcommand*{\Fcal}{\mathcal{F}}
\newcommand*{\Tcal}{\mathcal{T}}

% MATH

\newcommand*{\floor}[1]{\left\lfloor #1 \right\rfloor}
\newcommand*{\smallfloor}[1]{\lfloor #1 \rfloor}
\newcommand*{\ceil}[1]{\left\lceil #1 \right\rceil}
\newcommand*{\smallceil}[1]{\lceil #1 \rceil}
\newcommand*{\abs}[1]{\left\lvert #1 \right\rvert}
\newcommand*{\smallabs}[1]{\lvert #1 \rvert}
\newcommand*{\norm}[1]{\left\lVert #1 \right\rVert}
\newcommand*{\smallnorm}[1]{\lVert #1 \rVert}
\newcommand*{\Z}{\mathbb{Z}}
\DeclareMathOperator*{\E}{E}
\DeclareMathOperator*{\Var}{Var}
\DeclareMathOperator*{\argmin}{argmin}
\DeclareMathOperator*{\argmax}{argmax}
\DeclareMathOperator{\poly}{poly}
\DeclareMathOperator{\opt}{opt}
\DeclareMathOperator{\support}{support}
\newcommand*{\OPT}{\mathrm{OPT}}


% INITIALIZATIONS

\newcommand{\initMinimal}{
\setlength{\parindent}{0pt}
\setlength{\parskip}{0.5em}
}
\newcommand{\initFromContents}{
\tableofcontents
\newpage
\initMinimal{}
}
\newcommand{\initAfterBeginDocument}{
\maketitle
\initFromContents{}
}
\newcommand{\addMyBib}{
\bibliographystyle{plainurl}
\bibliography{bibdb}
}

\author{Eklavya Sharma}
\date{\empty}

\input{src/cmo/common.texlib}

\title{CMO: Constrained Optimization}

\begin{document}

\maketitle
\initMinimal{}

In constrained optimization, we have to find
\[ x^* = \argmin_{x \in C} f(x) \]
where $C \in \mathbb{R}^d$ is a closed set.
$C$ is called the feasible region.
We say that $x$ is feasible iff $x \in C$.

The methods which we developed for unconstrained optimization
often don't work for constrained optimization because properties
of optimal solutions are different here.
For example, if $x^*$ is an unconstrained minimum of $f$,
then $\grad_f(x^*) = 0$. This doesn't hold for constrained minima.
$\min_{x \in [1, 2]} x^2$ is an example.

We'll consider several special cases of constrained optimization.

\newcommand{\FS}{\operatorname{FS}}
\newcommand{\LFS}{\operatorname{LFS}}
\newcommand{\DS}{\operatorname{DS}}

\tableofcontents

\section{Introduction}

\begin{definition}[Feasible directions]
$u \in \mathbb{R}^d$ is feasible direction at $x \in C$ iff
\[ \exists \overline{\alpha} > 0, \forall \alpha \in [0, \overline{\alpha}], x + \alpha u \in C \]
The set of feasible directions at $x$ is denoted by $\FS(x)$.
\end{definition}

\begin{theorem} \label{thm:locmin-implies-nofs}
If $x$ is a local minimum of $f$, then there is no feasible descent direction.
Formally,
\[ \forall u \in \FS(x), \grad_f(x)^Tu \ge 0 \]
\end{theorem}
\begin{proof}[Proof Sketch]
If there is a feasible descent direction $u$ at $x$,
then for any arbitrarily small $\alpha$, we can decrease $f$
by moving $\alpha$ distance towards $u$.
So $f$ is not a local minimum.
\end{proof}

Note that the converse need not be true.
Let $x$ be a saddle point of $f$ and let there be no constraints.
Then every direction is not a descent direction (and not an ascent direction)
but $x$ is not a local minimum.

\section{Projection onto a convex set}

\begin{theorem} \label{thm:convex-fs}
Let $C$ be a convex set. Let $x^* = \argmin_{x \in C} f(x)$. Then
\[ \forall x \in C, x-x^* \in \FS(x^*) \]
\end{theorem}

In this section, we'll now fix the objective function to be $f(x) = \frac{1}{2}\|x-z\|^2$
and consider the feasible region $C$ to be convex. Also, assume that $z \not\in C$.

\begin{definition}[Projection]
Let $x^* = \argmin_{x \in C} f(x)$.
Then $x^*$ is called the projection of $z$ onto $C$.
\end{definition}

\begin{theorem} \label{thm:proj-cond}
\[ x^* = \argmin_{x \in C} f(x) \iff (\forall x \in C, (x^*-z)^T(x-x^*) \ge 0) \]
\end{theorem}
\begin{proof}
Let $x^* = \argmin_{x \in C}$.
By theorem \ref{thm:convex-fs}, we get that
\[ \forall x \in C, x-x^* \in \FS(x^*) \]
By theorem \ref{thm:locmin-implies-nofs}, we get that
\begin{align*}
& \forall x \in C, \grad_f(x^*)^T(x-x^*) \ge 0
\\ &\implies \forall x \in C, (x^*-z)^T(x-x^*) \ge 0
\end{align*}

Now assume that $\forall x \in C, (x^*-z)^T(x-x^*) \ge 0$.
\begin{align*}
f(x) &= \frac{1}{2} \|(x-x^*)+(x^*-z)\|^2
\\ &= \frac{1}{2} \|x-x^*\|^2 + \frac{1}{2} \|x^*-z\|^2 + (x^*-z)^T(x-x^*)
\\ &\ge f(x^*)
\end{align*}
Therefore, $x^* = \argmin_{x \in C} f(x)$.
\end{proof}

\begin{theorem}
There is a half-space which separates $C$ and $z$. Formally,
\[ \forall x \in C, w^Tx > w^Tz \]
where $w = x^* - z$.
\end{theorem}
\begin{proof}
\begin{align*}
& (x^*-z)^T(x-x^*) \ge 0 \tag{by theorem \ref{thm:proj-cond}}
\\ & \implies (x^*-z)^T x & &\ge (x^*-z)^T x^*
\\ & & &\ge (x^*-z)^T (x^* - z + z)
\\ & & &\ge \|x^*-z\|^2 + (x^*-z)^T z
\\ & & &> (x^*-z)^T z
\end{align*}
\end{proof}

\section{Inequality constraints}

Define the feasible region as
\[ C = \{x: (\forall i \in I, c_i(x) \ge 0) \wedge (\forall i \in E, h_i(x) = 0) \]

Here $\{c_i: i \in I\}$ is the set of inequality constraints
and $\{h_i: i \in I\}$ is the set of equality constraints.
Since we can write the constraint $h_i(x) = 0$ as the 2 constraints
$h_i(x) \ge 0$ and $-h_i(x) \ge 0$, we'll ignore equality constraints for now.

Our minimization algorithm will iteratively choose a feasible descent direction
and make a small step in that direction.

By the definition of feasible direction, we get
\[ u \in \FS(x) \iff \exists \overline{\alpha} > 0, \forall \alpha \in [0, \overline{\alpha}],
c_i(x+\alpha u) \ge 0 \]
Also, for $x \in C$, define LFS (called linearized feasible directions) as
\[ \LFS(x) = \bigcap_{i \in I} \begin{cases} \mathbb{R}^d & \textrm{ if } c_i(x) > 0
\\ \{u: \grad_{c_i}(x)^Tu \ge 0\} & \textrm{ if } c_i(x) = 0 \end{cases} \]
Intuitively, LFS should be the same as FS.
Unfortunately, they need not be the same.

Define descent directions (DS) as
\[ u \in \DS(x) \iff \grad_f(x)^Tu < 0 \]
When $\FS(x) \cap \DS(x) = \LFS(x) \cap \DS(x)$, we say that $x$ is regular.
Regularity always holds when the constraints are linear.

At a point $x$, a constraint $c_i$ is said to be active iff $c_i(x) = 0$.

\begin{theorem}[Farkas' Lemma]
Let $A$ be a $d$ by $m$ matrix and $b \in \mathbb{R}^d$.
For a vector $x$, let $x \ge 0$ mean that all components of $x$ are non-negative.
Let $T = \{u \mid b^Tu < 0 \wedge A^Tu \ge 0\}$.
Let $L = \{\lambda \mid b = A\lambda \wedge \lambda \ge 0\}$.
Then $T = \{\} \iff L \neq \{\}$.
\end{theorem}

Let $I'$ be the set of active constraints at $x^*$. Let $|I'| = m$.
Let $A$ be the matrix whose $i^{\textrm{th}}$ column is $\grad_{c_i}(x^*)$.
Then $A$ is a $d$ by $m$ matrix.
Let $b = \grad_f(x^*)$.
Then
\begin{align*}
u \in \LFS(x^*) &\iff A^Tu \ge 0 & u \in \DS(x^*) &\iff b^Tu < 0
\end{align*}
Then by Farkas' lemma, we get that
\[ \LFS(x^*) \cap \DS(x^*) = \{\} \iff (\exists \lambda \ge 0, b = A\lambda) \]
For such a $\lambda$, we have
\[ \grad_f(x^*) = A\lambda = \sum_{i \in I'} \lambda_i\grad_{c_i}(x^*) \]
This is equivalent to saying that
\[ \grad_f(x^*) = \sum_{i \in I} \lambda_i\grad_{c_i}(x^*) \quad \textrm{ where } \lambda_ic_i(x^*) = 0 \]

If $x^*$ is a local minimum and a regular point, then $\LFS(x^*) \cap \DS(x^*) = \{\}$.
So there exists $\lambda \in \mathbb{R}^m$ such that
\begin{itemize}
\item (Primal feasibility) $\forall i \in I, c_i(x^*) \ge 0$.
\item (Stationarity) $\grad_f(x^*) = \sum_{i \in I} \lambda_i\grad_{c_i}(x^*)$.
\item (Dual feasibility) $\forall i \in I, \lambda_i \ge 0$.
\item (Complementary slackness) $\forall i \in I, \lambda_ic_i(x^*) = 0$.
\end{itemize}
These 4 conditions are called `KKT conditions'.
When these conditions hold for $x$ and $\lambda$, $(x, \lambda)$ is said to be a KKT point.

This is generally stated using the Lagrangian function
(we're also going to consider the equality constraints now):
\[ L(x, \lambda, \mu) = f(x) - \lambda^Tc(x) - \mu^Th(x) \]
\begin{itemize}
\item (Primal feasibility) $c(x^*) \ge 0$ and $h(x^*) = 0$.
\item (Stationarity) $\grad_x L(x, \lambda, \mu) = 0$.
\item (Dual feasibility) $\lambda \ge 0$.
\item (Complementary slackness) $\forall i \in I, \lambda_ic_i(x^*) = 0$.
\end{itemize}

\end{document}

\documentclass[a4paper, 12pt, fleqn]{article}

\usepackage{amsmath}
\usepackage{amsthm}
\usepackage{amssymb}
\usepackage[margin=1in]{geometry}
\usepackage{xcolor}
\usepackage{float}
\usepackage{url}
\usepackage{array}
\usepackage{comment}
\usepackage[bookmarksnumbered=true,hypertexnames=false]{hyperref}
\usepackage{algorithm, algpseudocode}
\usepackage[capitalize,sort]{cleveref}

\hypersetup{
    colorlinks,
    linkcolor={red!50!black},
    citecolor={red!50!black},
    urlcolor={blue!50!black}
}

% THEOREMS

\newtheorem{theorem}{Theorem}
\newtheorem{definition}{Definition}
\newtheorem{example}{Example}
\newtheorem{corollary}{Corollary}[theorem]
\newtheorem{lemma}[theorem]{Lemma}

% SHORTHANDS

\newcommand*{\Th}{^{\textrm{th}}}
\newcommand*{\WLoG}{Without loss of generality}
\newcommand*{\wLoG}{without loss of generality}
\newcommand*{\wrt}{with respect to}

% SYMBOLS

\let\eps\epsilon
\newcommand*{\defeq}{:=}

% MATH

\newcommand*{\floor}[1]{\left\lfloor #1 \right\rfloor}
\newcommand*{\smallfloor}[1]{\lfloor #1 \rfloor}
\newcommand*{\ceil}[1]{\left\lceil #1 \right\rceil}
\newcommand*{\smallceil}[1]{\lceil #1 \rceil}
\newcommand*{\abs}[1]{\left\lvert #1 \right\rvert}
\newcommand*{\smallabs}[1]{\lvert #1 \rvert}
\newcommand*{\norm}[1]{\left\lVert #1 \right\rVert}
\newcommand*{\smallnorm}[1]{\lVert #1 \rVert}
\newcommand*{\Z}{\mathbb{Z}}
\DeclareMathOperator*{\E}{E}
\DeclareMathOperator*{\Var}{Var}
\DeclareMathOperator*{\argmin}{argmin}
\DeclareMathOperator*{\argmax}{argmax}
\DeclareMathOperator{\poly}{poly}
\DeclareMathOperator{\opt}{opt}
\newcommand*{\OPT}{\mathrm{OPT}}

% INITIALIZATIONS

\newcommand{\initMinimal}{
\setlength{\parindent}{0pt}
\setlength{\parskip}{0.5em}
}
\newcommand{\initFromContents}{
\tableofcontents
\newpage
\initMinimal{}
}
\newcommand{\initAfterBeginDocument}{
\maketitle
\initFromContents{}
}
\newcommand{\addMyBib}{
\bibliographystyle{plainurl}
\bibliography{bibdb}
}

\author{Eklavya Sharma}
\date{\empty}


\title{Random Walks}

\begin{document}

\maketitle
\initMinimal{}

\section{One-dimensional random walk with left bounce and right absorb}

There are $n+1$ nodes, numbered from $0$ to $n$, where $n \ge 1$.
Let $X_t$ be a random variable denoting our position at time $t$.
Let $X_0 = 0$ (i.e., we start at node 0).

At node 0 we always move to node 1 in the next time step,
i.e., $X_t = 0 \implies X_{t+1} = 1$.
Node $n$ is absorbing, i.e., $X_t = n \implies X_{t+1} = n$.
At every other node $i$, we move to node $i+1$ with probability $p$
and node $i-1$ with probability $1-p$, where $p \in (0, 1)$ is a constant.
So for $0 < i < n$, we have
\[ X_t = i \implies X_{t+1} = \begin{cases} i+1 & \textrm{ with probability } p
\\ i-1 & \textrm{ with probability } 1-p \end{cases} \]

Our aim is to find the expected number of moves to reach $n$ from $0$.

For $0 \le i < n$, let $s_i$ be the expected number of moves to reach $i+1$ from $i$.
Then by linearity of expectation, the expected number of moves to reach $n$ from $0$
is $\sum_{i=0}^{n-1} s_i$.

Consider the sequence of nodes corresponding to a random walk that
starts at $i$ and ends at $i+1$.
Suppose this sequence contains $t$ occurrences of node $i$.
This means that we moved $t-1$ times from node $i$ to node $i-1$
and we moved once from $i$ to $i+1$.
The probability of observing such a sequence is $(1-p)^{t-1}p$,
which means that $t$ is a geometric random variable.
Therefore, the expected number of times we will move from $i$ to $i-1$ is $1/p - 1$.
Whenever we move from $i$ to $i-1$, we will have to random-walk our way back to $i$ from $i-1$,
which will take $s_{i-1}$ steps in expectation.
Therefore,
\[ s_i = 1 + \left(\frac{1}{p}-1\right)(s_{i-1} + 1) = \frac{1-p}{p}s_{i-1} + \frac{1}{p} \]

As a base case, we know that $s_0 = 1$.
Therefore, we get
\[ s_i = \begin{cases} 2i + 1 & \textrm{ if } p = 1/2
\\ \frac{2(1-p)}{1-2p}\left(\frac{1-p}{p}\right)^i - \frac{1}{1-2p} & \textrm{ if } p \neq 1/2
\end{cases} \]
When $p = 1/2$, the expected number of steps to reach $n$ from $0$ is
\[ \sum_{i=0}^{n-1} s_i = \sum_{i=0}^{n-1} (2i+1)
= \sum_{i=0}^{n-1} ((i+1)^2 - i^2) = n^2 \]
When $p \neq 1/2$, the expected number of steps to reach $n$ from $0$ is
\[ \sum_{i=0}^{n-1} s_i
= \frac{2p(1-p)}{(1-2p)^2}\left(\left(\frac{1-p}{p}\right)^n - 1\right) - \frac{n}{1-2p} \]

\subsection{Alternative proof}

Let $Z_i$ be the expected number of steps needed to reach $n$ from $i$.
Then we have $Z_n = 0$ and $Z_0 = Z_1 + 1$. For all other $i$ from 1 to $n-1$,
we either move to $i+1$ with probability $p$ and use $Z_{i+1}$ steps to reach node $n$,
or we move to $i-1$ with probability $1-p$ and use $Z_{i-1}$ steps to reach node $n$.
So, $Z_i = 1 + pZ_{i+1} + (1-p)Z_{i-1}$. Our aim is to find $Z_0$.
\begin{align*}
& Z_i = 1 + pZ_{i+1} + (1-p)Z_{i-1}
\\ &\implies Z_i - Z_{i+1} = \frac{1}{p} + \frac{1-p}{p}(Z_{i-1} - Z_i)
\end{align*}

For $p = 1/2$, we get that for $0 < j < n$,
\begin{align*}
& Z_j - Z_{j+1} = 2 + (Z_{j-1} - Z_j)
\\ &\implies \sum_{j=1}^i (Z_j - Z_{j+1}) = \sum_{j=1}^i 2 + \sum_{j=1}^i (Z_{j-1} - Z_j)
\\ &\implies Z_1 - Z_{i+1} = 2i + Z_{0} - Z_i
\\ &\implies Z_i - Z_{i+1} = 2i + 1
\\ &\implies \sum_{i=1}^{n-1} (Z_i - Z_{i+1}) = \sum_{i=1}^{n-1} (2i + 1)
\\ &\implies Z_1 - Z_n = \sum_{i=1}^{n-1} ((i+1)^2 - i^2) = n^2 - 1
\\ &\implies Z_0 = n^2
\end{align*}

%\addMyBib{}

\end{document}

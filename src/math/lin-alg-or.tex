\documentclass[a4paper,12pt,fleqn]{article}

\usepackage{amsmath,amssymb,amsthm}
\usepackage[margin=1in]{geometry}
\usepackage{xcolor}
\usepackage{url}
\usepackage{float}
\usepackage{array}
\usepackage{comment}
%\usepackage{tikz}
\usepackage[shortlabels]{enumitem}
\usepackage[hypertexnames=false,bookmarksnumbered=true,final]{hyperref}
%\usepackage{algorithm,algpseudocode}
\usepackage[capitalize,sort]{cleveref}

\def\colorschemesepia{sepia}
\def\colorschemedark{dark}
\def\colorschemelight{light}

\ifx\colorscheme\undefined
\let\colorscheme\colorschemelight
\fi

\ifx\colorscheme\colorschemelight
\colorlet{textColor}{black}
\colorlet{bgColor}{white}
\fi

\ifx\colorscheme\colorschemesepia
\definecolor{textColor}{HTML}{433423}
\definecolor{bgColor}{HTML}{fbf0da}
\fi

\ifx\colorscheme\colorschemedark
\definecolor{textColor}{HTML}{bdc1c6}
\definecolor{bgColor}{HTML}{202124}
\definecolor{textBlue}{HTML}{8ab4f8}
\definecolor{textRed}{HTML}{f9968b}
\colorlet{bgRed}{red!40!black}
\colorlet{bgBlue}{red!40!black}
\else
\colorlet{textBlue}{blue!50!black}
\colorlet{textRed}{red!50!black}
\fi

\ifx\colorscheme\colorschemelight\else
\pagecolor{bgColor}
\color{textColor}
\fi

\hypersetup{colorlinks,linkcolor=textRed,citecolor=textRed,urlcolor=textBlue}
% THEOREMS

\newtheorem{theorem}{Theorem}
\newtheorem{definition}{Definition}
\newtheorem{example}{Example}
\newtheorem{corollary}{Corollary}[theorem]
\newtheorem{lemma}[theorem]{Lemma}
\newtheorem{claim}[theorem]{Claim}

% SHORTHANDS

\newcommand*{\Th}{^{\textrm{th}}}
\newcommand*{\WLoG}{Without loss of generality}
\newcommand*{\wLoG}{without loss of generality}
\newcommand*{\wrt}{with respect to}
\let\opname\operatorname

% SYMBOLS

\let\eps\epsilon
\newcommand*{\defeq}{:=}

\newcommand*{\alphahat}{\widehat{\alpha}}
\newcommand*{\betahat}{\widehat{\beta}}
\newcommand*{\Ahat}{\widehat{A}}
\newcommand*{\Bhat}{\widehat{B}}
\newcommand*{\Chat}{\widehat{C}}
\newcommand*{\Jhat}{\widehat{J}}
\newcommand*{\Shat}{\widehat{S}}
\newcommand*{\Yhat}{\widehat{Y}}
\newcommand*{\bhat}{\widehat{b}}
\newcommand*{\shat}{\widehat{s}}
\newcommand*{\what}{\widehat{w}}
\newcommand*{\xhat}{\widehat{x}}
\newcommand*{\yhat}{\widehat{y}}
\newcommand*{\zhat}{\widehat{z}}

\newcommand*{\Btild}{\widetilde{B}}
\newcommand*{\Ctild}{\widetilde{C}}
\newcommand*{\Jtild}{\widetilde{J}}
\newcommand*{\Stild}{\widetilde{S}}
\newcommand*{\Ytild}{\widetilde{Y}}
\newcommand*{\btild}{\widetilde{b}}
\newcommand*{\stild}{\widetilde{s}}
\newcommand*{\wtild}{\widetilde{w}}
\newcommand*{\xtild}{\widetilde{x}}
\newcommand*{\ytild}{\widetilde{y}}
\newcommand*{\ztild}{\widetilde{z}}

\newcommand*{\Fcal}{\mathcal{F}}
\newcommand*{\Tcal}{\mathcal{T}}

\newcommand*{\ebar}{\overline{e}}
\newcommand*{\Xbar}{\overline{X}}
\newcommand*{\xbar}{\overline{x}}
\newcommand*{\Ybar}{\overline{Y}}
\newcommand*{\ybar}{\overline{y}}

% MATH

\newcommand*{\floor}[1]{\left\lfloor #1 \right\rfloor}
\newcommand*{\smallfloor}[1]{\lfloor #1 \rfloor}
\newcommand*{\ceil}[1]{\left\lceil #1 \right\rceil}
\newcommand*{\smallceil}[1]{\lceil #1 \rceil}
\newcommand*{\abs}[1]{\left\lvert #1 \right\rvert}
\newcommand*{\smallabs}[1]{\lvert #1 \rvert}
\newcommand*{\norm}[1]{\left\lVert #1 \right\rVert}
\newcommand*{\smallnorm}[1]{\lVert #1 \rVert}
\newcommand*{\Z}{\mathbb{Z}}
\DeclareMathOperator*{\E}{E}
\DeclareMathOperator*{\Var}{Var}
\DeclareMathOperator*{\argmin}{argmin}
\DeclareMathOperator*{\argmax}{argmax}
\DeclareMathOperator{\poly}{poly}
\DeclareMathOperator{\opt}{opt}
\DeclareMathOperator{\argopt}{argopt}
\DeclareMathOperator{\support}{support}
\newcommand*{\OPT}{\mathrm{OPT}}


\newcommand*{\thmdepurl}[1]{https://sharmaeklavya2.github.io/theoremdep/nodes#1}
\newcommand*{\thmdephref}[2]{\href{\thmdepurl{#1}}{#2}}
\newenvironment*{tightemize}{\begin{itemize}[noitemsep]}{\end{itemize}}

\DeclareMathOperator{\Span}{span}
\DeclareMathOperator{\rank}{rank}
\newcommand*{\zerovec}{\mathbf{0}}
\DeclareMathOperator{\rows}{rows}
\DeclareMathOperator{\cols}{cols}
\DeclareMathOperator{\nullSpace}{nullSpace}
\DeclareMathOperator{\nullity}{nullity}

\title{Linear Algebra Cheat Sheet for OR}
\author{Eklavya Sharma}
\date{\empty}

\begin{document}

\maketitle

\begin{abstract}
This document contains a list of all linear algebra results
that I needed for my Operations Research coursework at UIUC.
The target audience of this document is Operations Research students.
It can be helpful to \href{https://en.wikipedia.org/wiki/Memory_paging}{page}
these results into your brain before a linear-algebra-heavy exam/project,
and proving these results yourself (except those marked DWAP) is a good exercise.
\end{abstract}

%\setlength{\parindent}{0pt}
\setlength{\parskip}{0.5em}

Notation:
\begin{tightemize}
\item DWAP abbreviates ``Don't Worry About the Proof".
\item For any integer $n \ge 0$, define $[n] \defeq \{1, 2, \ldots, n\}$.
\end{tightemize}

\tableofcontents

\section{Vector Spaces}
\label{sec:vector-space}

\subsection{Field}

\begin{definition}[Field]
A field is a set $F$ equipped with two operations $+$ and $\times$ that satisfy some special properties.
(you don't need to know the properties, unless you want to be a pure mathematician,
in which case you can find them \thmdephref{/abstract-algebra/rings/field.html}{here}).
\\ Every field has two special elements, called the additive identity (usually denoted by 0),
and the multiplicative identity (usually denoted by 1).
\end{definition}

\begin{theorem}[DWAP]
The following are fields:
$\mathbb{Q}$ (the set of rational numbers),
$\mathbb{R}$ (the set of real numbers),
$\mathbb{C}$ (the set of complex numbers).
\end{theorem}

\begin{theorem}[DWAP]
$\mathbb{Z}_p \defeq \{0, 1, \ldots, p-1\}$ is a field when $p$ is prime,
where $+$ and $\times$ are addition and multiplication modulo $p$.
\end{theorem}

\subsection{Vector Space}

\begin{definition}
A vector space $V$ over a field $F$ is a set with
a vector addition operation ($V \times V \mapsto V$)
and a scalar multiplication operation ($F \times V \mapsto V$)
which satisfies some special properties.
(you don't need to know the properties, unless you want to be a pure mathematician,
in which case you can find them \thmdephref{/linear-algebra/vector-spaces/vector-space.html}{here}).
\\ Every vector space has a special vector, called the additive identity, denoted by $\zerovec$.
\begin{comment}
\begin{itemize}
\item Properties of $+$:
\begin{itemize}
\item \textbf{Additive closure}: $\forall x \in V, \forall y \in V, x + y \in V$.
\item \textbf{Associativity}: $\forall x,y,z \in V, (x+y)+z = x+(y+z)$.
\item \textbf{Commutativity}: For each $x, y \in V$, $x + y = y + x$.
\item \textbf{Identity}: There exists a special vector in $V$,
    denoted as $\zerovec$, called the \emph{additive identity} of $V$,
    such that $\forall x \in V$, $x + \zerovec = x$.
    (It can be proved that the additive identity is unique.)
\item \textbf{Inverse}: For each $x \in V$, there exists another vector in $V$,
    denoted as $-x$, called the \emph{additive inverse} of $x$,
    such that $x + (-x) = \zerovec$.
    (It can be proved that for a given $x$, the additive inverse is unique.)
\end{itemize}
\item Properties of scalar multiplication:
\begin{itemize}
\item \textbf{Associativity}: $\forall a \in F, \forall b \in F, \forall x \in V, a(bx) = (ab)x$.
\item \textbf{Unit scalar}: $\forall x \in V, 1x = x$,
    where $1$ is the multiplicative identity of $F$.
\end{itemize}
\item Distributivity:
\begin{itemize}
\item $\forall a \in F, \forall b \in F, \forall x \in V, (a+b)x = (ax + bx)$.
\item $\forall a \in F, \forall x \in V, \forall y \in V, a(x+y) = (ax + ay)$.
\end{itemize}
\end{itemize}
\end{comment}
The elements of $V$ are called vectors. The elements of $F$ are called scalars.
\end{definition}

\begin{definition}[Subspace]
Let $V$ be a vector space. $U$ is called a subspace of $V$ if
$U \subseteq V$ and $U$ is also a vector space.
\end{definition}

\begin{comment}
\begin{lemma}[\thmdephref{/linear-algebra/vector-spaces/zero.html}{Zeros in vector space}, DWAP]
For $a, 0$ in field $F$ and $x, \zerovec$ in vector space $V$,
$ax = 0 \iff a=0 \textrm{ or } x=\zerovec$.
\end{lemma}
\end{comment}

\begin{theorem}[DWAP]
The set of all polynomials over field $F$ forms a vector space.
\end{theorem}

\begin{theorem}[DWAP]
For a field $F$, $F^d$ is a vector space.
\end{theorem}

\begin{definition}[Linear and affine combinations]
Let $V$ be a vector space over field $F$.
Let $X \defeq \{x^{(i)}: i \in [k]\}$, where $x^{(i)} \in V$.
Let $y = \sum_{i=1}^k \alpha_i x^{(i)}$, where $\alpha_i \in F$.
\begin{tightemize}
\item $y$ is a called a \emph{linear combination} of $X$.
\item If $\sum_{i=1}^k \alpha_i = 1$, then
    $y$ is called an \emph{affine} combination of $X$.
\end{tightemize}
\end{definition}

\begin{definition}[Span]
$\Span(X)$ is defined as the set of all linear combinations of $X$.
For sets $X$ and $Y$ of vectors, $X$ is called a spanning set of $Y$ if $Y \subseteq \Span(X)$.
\end{definition}

\begin{lemma}
If $X$ spans $S$, then $X$ also spans $\Span(S)$.
\end{lemma}

\begin{theorem}[DWAP]
Let $X$ be a finite subset of $F^d$, where $F$ is a field. Then $\Span(X)$ is a vector space.
\end{theorem}

\subsection{Linear Independence and Basis}

\begin{definition}[Linear independence]
A set $\{x_1, x_2, \ldots, x_n\}$ of vectors over field $F$
is called \emph{linearly independent} iff
\[ \forall (\alpha_1, \ldots, \alpha_n) \in F^n,
\left( \sum_{i=1}^n \alpha_i x_i = 0 \implies (\alpha_i = 0 \;\forall i \in [n])\right). \]
\end{definition}

\begin{lemma}[Incrementing a linearly independent set]
\label{thm:vecsp:inc-linindep}
Let $X$ be a linearly independent set of vectors and $y$ be a vector.
If $y \not\in \Span(X)$, then $X \cup \{y\}$ is linearly independent.
\end{lemma}

\begin{lemma}[Decrementing a linearly dependent set]
\label{thm:vecsp:dec-lindep}
Let $X$ be a linearly dependent set of vectors.
Then $\exists x \in X$ such that $\Span(X) = \Span(X - \{x\})$.
\end{lemma}

\begin{theorem}[DWAP]
\label{thm:vecsp:larger-than-span-is-dep}
Let $X$ be a spanning set of vector space $V$.
If $Y \subseteq V$ and $|Y| > |X|$, then $Y$ is linearly dependent.
\end{theorem}

\begin{definition}[Basis]
Let $S$ be a subset of vector space $V$.
Then $X \subseteq S$ is a basis of $S$ iff (the following definitions are equivalent):
\begin{tightemize}
\item $X$ is linearly independent and spans $S$.
\item $X$ is the largest linearly independent subset of $S$.
\item $X$ is a maximal linearly independent subset of $S$.
\item $X$ is the smallest spanning subset of $S$.
\item $X$ is a minimal spanning subset of $S$.
\end{tightemize}
\end{definition}
Equivalence of these definitions can be proven using
\cref{thm:vecsp:larger-than-span-is-dep,thm:vecsp:inc-linindep,thm:vecsp:dec-lindep}.

\begin{lemma}
If $X$ is a basis of $S$, then $X$ is also a basis of $\Span(S)$.
\end{lemma}

\begin{lemma}
Let $F$ be a field. Let $e^{(i)} \in F^d$ be a vector whose $i\Th$ coordinate is 1
and other coordinates are 0. Then $E \defeq \{e^{(i)}: i \in [d]\}$ is a basis of $F^d$.
($E$ is called the \emph{standard basis} of $F^d$.)
\end{lemma}

\begin{theorem}
All bases of $S$ have the same size. This size is called the rank of $S$ (denoted as $\rank(S)$).
If $S$ is a vector space, it's called the dimension of $S$ (denoted as $\dim(S)$).
\end{theorem}

\begin{theorem}
Let $X$ be a set of $\rank(S)$ vectors. Then
\[ X \textrm{ is a basis of } S \iff X \textrm{ is linearly independent } \iff X \textrm{ spans } S. \]
\end{theorem}

\begin{theorem}[Coordinatization]
Let $B \defeq \{b^{(1)}, b^{(2)}, \ldots, b^{(k)}\}$ be a basis of vector space $V$.
Then $\forall x \in V$ there is a unique tuple $(\alpha_1, \alpha_2, \ldots, \alpha_k)$
such that $x = \sum_{i=1}^k \alpha_ib^{(i)}$.
\end{theorem}

\subsection{Elementary Operations}

\begin{definition}
For a sequence $X \defeq [x_1, \ldots, x_n]$ of vectors (over field $F$),
an elementary operation is one of the following:
\begin{tightemize}
\item For $i \neq j$ and $\alpha \in F$, replace $x_i$ by $x_i + \alpha x_j$.
\item For $i \in [n]$ and $\alpha \in F - \{0\}$, replace $x_i$ by $\alpha x_i$.
\item For $i \neq j$, swap $x_i$ and $x_j$.
\end{tightemize}
\end{definition}

\begin{lemma}[Reversibililty]
Let $X$ be a set of vectors. Let $Y$ be the vectors obtained by applying
an elementary operation on $X$. Then $X$ can be obtained
by applying an elementary operation to $Y$.
\end{lemma}

\begin{lemma}
For a set $X$ of vectors, applying elementary operations doesn't change
$\Span(X)$ or $\rank(X)$.
\end{lemma}

\subsection{Affine Independence}

\begin{definition}
A set $\{x_1, x_2, \ldots, x_n\}$ of vectors over field $F$
is called \emph{affinely independent} iff
\[ \forall (\alpha_1, \ldots, \alpha_n) \in F^n,
\left( \left(\sum_{i=1}^n \alpha_i = 0 \textrm{ and } \sum_{i=1}^n \alpha_i x_i = 0 \right)
    \implies (\alpha_i = 0 \;\forall i \in [n])\right). \]
\end{definition}

\begin{theorem}
The set $\{x^{(i)}: i \in [n]\}$ of vectors is affinely independent iff
$\{x^{(i)} - x^{(n)}: i \in [n-1]\}$ is linearly independent.
\end{theorem}

\begin{theorem}
The set $\{x^{(i)}: i \in [n]\}$ of vectors from $F^d$ is affinely independent iff
$\{(x^{(i)}, 1): i \in [n]\}$ is linearly independent.
\end{theorem}

\section{Matrices}

\begin{lemma}[Matrix of elementary rowops]
For a sequence $S$ of elementary operations, there is a unique matrix $R$ such that
applying $S$ to rows of any matrix $A$ gives us $RA$.
\end{lemma}

\begin{definition}[row space, column space]
Let $F$ be a field and $A \in F^{m \times n}$ be a matrix,
Let $\rows(A)$ be the set of all row vectors of $A$,
and $\cols(A)$ be the set of all column vectors of $A$. Then
\begin{tightemize}
\item $\operatorname{rowSpace}(A) \defeq \Span(\rows(A))$,
\item $\operatorname{colSpace}(A) \defeq \Span(\cols(A))$,
\item $\rank(A) \defeq \operatorname{rowRank}(A) \defeq \rank(\rows(A))$,
\item $\operatorname{colRank}(A) \defeq \rank(\cols(A))$.
\end{tightemize}
\end{definition}

\begin{theorem}[DWAP]
For any matrix $A$, $\operatorname{rowRank}(A) = \operatorname{colRank}(A)$.
\end{theorem}

\begin{definition}[Nullspace and nullity]
For a matrix $A \in F^{m \times n}$, $\nullSpace(A) \defeq \{x \in F^n: Ax = 0\}$
and $\nullity(A) \defeq \dim(\nullSpace(A))$.
\end{definition}

\begin{theorem}[Rank-nullity theorem, DWAP]
$\rank(A) + \nullity(A) = |\cols(A)|$.
\end{theorem}
\begin{proof}[Proof sketch]
We can show that row space and nullspace are not affected by elementary row operations on $A$.
Hence, we can assume that $A$ is in Reduced-Row Echelon Form.
There are $\rank(A)$ pivot columns in $A$.
Given any value of non-pivot variables, we can compute the value of pivot variables
such that $Ax = 0$. Hence, $\nullity(A) = |\cols(A)| - \rank(A)$.
\end{proof}

\begin{theorem}[DWAP]
If $V$ is a subspace of $F^d$, then $\exists A$ such that $V = \nullSpace(A)$.
\end{theorem}

\begin{theorem}[DWAP]
Basic results on matrix multiplication:
\begin{tightemize}
\item Matrix multiplication is associative, i.e., $(AB)C = A(BC)$.
\item $(AB)^T = B^TA^T$.
\item $|AB| = |A||B|$ if $A$ and $B$ are square ($|A|$ is the determinant of $A$).
\item $(AB)^{-1} = B^{-1}A^{-1}$ if $A$ and $B$ are invertible.
\end{tightemize}
\end{theorem}

\begin{theorem}[Matrix singularity, DWAP]
Let $A \in F^{n \times n}$. The following are equivalent:
\begin{tightemize}
\item $\rank(A) = n$.
\item 0 is the unique solution to $Ax = 0$.
\item $|A| \neq 0$ ($|A|$ is the determinant of $A$).
\item $A$ is invertible, i.e., $\exists B \in F^{n \times n}$ such that $AB = BA = I$.
\end{tightemize}
\end{theorem}

\section{Miscellaneous}

\begin{definition}[$p$-norms]
For $x \in \mathbb{R}^d$,
\begin{align*}
\|x\|_p &\defeq \left(\sum_{i=1}^d |x_i|^p\right)^{1/p}
& \|x\|_{\infty} &\defeq \max_{i=1}^d |x_i|
& \|x\| &\defeq \|x\|_2
\end{align*}
\end{definition}

\begin{definition}[Linear combinations]
Let $V$ be a vector space over field $\mathbb{R}$.
Let $X \defeq \{x^{(i)}: i \in [k]\}$, where $x^{(i)} \in V$.
Let $y = \sum_{i=1}^k \alpha_i x^{(i)}$, where $\alpha_i \in F$.
\begin{tightemize}
\item $y$ is a called a \emph{linear combination} of $X$.
\item If $\alpha_i \ge 0$ for all $i \in [k]$, then
    $y$ is a called a non-negative linear combination of $X$.
\item If $\sum_{i=1}^k \alpha_i = 1$, then
    $y$ is called an \emph{affine} combination of $X$.
\item A non-negative affine combination is called a convex combination.
\end{tightemize}

\end{definition}
\begin{theorem}[\thmdephref{/linear-algebra/inner-product-spaces/cauchy-schwarz-inequality.html}{Cauchy-Schwarz inequality}]
$\forall x, y \in \mathbb{R}^d$, $|x^Ty| \le \|x\|\|y\|$.
\end{theorem}

%\bibliographystyle{plainurl}
%\bibliography{bibdb}

\end{document}

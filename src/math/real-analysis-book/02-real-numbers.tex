\documentclass[a4paper, 12pt, fleqn]{article}

\usepackage{amsmath,amsthm,amssymb}
\usepackage[margin=1in]{geometry}
\usepackage{xcolor}
\usepackage{array}
\usepackage{comment}
\usepackage[bookmarksnumbered=true,hypertexnames=true]{hyperref}
\usepackage{algorithm, algpseudocode}
\usepackage[capitalize,sort]{cleveref}

%\def\colorscheme{dark}
\def\colorschemesepia{sepia}
\def\colorschemedark{dark}
\def\colorschemelight{light}

\ifx\colorscheme\undefined
\let\colorscheme\colorschemelight
\fi

\ifx\colorscheme\colorschemelight
\colorlet{textColor}{black}
\colorlet{bgColor}{white}
\fi

\ifx\colorscheme\colorschemesepia
\definecolor{textColor}{HTML}{433423}
\definecolor{bgColor}{HTML}{fbf0da}
\fi

\ifx\colorscheme\colorschemedark
\definecolor{textColor}{HTML}{bdc1c6}
\definecolor{bgColor}{HTML}{202124}
\definecolor{textBlue}{HTML}{8ab4f8}
\definecolor{textRed}{HTML}{f9968b}
\colorlet{bgRed}{red!40!black}
\colorlet{bgBlue}{red!40!black}
\else
\colorlet{textBlue}{blue!50!black}
\colorlet{textRed}{red!50!black}
\fi

\ifx\colorscheme\colorschemelight\else
\pagecolor{bgColor}
\color{textColor}
\fi

\hypersetup{colorlinks,linkcolor=textRed,citecolor=textRed,urlcolor=textBlue}
% THEOREMS

\newtheorem{theorem}{Theorem}
\newtheorem{definition}{Definition}
\newtheorem{example}{Example}
\newtheorem{corollary}{Corollary}[theorem]
\newtheorem{lemma}[theorem]{Lemma}
\newtheorem{claim}[theorem]{Claim}

% SHORTHANDS

\newcommand*{\Th}{^{\textrm{th}}}
\newcommand*{\WLoG}{Without loss of generality}
\newcommand*{\wLoG}{without loss of generality}
\newcommand*{\wrt}{with respect to}
\let\opname\operatorname

% SYMBOLS

\let\eps\epsilon
\newcommand*{\defeq}{:=}

\newcommand*{\alphahat}{\widehat{\alpha}}
\newcommand*{\betahat}{\widehat{\beta}}
\newcommand*{\Ahat}{\widehat{A}}
\newcommand*{\Bhat}{\widehat{B}}
\newcommand*{\Chat}{\widehat{C}}
\newcommand*{\Jhat}{\widehat{J}}
\newcommand*{\Shat}{\widehat{S}}
\newcommand*{\Yhat}{\widehat{Y}}
\newcommand*{\bhat}{\widehat{b}}
\newcommand*{\shat}{\widehat{s}}
\newcommand*{\what}{\widehat{w}}
\newcommand*{\xhat}{\widehat{x}}
\newcommand*{\yhat}{\widehat{y}}
\newcommand*{\zhat}{\widehat{z}}

\newcommand*{\Btild}{\widetilde{B}}
\newcommand*{\Ctild}{\widetilde{C}}
\newcommand*{\Jtild}{\widetilde{J}}
\newcommand*{\Stild}{\widetilde{S}}
\newcommand*{\Ytild}{\widetilde{Y}}
\newcommand*{\btild}{\widetilde{b}}
\newcommand*{\stild}{\widetilde{s}}
\newcommand*{\wtild}{\widetilde{w}}
\newcommand*{\xtild}{\widetilde{x}}
\newcommand*{\ytild}{\widetilde{y}}
\newcommand*{\ztild}{\widetilde{z}}

\newcommand*{\Fcal}{\mathcal{F}}
\newcommand*{\Tcal}{\mathcal{T}}

\newcommand*{\ebar}{\overline{e}}
\newcommand*{\Xbar}{\overline{X}}
\newcommand*{\xbar}{\overline{x}}
\newcommand*{\Ybar}{\overline{Y}}
\newcommand*{\ybar}{\overline{y}}

% MATH

\newcommand*{\floor}[1]{\left\lfloor #1 \right\rfloor}
\newcommand*{\smallfloor}[1]{\lfloor #1 \rfloor}
\newcommand*{\ceil}[1]{\left\lceil #1 \right\rceil}
\newcommand*{\smallceil}[1]{\lceil #1 \rceil}
\newcommand*{\abs}[1]{\left\lvert #1 \right\rvert}
\newcommand*{\smallabs}[1]{\lvert #1 \rvert}
\newcommand*{\norm}[1]{\left\lVert #1 \right\rVert}
\newcommand*{\smallnorm}[1]{\lVert #1 \rVert}
\newcommand*{\Z}{\mathbb{Z}}
\DeclareMathOperator*{\E}{E}
\DeclareMathOperator*{\Var}{Var}
\DeclareMathOperator*{\argmin}{argmin}
\DeclareMathOperator*{\argmax}{argmax}
\DeclareMathOperator{\poly}{poly}
\DeclareMathOperator{\opt}{opt}
\DeclareMathOperator{\argopt}{argopt}
\DeclareMathOperator{\support}{support}
\newcommand*{\OPT}{\mathrm{OPT}}


\author{\empty}
\date{\empty}

\title{Chapter 2: Real numbers}

\begin{document}

\maketitle
\setlength{\parskip}{0.2em}

\section{Groups}

\begin{definition}[Group]
Let $G$ be a non-empty set and $\circ: G \times G \to G$ be a binary operator.
Then $(G, \circ)$ is a group iff all of the following hold:
\begin{enumerate}
\item Associativity: $a \circ (b \circ c) = (a \circ b) \circ c$ for all $a, b, c \in G$.
\item Identity exists: $\exists e \in G$ such that $\forall a \in G$, $e \circ a = a \circ e = a$.
    Such an $e$ is called an \emph{identity} of $(G, \circ)$.
    We can prove that the identity is unique.
\item Inverses exist: Let $e$ be an identity of $(G, \circ)$. Then $\forall a \in G$,
    $(\exists \ell \in G, \ell \circ a = e)$ and $(\exists r \in G, a \circ r = e)$.
    $\ell$ is called a \emph{left inverse} of $a$. $r$ is called a \emph{right inverse} of $a$.
\end{enumerate}
$(G, \circ)$ is called \emph{symmetric}, \emph{commutative}, or \emph{abelian} iff
$\forall a \in G$, $\forall b \in G$, $a \circ b = b \circ a$.
\end{definition}

\begin{lemma}
In a group $(G, \circ)$, the identity is unique and each element has a unique inverse.
\end{lemma}
\begin{proof}
Let $e_1$ and $e_2$ be identities of $(G, \circ)$.
Then $e_1 \circ e_2 = e_1$, since $e_2$ is an identity, and $e_2 \circ e_1 = e_2$, since $e_1$ is an identity.
Hence, $e_1 = e_2$.

Let $\ell$ be a left inverse and $r$ be a right inverse of $a \in G$. Then
\[ \ell = \ell \circ e = \ell \circ (a \circ r) = (\ell \circ a) \circ r = e \circ r = r. \]
Hence, every left inverse equals every right inverse. Hence, they are all equal.
\end{proof}

If we use $+$ as a group operator, we denote identity as 0 and inverse of $g$ as $-g$.
If we use $\times$ as a group operator, we denote identity as 1 and inverse of $g$ as $g^{-1}$.

\begin{definition}
Let $(G, \times)$ be a group. Then for any $n \in \mathbb{Z}$ and any $g \in G$, define
\[ g^n = \begin{cases}
g \times g \times \ldots \times g \; (n \textrm{ times}) & \textrm{if } n > 0
\\ 1 & \textrm{if } n = 0
\\ g^{-1} \times g^{-1} \times \ldots \times g^{-1} \; (-n \textrm{ times}) & \textrm{if } n < 0
\end{cases}. \]
\end{definition}

\begin{lemma}[Basic properties]
Let $(G, \cdot)$ be a group. Let $a, b \in G$ and $m, n \in \mathbb{Z}$.
\begin{enumerate}
\item $(ab)^{-1} = b^{-1}a^{-1}$.
\item $(a^{-1})^{-1} = a$.
\item $a^ma^n = a^{m+n}$.
\item $(a^m)^n = a^{mn}$.
\item If $G$ is symmetric, $(ab)^n = a^nb^n$.
\end{enumerate}
\end{lemma}

\section{Fields}

\begin{definition}[Field]
$(F, +, \times)$ is a field iff it satisfies all of the following:
\begin{enumerate}
\item $(F, +)$ is a symmetric group. It's identity is denoted as 0.
\item $(F - \{0\}, \times)$ is a symmetric group. It's identity is denoted as 1.
\item Distributivity: $a(b + c) = ab + ac$ and $(a+b)c = ac + bc$.
\end{enumerate}
\end{definition}

\begin{lemma}[Basic properties]
Let $(F, +, \times)$ be a field. Let $a, b \in F$.
\begin{enumerate}
\item $a0 = 0a = 0$.
\item $a(-b) = (-a)b = -(ab)$.
\item $(-a)(-b) = ab$.
\item $ab = 0 \iff (a = 0 \textrm{ or } b = 0)$.
\item $(-a)^{-1} = -a^{-1}$.
\end{enumerate}
\end{lemma}
\begin{proof}[Proof sketches]
\leavevmode
\begin{enumerate}
\item $a0 = a(0 + 0) = a0 + a0$.
\item $0 = a0 = a(b + (-b)) = ab + a(-b)$.
\item $(-a)(-b) = a(-(-b)) = ab$.
\item Suppose $a \neq 0$. Then $ab = 0 \implies b = a^{-1}0 = 0$.
\item $(-1)(-1) = 1$, so $(-1)^{-1} = -1$. $(-a)^{-1} = ((-1)a)^{-1} = (-1)^{-1}a^{-1} = -a^{-1}$.
\end{enumerate}
\end{proof}

\end{document}

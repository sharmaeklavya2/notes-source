\documentclass[a4paper, 12pt, fleqn]{article}

\usepackage{amsmath,amsthm,amssymb}
\usepackage[margin=1in]{geometry}
\usepackage{xcolor}
\usepackage{array}
\usepackage{comment}
\usepackage[bookmarksnumbered=true,hypertexnames=true]{hyperref}
\usepackage{algorithm, algpseudocode}
\usepackage[capitalize,sort]{cleveref}

%\def\colorscheme{dark}
\def\colorschemesepia{sepia}
\def\colorschemedark{dark}
\def\colorschemelight{light}

\ifx\colorscheme\undefined
\let\colorscheme\colorschemelight
\fi

\ifx\colorscheme\colorschemelight
\colorlet{textColor}{black}
\colorlet{bgColor}{white}
\fi

\ifx\colorscheme\colorschemesepia
\definecolor{textColor}{HTML}{433423}
\definecolor{bgColor}{HTML}{fbf0da}
\fi

\ifx\colorscheme\colorschemedark
\definecolor{textColor}{HTML}{bdc1c6}
\definecolor{bgColor}{HTML}{202124}
\definecolor{textBlue}{HTML}{8ab4f8}
\definecolor{textRed}{HTML}{f9968b}
\colorlet{bgRed}{red!40!black}
\colorlet{bgBlue}{red!40!black}
\else
\colorlet{textBlue}{blue!50!black}
\colorlet{textRed}{red!50!black}
\fi

\ifx\colorscheme\colorschemelight\else
\pagecolor{bgColor}
\color{textColor}
\fi

\hypersetup{colorlinks,linkcolor=textRed,citecolor=textRed,urlcolor=textBlue}
% THEOREMS

\newtheorem{theorem}{Theorem}
\newtheorem{definition}{Definition}
\newtheorem{example}{Example}
\newtheorem{corollary}{Corollary}[theorem]
\newtheorem{lemma}[theorem]{Lemma}

% SHORTHANDS

\newcommand*{\Th}{^{\textrm{th}}}
\newcommand*{\WLoG}{Without loss of generality}
\newcommand*{\wLoG}{without loss of generality}
\newcommand*{\wrt}{with respect to}

% SYMBOLS

\let\eps\epsilon
\newcommand*{\defeq}{:=}

\newcommand*{\Ahat}{\widehat{A}}
\newcommand*{\Bhat}{\widehat{B}}
\newcommand*{\Chat}{\widehat{C}}
\newcommand*{\Jhat}{\widehat{J}}
\newcommand*{\Shat}{\widehat{S}}
\newcommand*{\Yhat}{\widehat{Y}}
\newcommand*{\bhat}{\widehat{b}}
\newcommand*{\what}{\widehat{w}}
\newcommand*{\xhat}{\widehat{x}}
\newcommand*{\yhat}{\widehat{y}}
\newcommand*{\zhat}{\widehat{z}}

\newcommand*{\Btild}{\widetilde{B}}
\newcommand*{\Ctild}{\widetilde{C}}
\newcommand*{\Jtild}{\widetilde{J}}
\newcommand*{\Stild}{\widetilde{S}}
\newcommand*{\Ytild}{\widetilde{Y}}
\newcommand*{\btild}{\widetilde{b}}
\newcommand*{\wtild}{\widetilde{w}}
\newcommand*{\xtild}{\widetilde{x}}
\newcommand*{\ytild}{\widetilde{y}}
\newcommand*{\ztild}{\widetilde{z}}

\newcommand*{\Fcal}{\mathcal{F}}
\newcommand*{\Tcal}{\mathcal{T}}

% MATH

\newcommand*{\floor}[1]{\left\lfloor #1 \right\rfloor}
\newcommand*{\smallfloor}[1]{\lfloor #1 \rfloor}
\newcommand*{\ceil}[1]{\left\lceil #1 \right\rceil}
\newcommand*{\smallceil}[1]{\lceil #1 \rceil}
\newcommand*{\abs}[1]{\left\lvert #1 \right\rvert}
\newcommand*{\smallabs}[1]{\lvert #1 \rvert}
\newcommand*{\norm}[1]{\left\lVert #1 \right\rVert}
\newcommand*{\smallnorm}[1]{\lVert #1 \rVert}
\newcommand*{\Z}{\mathbb{Z}}
\DeclareMathOperator*{\E}{E}
\DeclareMathOperator*{\Var}{Var}
\DeclareMathOperator*{\argmin}{argmin}
\DeclareMathOperator*{\argmax}{argmax}
\DeclareMathOperator{\poly}{poly}
\DeclareMathOperator{\opt}{opt}
\DeclareMathOperator{\support}{support}
\newcommand*{\OPT}{\mathrm{OPT}}


\author{\empty}
\date{\empty}

\title{Chapter 2: Real numbers}

\begin{document}

\maketitle
\setlength{\parskip}{0.2em}

\section{Groups}

\begin{definition}[Group]
Let $G$ be a non-empty set and $\circ: G \times G \to G$ be a binary operator.
Then $(G, \circ)$ is a group iff all of the following hold:
\begin{enumerate}
\item Associativity: $a \circ (b \circ c) = (a \circ b) \circ c$ for all $a, b, c \in G$.
\item Identity exists: $\exists e \in G$ such that $\forall a \in G$, $e \circ a = a \circ e = a$.
    Such an $e$ is called an \emph{identity} of $(G, \circ)$.
    We can prove that the identity is unique.
\item Inverses exist: Let $e$ be an identity of $(G, \circ)$. Then $\forall a \in G$,
    $(\exists \ell \in G, \ell \circ a = e)$ and $(\exists r \in G, a \circ r = e)$.
    $\ell$ is called a \emph{left inverse} of $a$. $r$ is called a \emph{right inverse} of $a$.
\end{enumerate}
$(G, \circ)$ is called \emph{symmetric}, \emph{commutative}, or \emph{abelian} iff
$\forall a \in G$, $\forall b \in G$, $a \circ b = b \circ a$.
\end{definition}

\begin{lemma}
In a group $(G, \circ)$, the identity is unique and each element has a unique inverse.
\end{lemma}
\begin{proof}
Let $e_1$ and $e_2$ be identities of $(G, \circ)$.
Then $e_1 \circ e_2 = e_1$, since $e_2$ is an identity, and $e_2 \circ e_1 = e_2$, since $e_1$ is an identity.
Hence, $e_1 = e_2$.

Let $\ell$ be a left inverse and $r$ be a right inverse of $a \in G$. Then
\[ \ell = \ell \circ e = \ell \circ (a \circ r) = (\ell \circ a) \circ r = e \circ r = r. \]
Hence, every left inverse equals every right inverse. Hence, they are all equal.
\end{proof}

\begin{definition}[Standard operators]
If we use $+$ as a group operator, we denote identity as 0 and inverse of $g$ as $-g$.
If we use $\times$ as a group operator, we denote identity as 1 and inverse of $g$ as $g^{-1}$.
$a - b \defeq a + (-b)$. $a / b \defeq ab^{-1}$.
\end{definition}

\begin{definition}
Let $(G, \times)$ be a group. Then for any $n \in \mathbb{Z}$ and any $g \in G$, define
\[ g^n = \begin{cases}
g \times g \times \ldots \times g \; (n \textrm{ times}) & \textrm{if } n > 0
\\ 1 & \textrm{if } n = 0
\\ g^{-1} \times g^{-1} \times \ldots \times g^{-1} \; (-n \textrm{ times}) & \textrm{if } n < 0
\end{cases}. \]
\end{definition}

\begin{lemma}[Basic properties]
Let $(G, \cdot)$ be a group. Let $a, b \in G$ and $m, n \in \mathbb{Z}$.
\begin{enumerate}
\item $(ab)^{-1} = b^{-1}a^{-1}$.
\item $(a^{-1})^{-1} = a$.
\item $a^ma^n = a^{m+n}$.
\item $(a^m)^n = a^{mn}$.
\item If $G$ is symmetric, $(ab)^n = a^nb^n$.
\end{enumerate}
\end{lemma}

\section{Fields}

\begin{definition}[Field]
$(F, +, \times)$ is a field iff it satisfies all of the following:
\begin{enumerate}
\item $(F, +)$ is a symmetric group. It's identity is denoted as 0.
\item $(F - \{0\}, \times)$ is a symmetric group. It's identity is denoted as 1.
\item Distributivity: $a(b + c) = ab + ac$ and $(a+b)c = ac + bc$.
\end{enumerate}
\end{definition}

\begin{lemma}[Basic properties]
\label{thm:field-basics}
Let $(F, +, \times)$ be a field. Let $a, b \in F$.
\begin{enumerate}
\item $a0 = 0a = 0$.
\item $a(-b) = (-a)b = -(ab)$.
\item $(-a)(-b) = ab$.
\item \label{item:field-basics:zero-prod}$ab = 0 \iff (a = 0 \textrm{ or } b = 0)$.
\item $(-a)^{-1} = -a^{-1}$.
\end{enumerate}
\end{lemma}
\begin{proof}[Proof sketches]
\leavevmode
\begin{enumerate}
\item $a0 = a(0 + 0) = a0 + a0$.
\item $0 = a0 = a(b + (-b)) = ab + a(-b)$.
\item $(-a)(-b) = a(-(-b)) = ab$.
\item Suppose $a \neq 0$. Then $ab = 0 \implies b = a^{-1}0 = 0$.
\item $(-1)(-1) = 1$, so $(-1)^{-1} = -1$. $(-a)^{-1} = ((-1)a)^{-1} = (-1)^{-1}a^{-1} = -a^{-1}$.
\end{enumerate}
\end{proof}

\section{Partial Orders}

\begin{definition}[Partial and total orders]
Let $L$ be a set and let $\le$ be a binary predicate over $L \times L$.
Then $(L, \le)$ is called a \emph{partial order} (aka poset) iff all of the following hold:
\begin{enumerate}
\item Reflexivity: $\forall a \in L$, $a \le a$.
\item Anti-symmetry: $a \le b$ and $b \le a$ $\implies a = b$.
\item Transitivity: $a \le b$ and $b \le c$ $\implies a \le c$.
\end{enumerate}
Additionally, if $\forall a, b \in L$, we have $a \le b$ or $b \le a$,
then $(L, \le)$ is called a \emph{total order}.

$a < b \defiff (a \le b \textrm{ and } a \neq b)$.
$a \ge b \defiff b \le a$. $a > b \defiff b < a$.
\end{definition}

\begin{definition}[Upper and lower bound]
Let $(L, \le)$ be a poset. Let $S \subseteq L$.
\begin{enumerate}
\item $u \in L$ is an \emph{upper bound} for $S$ iff $s \le u$ for all $s \in S$.
    $S$ is called \emph{upper-bounded} iff an upper bound exists for $S$.
\item $u \in L$ is a \emph{least upper bound} or \emph{supremum} for $S$ (denoted $\sup(S)$)
    iff $u$ is an upper bound for $S$ and for every upper bound $v$ of $S$, we have $u \le v$.
\item $u \in L$ is a \emph{lower bound} for $S$ iff $u \le s$ for all $s \in S$.
    $S$ is called \emph{lower-bounded} iff a lower bound exists for $S$.
\item $u \in L$ is a \emph{greatest lower bound} or \emph{infimum} for $S$ (denoted $\inf(S)$)
    iff $u$ is a lower bound for $S$ and for every lower bound $v$ of $S$, we have $v \le u$.
\item $S$ is called \emph{bounded} iff it has a lower bound and an upper bound.
\end{enumerate}
\end{definition}

\begin{lemma}
$\sup(S)$, if it exists, is unique.
$\inf(S)$, if it exists, is unique.
\end{lemma}

\section{Ordered Field}

\begin{definition}[Ordered field]
\label{defn:ord-field}
Let $(F, +, \times)$ be a field.
$(F, +, \times, \le)$ is an ordered field iff all of the following hold:
\begin{enumerate}
\item $(F, \le)$ is a total order.
\item $a \le b \implies (\forall c \in F, a + c \le b + c)$.
\item $a \ge 0$ and $b \ge 0$ $\implies ab \ge 0$.
\end{enumerate}
\end{definition}

\begin{lemma}[Strict inequalities]
Let $(F, +, \times, \le)$ be an ordered field. Then
\begin{enumerate}
\item $a < b$ and $b < c$ $\implies a < c$.
\item $a < b \implies (\forall c \in F, a + c < b + c)$.
\item $a > 0$ and $b > 0$ $\implies ab > 0$.
\end{enumerate}
\end{lemma}
\begin{comment}
\leavevmode
\begin{enumerate}
\item $a = c \implies (a < b \textrm{ and } b < a) \implies \bot$.
\item $a + c = b + c \implies a = b \implies \bot$.
\item $ab = 0 \implies (a = 0 \textrm{ or } b = 0) \implies \bot$ by
    \cref{thm:field-basics}.\ref{item:field-basics:zero-prod}.
\end{enumerate}
\end{comment}

\begin{definition}[Field with positives (non-standard terminology)]
\label{defn:field-with-positives}
Let $(F, +, \times)$ be a field. Let $P \subseteq F$.
$(F, +, \times, P)$ is called a \emph{field with positives} iff
\begin{enumerate}
\item $a, b \in P \implies a + b \in P$.
\item $a, b \in P \implies ab \in P$.
\item $\forall a \in F$, exactly one of these is true:
    $a = 0$, $a \in P$, $-a \in P$.
\end{enumerate}
\end{definition}

The following two results state that either of \cref{defn:ord-field,defn:field-with-positives}
could be used to define the other.

\begin{lemma}
Let $(F, +, \times, P)$ be a field with positives.
Let $a \le b \defiff (b - a \in P \textrm{ or } b = a)$.
Then $(F, +, \times, \le)$ is an ordered field.
\end{lemma}

\begin{lemma}
Let $(F, +, \times, \le)$ be an ordered field. Let $P \defeq \{x \in F: x > 0\}$.
Then $(F, +, \times, P)$ is a field with positives.
\end{lemma}

\begin{lemma}
Let $(F, +, \times, \le)$ be an ordered field.
\begin{enumerate}
\item $a_1 \le b_1$ and $a_2 \le b_2 \implies a_1 + a_2 \le b_1 + b_2$.
\item $a^2 \ge 0$ and $(a^2 = 0 \iff a = 0)$.
\item $1 > 0$.
\item $ab > 0 \implies (a > 0 \textrm{ and } b > 0) \textrm{ or } (a < 0 \textrm{ and } b < 0)$.
\item $a > 0 \implies a^{-1} > 0$.
\end{enumerate}
\end{lemma}

\begin{lemma}
$(\forall \eps > 0, a \le \eps) \implies a \le 0$.
\end{lemma}

\begin{definition}
$\displaystyle |a| \defeq \begin{cases} a & \textrm{ if } a \ge 0
\\ -a & \textrm{ if } a < 0 \end{cases}$.
\end{definition}

\begin{lemma}
\label{thm:abs-basics}
Let $(F, +, \times, \le)$ be an ordered field.
\begin{enumerate}
\item $|a| \ge 0$ and $(|a| = 0 \iff a = 0)$.
\item $|-a| = |a|$.
\item $|a| \ge a$ and $|a| \ge -a$.
\item \label{item:abs-basics:dbl}Let $c \ge 0$. Then $|a| \le c \iff -c \le a \le c$.
\item $-|a| \le a \le |a|$.
\item $|ab| = |a||b|$.
\item For $a \neq 0$, $|a^{-1}| = |a|^{-1}$.
\end{enumerate}
\end{lemma}

\begin{lemma}[Triangle inequalities]
$||a|-|b|| \le |a + b| \le |a| + |b|$.
\end{lemma}
\begin{proof}
$-|a| \le a \le |a|$ and $-|b| \le b \le |b|$.
Add these to get $-(|a|+|b|) \le a + b \le |a| + |b|$.
By \cref{thm:abs-basics}.\ref{item:abs-basics:dbl}, we get $|a + b| \le |a| + |b|$.

By previous result, $|a| = |(a+b) + (-b)| \le |a+b| + |b|$, so $|a|-|b| \le |a+b|$.
Also, $|b| = |(a+b) + (-a)| \le |a+b| + |a|$, so $-|a+b| \le |a|-|b|$.
Hence, $-|a+b| \le |a|-|b| \le |a+b|$.
By \cref{thm:abs-basics}.\ref{item:abs-basics:dbl}, we get $||a|-|b|| \le |a+b|$.
\end{proof}

\begin{definition}
Define $\max$ and $\min$ as
\begin{align*}
\max(x, y) &\defeq \begin{cases}x & \textrm{ if } x \ge y \\ y & \textrm{ if } y > x\end{cases}.
& \min(x, y) &\defeq \begin{cases}y & \textrm{ if } x \ge y \\ x & \textrm{ if } y > x\end{cases}.
\end{align*}
\end{definition}

\begin{lemma}
$\max$ and $\min$ are symmetric and associative, i.e.,
$\max(a, b) = \max(b, a)$, $\max(\max(a, b), c) = \max(a, \max(b, c))$.
$\min(a, b) = \min(b, a)$, and $\min(\min(a, b), c) = \min(a, \min(b, c))$.
\end{lemma}

\section{Supremum, Infimum, and Real Numbers}

\begin{definition}
The set of real numbers is an ordered field $(\mathbb{R}, +, \times, \le)$
in which every set with an upper bound has a supremum.
(In fact, such an ordered field is unique, but proving that is beyond the scope of the course/book.)
\end{definition}

\begin{lemma}
Let $S \subseteq \mathbb{R}$ and $S' = \{-x: x \in S\}$.
Then $\inf(S) = -\sup(S')$ and $\sup(S) = -\inf(S')$.
\end{lemma}

\begin{lemma}
Let $S \subseteq \mathbb{R}$. Then for any $\alpha \in \mathbb{R}$,
$(\forall x \in S, x \le \alpha) \iff \sup(S) \le \alpha$,
and $(\forall x \in S, x \ge \alpha) \iff \inf(S) \ge \alpha$.
\end{lemma}

\begin{lemma}
Let $A, B \subseteq \mathbb{R}$. Then
\begin{enumerate}
\item $\sup(A \cup B) = \max(\sup(A), \sup(B))$ and $\inf(A \cup B) = \min(\inf(A), \inf(B))$.
\item $A \subseteq B \implies \inf(B) \le \inf(A) \le \sup(A) \le \sup(B)$.
\end{enumerate}
\end{lemma}

\begin{definition}
Let $f: D \to \mathbb{R}$. Then $\sup_{x \in D} f(x) \defeq \sup(f(D))$.
\end{definition}

\begin{lemma}[Archimedian Properties, floor, and ceil]
Let $x \in \mathbb{R}_{> 0}$. Then
\begin{enumerate}
\item $\exists n \in \mathbb{N}$ such that $x < n$.
\item $\exists n \in \mathbb{N}$ such that $1/n < x$.
\item There is a unique $n \in \mathbb{N} \cup \{0\}$ such that $n \le x < n+1$.
    (We denote $n$ as $\smallfloor{x}$.)
\item There is a unique $n \in \mathbb{N}$ such that $n-1 < x \le n$.
    (We denote $n$ as $\smallceil{x}$.)
\end{enumerate}
\end{lemma}
\begin{proof}
\begin{enumerate}
\item Suppose this is not true. Then $x$ is an upper-bound of $\mathbb{N}$.
    By completeness property of $\mathbb{R}$, $u \defeq \sup(\mathbb{N})$ exists.
    Hence, $u-1$ is not an upper-bound of $\mathbb{N}$, and so $\exists m \in \mathbb{N}$
    such that $u-1 < m$. Hence, $u \le m+1$. This is a contradiction, since $m+1 \in \mathbb{N}$.
\item $\exists n \in \mathbb{N}$ such that $n > 1/x$. Hence, $1/n < x$.
\item Let $T \defeq \{m \in \mathbb{N}: x < m\}$. By part 1, $T \neq \emptyset$.
    By well-ordering of $\mathbb{N}$, $T$ has a least element $t$.
    Then $t-1 \not\in T$, so $t-1 \le x$. Hence, $t-1 \le x < t$. Set $n = t-1$.
\item Let $T \defeq \{m \in \mathbb{N}: x \le m\}$. By part 1, $T \neq \emptyset$.
    By well-ordering of $\mathbb{N}$, $T$ has a least element $n$.
    Then $n-1 \not\in T$, so $n-1 < x$.
\end{enumerate}
\end{proof}

\begin{lemma}[$\mathbb{Q}$ is dense in $\mathbb{R}$]
Let $x, y \in \mathbb{R}$ and $x < y$. Then $\exists z \in \mathbb{Q}$ such that $x < z < y$.
\end{lemma}
\begin{proof}
By Archimedian property, $\exists n \in \mathbb{N}$ such that $1/n < y-x$. Then $nx + 1 < y$.
Let $k \defeq \floor{nx}+1$. Then $nx < \floor{nx} + 1 \le nx + 1 < ny$.
Hence, $x < k/n < y$.
\end{proof}

\begin{lemma}[Principle of iterated suprema]
Let $X$ and $Y$ be non-empty sets and $f: X \times Y \to \mathbb{R}$ be upper-bounded. Then
\[ \sup_{(x, y) \in X \times Y} f(x, y)
= \sup_{x \in X}\sup_{y \in Y} f(x, y)
= \sup_{y \in Y}\sup_{x \in X} f(x, y). \]
\end{lemma}
\begin{proof}
We will prove the first equality, since the second's proof is similar. Let
\begin{align*}
g(x) &\defeq \sup_{y \in Y} f(x, y)
& \alpha &\defeq \sup_{x \in X} g(x)
& \beta &\defeq \sup_{(x,y) \in X \times Y} f(x, y)
\end{align*}
We need to show that $\alpha = \beta$. For any $z \in \mathbb{R}$,
\begin{align*}
& \beta \le z
\\ \iff &\forall (x, y) \in X \times Y, f(x, y) \le z
\\ \iff &\forall x \in X, \forall y \in Y, f(x, y) \le z
\\ \iff &\forall x \in X, g(x) \le z
\\ \iff &\alpha \le z.
\qedhere
\end{align*}
\end{proof}

\begin{lemma}
Let $X$ and $Y$ be non-empty sets and $f: X \times Y \to \mathbb{R}$ be bounded.
Then $\alpha \le \beta$, where
\begin{align*}
\alpha &\defeq \sup_{x \in X}\inf_{y \in Y} f(x, y),
& \beta &\defeq \inf_{x \in X}\sup_{y \in Y} f(x, y).
\end{align*}
\end{lemma}
(Hint: Consider the special case where $X$ and $Y$ are finite, and then generalize.)
\begin{proof}
Pick any $\eps > 0$.
Then $\exists x^* \in X$ such that $\inf_{y \in Y} f(x^*, y) \ge \alpha - \eps$,
and $\exists y^* \in Y$ such that $\sup_{x \in X} f(x, y*) \le \beta + \eps$. Hence,
\[ \alpha - \eps \le \inf_{y \in Y} f(x^*, y) \le f(x^*, y^*) \le \sup_{x \in X} f(x, y^*) \le \beta + \eps. \]
Hence, $\forall \eps > 0$, we get $\alpha - \beta \le 2\eps$.
Hence, $\alpha - \beta \le 0$.
\end{proof}

\section{Intervals}

\begin{definition}[Interval]
Let $a, b \in \mathbb{R}$, such that $a \le b$.
\\ The following are called \emph{closed intervals}:
\begin{enumerate}
\item $[a, b] \defeq \{x \in \mathbb{R}: a \le x \le b\}$.
\item $[a, \infty) \defeq \{x \in \mathbb{R}: a \le x\}$.
\item $(-\infty, b] \defeq \{x \in \mathbb{R}: x \le b\}$.
\end{enumerate}
The following are called \emph{open intervals}:
\begin{enumerate}
\item $(a, b) \defeq \{x \in \mathbb{R}: a < x < b\}$.
\item $(a, \infty) \defeq \{x \in \mathbb{R}: a < x\}$.
\item $(-\infty, b) \defeq \{x \in \mathbb{R}: x < b\}$.
\end{enumerate}
The following are called \emph{half-open intervals}:
\begin{enumerate}
\item $[a, b) \defeq \{x \in \mathbb{R}: a \le x < b\}$.
\item $(a, b] \defeq \{x \in \mathbb{R}: a < x \le b\}$.
\end{enumerate}
$(-\infty, \infty) \defeq \mathbb{R}$ is both an open and closed interval.
\end{definition}

\begin{lemma}
Let $S \subseteq \mathbb{R}$ be a non-empty set.
Then $S$ is an interval iff $\forall x \in S, \forall y \in S, (x < y \implies [x, y] \subseteq S)$.
\end{lemma}
\begin{proof}[Proof sketch]
Let $a \defeq \inf(S)$ and $b \defeq \sup(S)$.
(Let $a \defeq -\infty$ if $S$ is not lower bounded,
and $b \defeq \infty$ if $S$ is not upper bounded.)

Pick any $z \in (a, b)$. Since $z$ is not a lower or upper bound of $S$,
$\exists x \in S$ such that $x < z$, and $\exists y \in S$ such that $y < z$.
Then $z \in [x, y]$ and $[x, y] \subseteq S$, so $z \in S$. Hence, $(a, b) \subseteq S$.
Also, $S \subseteq [a, b]$ (where $[a, \infty] \defeq [a, \infty)$ and $[-\infty, b] \defeq (-\infty, b]$).
\end{proof}

\begin{lemma}
Let $[a_i]_{i \in \mathbb{N}}$ and $[b_i]_{i \in \mathbb{N}}$ be infinite sequences
and $I_n \defeq [a_n, b_n]$ for all $n \in \mathbb{N}$. Then
\end{lemma}

\subsection{Nested Intervals}

Let $[a_n]_{n \in \mathbb{N}}$ and $[b_n]_{n \in \mathbb{N}}$ be two sequences of real numbers
such that $a_i \le a_{i+1} \le b_{i+1} \le b_i$ for all $i \in \mathbb{N}$.
Let $I_n \defeq [a_n, b_n]$ for $n \in \mathbb{N}$.
Let $I \defeq \cap_{n \in \mathbb{N}} I_n$.

$\forall n \in \mathbb{N}$, $a_1 \le a_n \le b_n \le b_1$.
Hence, sequences $[a_n]_{n \in \mathbb{N}}$ and $[b_n]_{n \in \mathbb{N}}$ are bounded.
Let $a \defeq \sup_{n \in \mathbb{N}} a_n$ and $b \defeq \inf_{n \in \mathbb{N}} b_n$.
Let $\ell \defeq \inf_{n \in \mathbb{N}} (b_n - a_n)$
($\ell \ge 0$, since $b_n - a_n \ge 0$ for all $n \in \mathbb{N}$).

\begin{lemma}
$I = [a, b]$.
\end{lemma}
\begin{proof}
Let $z \in \mathbb{R}$.
\begin{align*}
& z \in [a, b]
\\ \iff & (\forall n \in \mathbb{N}, a_n \le z) \textrm{ and } (\forall n \in \mathbb{N}, b_n \le z)
\\ \iff & (\forall n \in \mathbb{N}, a_n \le z \le b_n)
\\ \iff & (\forall n \in \mathbb{N}, z \in I_n)
\\ \iff & z \in I.
\qedhere
\end{align*}
\end{proof}

\begin{lemma}
$\ell \defeq b-a$.
\end{lemma}
\begin{proof}
\begin{align*}
& (\forall n \in \mathbb{N}, a_n \le a) \textrm{ and } (\forall n \in \mathbb{N}, b \le b_n)
\\ \implies & (\forall n \in \mathbb{N}, a_n \le a \textrm{ and } b \le b_n)
\\ \implies & (\forall n \in \mathbb{N}, b - a \le b_n - a_n )
\\ \implies & b - a \le \inf_{n \in \mathbb{N}} (b_n - a_n) = \ell.
\end{align*}
Let $\eps > 0$. Then $\exists p \in \mathbb{N}$ such that $a_p \ge a - \eps$,
and $\exists q \in \mathbb{N}$ such that $b_q \le b + \eps$.
Let $r \defeq \max(p, q)$. Then
\[ a - \eps \le a_p \le a_r \le b_r \le b_q \le b + \eps. \]
Hence,
\[ \ell \le b_r - a_r \le (b + \eps) - (a - \eps) = (b-a) + 2\eps. \]
Since this is true for all $\eps > 0$, we get $\ell \le b-a$.
\end{proof}

\end{document}

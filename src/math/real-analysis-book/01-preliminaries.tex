\documentclass[a4paper, 12pt, fleqn]{article}

\usepackage{amsmath,amsthm,amssymb}
\usepackage[margin=1in]{geometry}
\usepackage{xcolor}
\usepackage{array}
\usepackage{comment}
\usepackage{booktabs}
\usepackage[bookmarksnumbered=true,hypertexnames=false]{hyperref}
%\usepackage{algorithm, algpseudocode}
\usepackage[capitalize,sort]{cleveref}

%\def\colorscheme{dark}
\def\colorschemesepia{sepia}
\def\colorschemedark{dark}
\def\colorschemelight{light}

\ifx\colorscheme\undefined
\let\colorscheme\colorschemelight
\fi

\ifx\colorscheme\colorschemelight
\colorlet{textColor}{black}
\colorlet{bgColor}{white}
\fi

\ifx\colorscheme\colorschemesepia
\definecolor{textColor}{HTML}{433423}
\definecolor{bgColor}{HTML}{fbf0da}
\fi

\ifx\colorscheme\colorschemedark
\definecolor{textColor}{HTML}{bdc1c6}
\definecolor{bgColor}{HTML}{202124}
\definecolor{textBlue}{HTML}{8ab4f8}
\definecolor{textRed}{HTML}{f9968b}
\colorlet{bgRed}{red!40!black}
\colorlet{bgBlue}{red!40!black}
\else
\colorlet{textBlue}{blue!50!black}
\colorlet{textRed}{red!50!black}
\fi

\ifx\colorscheme\colorschemelight\else
\pagecolor{bgColor}
\color{textColor}
\fi

\hypersetup{colorlinks,linkcolor=textRed,citecolor=textRed,urlcolor=textBlue}
% THEOREMS

\newtheorem{theorem}{Theorem}
\newtheorem{definition}{Definition}
\newtheorem{example}{Example}
\newtheorem{corollary}{Corollary}[theorem]
\newtheorem{lemma}[theorem]{Lemma}
\newtheorem{claim}[theorem]{Claim}

% SHORTHANDS

\newcommand*{\Th}{^{\textrm{th}}}
\newcommand*{\WLoG}{Without loss of generality}
\newcommand*{\wLoG}{without loss of generality}
\newcommand*{\wrt}{with respect to}
\let\opname\operatorname

% SYMBOLS

\let\eps\epsilon
\newcommand*{\defeq}{:=}

\newcommand*{\alphahat}{\widehat{\alpha}}
\newcommand*{\betahat}{\widehat{\beta}}
\newcommand*{\Ahat}{\widehat{A}}
\newcommand*{\Bhat}{\widehat{B}}
\newcommand*{\Chat}{\widehat{C}}
\newcommand*{\Jhat}{\widehat{J}}
\newcommand*{\Shat}{\widehat{S}}
\newcommand*{\Yhat}{\widehat{Y}}
\newcommand*{\bhat}{\widehat{b}}
\newcommand*{\shat}{\widehat{s}}
\newcommand*{\what}{\widehat{w}}
\newcommand*{\xhat}{\widehat{x}}
\newcommand*{\yhat}{\widehat{y}}
\newcommand*{\zhat}{\widehat{z}}

\newcommand*{\Btild}{\widetilde{B}}
\newcommand*{\Ctild}{\widetilde{C}}
\newcommand*{\Jtild}{\widetilde{J}}
\newcommand*{\Stild}{\widetilde{S}}
\newcommand*{\Ytild}{\widetilde{Y}}
\newcommand*{\btild}{\widetilde{b}}
\newcommand*{\stild}{\widetilde{s}}
\newcommand*{\wtild}{\widetilde{w}}
\newcommand*{\xtild}{\widetilde{x}}
\newcommand*{\ytild}{\widetilde{y}}
\newcommand*{\ztild}{\widetilde{z}}

\newcommand*{\Fcal}{\mathcal{F}}
\newcommand*{\Tcal}{\mathcal{T}}

\newcommand*{\ebar}{\overline{e}}
\newcommand*{\Xbar}{\overline{X}}
\newcommand*{\xbar}{\overline{x}}
\newcommand*{\Ybar}{\overline{Y}}
\newcommand*{\ybar}{\overline{y}}

% MATH

\newcommand*{\floor}[1]{\left\lfloor #1 \right\rfloor}
\newcommand*{\smallfloor}[1]{\lfloor #1 \rfloor}
\newcommand*{\ceil}[1]{\left\lceil #1 \right\rceil}
\newcommand*{\smallceil}[1]{\lceil #1 \rceil}
\newcommand*{\abs}[1]{\left\lvert #1 \right\rvert}
\newcommand*{\smallabs}[1]{\lvert #1 \rvert}
\newcommand*{\norm}[1]{\left\lVert #1 \right\rVert}
\newcommand*{\smallnorm}[1]{\lVert #1 \rVert}
\newcommand*{\Z}{\mathbb{Z}}
\DeclareMathOperator*{\E}{E}
\DeclareMathOperator*{\Var}{Var}
\DeclareMathOperator*{\argmin}{argmin}
\DeclareMathOperator*{\argmax}{argmax}
\DeclareMathOperator{\poly}{poly}
\DeclareMathOperator{\opt}{opt}
\DeclareMathOperator{\argopt}{argopt}
\DeclareMathOperator{\support}{support}
\newcommand*{\OPT}{\mathrm{OPT}}


\DeclareMathOperator{\idFunc}{id}

\author{\empty}
\date{\empty}

\title{Chapter 1: Preliminaries}

\begin{document}

\maketitle
\setlength{\parskip}{0.2em}

\section{Sets}

\begin{definition}[Set basics]
\leavevmode
\begin{enumerate}
\item $A \subseteq B \defiff (\forall x \in A, x \in B)$.
\item $A = B \defiff (A \subseteq B \land B \subseteq A)$.
\item $A \subset B \defiff (A \subseteq B \land B \not\subseteq A)$.
\item $A \cup B \defeq \{x: x \in A \textrm{ or } x \in B\}$.
\item $A \cap B \defeq \{x \in A: x \in B\}$.
\item $A \setminus B \defeq \{x \in A: x \not\in B\}$.
\item $\displaystyle \bigcup_{i \in I} A_i \defeq \{x: \exists i \in I \textrm{ such that } x \in A_i\}$.
\item $\displaystyle \bigcap_{i \in I} A_i \defeq \{x: \forall i \in I, x \in A_i\}$.
\item $A \times B \defeq \{(x, y): x \in A, y \in B\}$.
\item $\displaystyle \prod_{i=1}^n A_i \defeq \{(x_1, x_2, \ldots, x_n): x_i \in A_i \textrm{ for all } i\}$.
\end{enumerate}
\end{definition}

\begin{theorem}
\begin{enumerate}
\item $A \subseteq B \iff A \cap B = A \iff A \cup B = B$.
\item $A \setminus (B \cup C) = (A \setminus B) \cap (A \setminus C)$.
\item $A \setminus (B \cap C) = (A \setminus B) \cup (A \setminus C)$.
\item $A \cup (B \cup C) = (A \cap B) \cup (A \cap C)$.
\item $A \cap (B \cap C) = (A \cup B) \cap (A \cup C)$.
\end{enumerate}
\end{theorem}

\section{Relations and Functions}

\begin{definition}[Relation and function]
A relation $R$ between $A$ and $B$ is a subset of $A \times B$.
A function $f: A \to B$ is a relation between $A$ and $B$ such that
\[ (a, b_1) \in f \textrm{ and } (a, b_2) \in f \implies b_1 = b_2. \]
$D(f) \defeq A$ (called \emph{domain} of $f$),
and $R(f) \defeq B$ (called \emph{range} of $f$).
\end{definition}

\begin{lemma}
Let $f: A \to B$ and $g: A \to B$.
Then $f = g \iff (\forall x \in A, f(x) = g(x))$.
\end{lemma}

\begin{definition}[Image and reverse image]
Let $f: A \to B$ be a function.
\begin{enumerate}
\item For $X \subseteq A$, $f(X) \defeq \{f(x): x \in X\}$ is called the \emph{image} of $X$ under $f$.
    \\ Equivalently, $y \in f(X) \iff (\exists x \in X, f(x) = y)$.
\item For $Y \subseteq B$, $f^{-1}(Y) = \{x: f(x) \in Y\}$ is called the \emph{inverse image}
    of $Y$ under $f$. Equivalently, $x \in f^{-1}(Y) \iff f(x) \in Y$.
\end{enumerate}
\end{definition}

\begin{lemma}
Let $f: A \to B$. Let $X_1, X_2 \subseteq A$ and $Y_1, Y_2 \subseteq B$.
\begin{enumerate}
\item $f(X_1 \cup X_2) = f(X_1) \cup f(X_2)$.
\item $f(X_1 \cap X_2) \subseteq f(X_1) \cap f(X_2)$.
\item $f^{-1}(Y_1 \cup Y_2) = f^{-1}(Y_1) \cup f^{-1}(Y_2)$.
\item $f^{-1}(Y_1 \cap Y_2) = f^{-1}(Y_1) \cap f^{-1}(Y_2)$.
\end{enumerate}
\end{lemma}

\begin{definition}[Composition]
\label{defn:func-compose}
For functions $f: A \to B$ and $g: B \to C$,
$g \circ f: A \to C$ is the defined as $(g \circ f)(x) = g(f(x))$.
\end{definition}

\begin{definition}[Injection and surjection]
Let $f: A \to B$.
\begin{enumerate}
\item $f$ is injective (aka one-to-one)
    $\defiff \forall x_1 \in A, \forall x_2 \in A, (f(x_1) = f(x_2) \implies x_1 = x_2)$.
\item $f$ is surjective (aka onto)
    $\defiff \forall y \in B, \exists x \in A, f(x) = y$.
\end{enumerate}
\end{definition}

\begin{lemma}[Composition]
\label{thm:inj-surj-comp}
Let $f: A \to B$ and $g: B \to C$.
\begin{enumerate}
\item If $f$ and $g$ are injective, then $g \circ f$ is injective.
\item If $g \circ f$ is injective, then $f$ is injective.
\item If $f$ and $g$ are surjective, then $g \circ f$ is surjective.
\item If $g \circ f$ is surjective, then $g$ is surjective.
\end{enumerate}
\end{lemma}

\begin{definition}[Identity]
The identity function $\idFunc_A: A \to A$ is given by $\idFunc_A(x) = x$ for all $x \in A$.
\end{definition}

\begin{definition}[Bijection]
A function $f: A \to B$ is a bijection iff (the following are equivalent):
\begin{enumerate}
\item \label{item:bij:inj-surj}$f$ is injective and surjective.
\item \label{item:bij:inv}$\exists g: B \to A$ such that $g \circ f = \idFunc_A$ and $f \circ g = \idFunc_B$.
    (Then $g$ is called the \emph{inverse} of $f$, and is denoted by $f^{-1}$.)
\end{enumerate}
\end{definition}
\begin{proof}[Proof sketch of equivalence]
If $f$ is injective and surjective, for each $y \in B$, there is a unique $x \in A$
such that $f(x) = y$. Define $g(y) = x$ and show condition \ref{item:bij:inv}.
To show that condition \ref{item:bij:inv} implies condition \ref{item:bij:inj-surj},
use \cref{thm:inj-surj-comp}.
\end{proof}

\begin{definition}[Restriction]
Let $f: A \to B$ be a function. Let $X \subseteq A$. Then $f|X$ is a function from $X$ to $B$
such that $(f|X)(x) = f(x)$ for all $x \in X$.
\end{definition}

\end{document}

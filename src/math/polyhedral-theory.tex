\documentclass[a4paper, 12pt, fleqn]{article}

\usepackage{amsmath,amsthm,amssymb}
\usepackage[margin=1in]{geometry}
\usepackage{xcolor}
\usepackage{array}
\usepackage{comment}
\usepackage{booktabs}
\usepackage{enumitem}
\usepackage[bookmarksnumbered=true,hypertexnames=false]{hyperref}
%\usepackage{algorithm, algpseudocode}
\usepackage[capitalize,sort]{cleveref}

%\def\colorscheme{dark}
\def\colorschemesepia{sepia}
\def\colorschemedark{dark}
\def\colorschemelight{light}

\ifx\colorscheme\undefined
\let\colorscheme\colorschemelight
\fi

\ifx\colorscheme\colorschemelight
\colorlet{textColor}{black}
\colorlet{bgColor}{white}
\fi

\ifx\colorscheme\colorschemesepia
\definecolor{textColor}{HTML}{433423}
\definecolor{bgColor}{HTML}{fbf0da}
\fi

\ifx\colorscheme\colorschemedark
\definecolor{textColor}{HTML}{bdc1c6}
\definecolor{bgColor}{HTML}{202124}
\definecolor{textBlue}{HTML}{8ab4f8}
\definecolor{textRed}{HTML}{f9968b}
\colorlet{bgRed}{red!40!black}
\colorlet{bgBlue}{red!40!black}
\else
\colorlet{textBlue}{blue!50!black}
\colorlet{textRed}{red!50!black}
\fi

\ifx\colorscheme\colorschemelight\else
\pagecolor{bgColor}
\color{textColor}
\fi

\hypersetup{colorlinks,linkcolor=textRed,citecolor=textRed,urlcolor=textBlue}
% THEOREMS

\newtheorem{theorem}{Theorem}
\newtheorem{definition}{Definition}
\newtheorem{example}{Example}
\newtheorem{corollary}{Corollary}[theorem]
\newtheorem{lemma}[theorem]{Lemma}
\newtheorem{claim}[theorem]{Claim}

% SHORTHANDS

\newcommand*{\Th}{^{\textrm{th}}}
\newcommand*{\WLoG}{Without loss of generality}
\newcommand*{\wLoG}{without loss of generality}
\newcommand*{\wrt}{with respect to}
\let\opname\operatorname

% SYMBOLS

\let\eps\epsilon
\newcommand*{\defeq}{:=}

\newcommand*{\alphahat}{\widehat{\alpha}}
\newcommand*{\betahat}{\widehat{\beta}}
\newcommand*{\Ahat}{\widehat{A}}
\newcommand*{\Bhat}{\widehat{B}}
\newcommand*{\Chat}{\widehat{C}}
\newcommand*{\Jhat}{\widehat{J}}
\newcommand*{\Shat}{\widehat{S}}
\newcommand*{\Yhat}{\widehat{Y}}
\newcommand*{\bhat}{\widehat{b}}
\newcommand*{\shat}{\widehat{s}}
\newcommand*{\what}{\widehat{w}}
\newcommand*{\xhat}{\widehat{x}}
\newcommand*{\yhat}{\widehat{y}}
\newcommand*{\zhat}{\widehat{z}}

\newcommand*{\Btild}{\widetilde{B}}
\newcommand*{\Ctild}{\widetilde{C}}
\newcommand*{\Jtild}{\widetilde{J}}
\newcommand*{\Stild}{\widetilde{S}}
\newcommand*{\Ytild}{\widetilde{Y}}
\newcommand*{\btild}{\widetilde{b}}
\newcommand*{\stild}{\widetilde{s}}
\newcommand*{\wtild}{\widetilde{w}}
\newcommand*{\xtild}{\widetilde{x}}
\newcommand*{\ytild}{\widetilde{y}}
\newcommand*{\ztild}{\widetilde{z}}

\newcommand*{\Fcal}{\mathcal{F}}
\newcommand*{\Tcal}{\mathcal{T}}

\newcommand*{\ebar}{\overline{e}}
\newcommand*{\Xbar}{\overline{X}}
\newcommand*{\xbar}{\overline{x}}
\newcommand*{\Ybar}{\overline{Y}}
\newcommand*{\ybar}{\overline{y}}

% MATH

\newcommand*{\floor}[1]{\left\lfloor #1 \right\rfloor}
\newcommand*{\smallfloor}[1]{\lfloor #1 \rfloor}
\newcommand*{\ceil}[1]{\left\lceil #1 \right\rceil}
\newcommand*{\smallceil}[1]{\lceil #1 \rceil}
\newcommand*{\abs}[1]{\left\lvert #1 \right\rvert}
\newcommand*{\smallabs}[1]{\lvert #1 \rvert}
\newcommand*{\norm}[1]{\left\lVert #1 \right\rVert}
\newcommand*{\smallnorm}[1]{\lVert #1 \rVert}
\newcommand*{\Z}{\mathbb{Z}}
\DeclareMathOperator*{\E}{E}
\DeclareMathOperator*{\Var}{Var}
\DeclareMathOperator*{\argmin}{argmin}
\DeclareMathOperator*{\argmax}{argmax}
\DeclareMathOperator{\poly}{poly}
\DeclareMathOperator{\opt}{opt}
\DeclareMathOperator{\argopt}{argopt}
\DeclareMathOperator{\support}{support}
\newcommand*{\OPT}{\mathrm{OPT}}


\newenvironment*{tightenum}{\begin{enumerate}[noitemsep]}{\end{enumerate}}
\DeclareMathOperator{\rank}{rank}
\DeclareMathOperator{\Span}{span}
\newcommand*{\LP}{\mathrm{LP}}
\newcommand*{\DLP}{\mathrm{DLP}}
\newcommand*{\karthikLec}[1]{http://karthik.ise.illinois.edu/courses/ie511/lectures-sp-21/lecture-#1.pdf}

\author{Eklavya Sharma}
\date{\empty}

\title{Polyhedral Theory}

\begin{document}

\maketitle
\setlength{\parskip}{0.2em}

Prerequisites:
\begin{tightenum}
\item Convex combination, strict convex combination, convex hull, convex set.
\item Vector spaces, rank of a set of vectors.
\end{tightenum}

\section{Definitions}

\begin{definition}[Line and Ray]
For $x \in \mathbb{R}^n$ and $d \in \mathbb{R}^n - \{0\}$,
a line is a set of the form $\{x + \lambda d: \lambda \in \mathbb{R}\}$,
and a ray is a set of the form $\{x + \lambda d: \lambda \ge 0\}$.
\end{definition}

\begin{definition}[Cone]
A set $S \in \mathbb{R}^n$ is called a cone if $\forall x \in S$ and $\forall \alpha \ge 0$,
we have $\alpha x \in S$. A cone is called \emph{pointed} if
$\forall x \in S - \{0\}$, we have $-x \not\in S$.
\end{definition}

\begin{definition}[Bounded set]
A set $S$ is bounded if $\exists \alpha \in \mathbb{R}$ such that
$\|x\| \le \alpha$ for all $x \in S$.
\end{definition}

\begin{definition}[Vertex of a set]
Let $S \subseteq \mathbb{R}^n$ and $x \in S$.
$x$ is a vertex of $S$ if $\exists c \in \mathbb{R}^n$
such that $\forall y \in S - \{x\}$, we have $c^Tx > c^Ty$.
\end{definition}

\begin{definition}[Direction of a set]
$d$ is a direction of $S \subseteq \mathbb{R}^n$ if
$\forall x \in S$, $\forall \lambda \ge 0$, we have $x + \lambda d \in S$.
\end{definition}

\begin{definition}[Extreme point]
$x \in S$ is an extreme point of $S \subseteq \mathbb{R}^n$ if
(the following definitions are equivalent):
\begin{tightenum}
\item $x$ is not a strict convex combination of at least 2 points in $S$.
\item $x$ is not a strict convex combination of exactly 2 points in $S$.
\item $\forall y \in \mathbb{R}^n$, $x + y \not\in S$ or $x - y \not\in S$.
\end{tightenum}
\end{definition}

\begin{definition}[Extreme direction]
Let $C$ be a pointed convex cone. Let $d \in C - \{0\}$.
Then $d$ is called an extreme direction of $C$ iff
$d$ cannot be expressed as $x + y$, where $x \in C$ and $y \in C$ are non-collinear.
\end{definition}

\begin{definition}[Projection]
Let $S \subseteq \mathbb{R}^n$. Let $R \defeq \{(x_1, \ldots, x_k): x \in S\}$ for any $k < n$.
Then $R$ is called the $k$-dimensional projection of $S$.
\end{definition}

\begin{definition}[Minkowski sum]
Let $S$ and $T$ be two sets. Then their Minkowski sum $S + T$ is $\{s + t: s \in S, t \in T\}$.
\end{definition}

\subsection{Polyhedra}

\begin{definition}[Polyhedron]
\label{defn:polyhedron}
A polyhedron is an intersection of half-spaces.
Formally, a polyhedron is represented using equalities and inequalities:
$P \defeq \{x \in \mathbb{R}^n: (a_i^Tx \ge b_i, \forall i \in I) \wedge (a_i^Tx = b_i, \forall i \in E)\}$.
\end{definition}

\begin{definition}[Polyhedral cone]
A polyhedral cone is (equiv defns):
\begin{tightenum}
\item a polyhedron that is also a cone.
\item a set of the form
$\{x \in \mathbb{R}^n: (a_i^Tx \ge 0, \forall i \in I) \wedge (a_i^Tx = 0, \forall i \in E)\}$.
\end{tightenum}
\end{definition}

\begin{definition}[BFS]
Let $Q$ be this system of inequations:
$(a_i^Tx \ge b_i, \forall i \in I) \wedge (a_i^Tx = b_i, \forall i \in E)$.
$\xhat \in \mathbb{R}^n$ and let $B \defeq \{a_i: a_i^T\xhat = b_i\}$.
Then $\xhat$ is called a \emph{basic feasible solution} (BFS) of $Q$
if $\xhat$ satisfies $Q$ and $\rank(B) = n$.
\end{definition}

\begin{definition}[BFD]
Let $Q$ be this system of inequations:
$(a_i^Tx \ge b_i, \forall i \in I) \wedge (a_i^Tx = b_i, \forall i \in E)$.
Let $Q'$ be this system of inequations:
$(a_i^Tx \ge 0, \forall i \in I) \wedge (a_i^Tx = 0, \forall i \in E)$.
$\xhat \in \mathbb{R}^n$ and let $B \defeq \{a_i: a_i^T\xhat = 0\}$.
Then $\xhat$ is called a \emph{basic feasible direction} (BFD) of $Q$ if
$\xhat$ satisfies $Q'$ and $\rank(B) = n-1$.
\end{definition}

\section{Results on Polyhedra}

\begin{theorem}[Fourier-Motzkin Elimination]
\label{thm:fme}
The projection of a polyhedron is also a polyhedron.
\end{theorem}

\begin{theorem}[Pointing a polyhedron]
Let $P \defeq \{x \in \mathbb{R}^n: (a_i^Tx = b_i, \forall i \in E) \wedge (a_i^Tx \ge b_i, \forall i \in I)\}$
be a non-empty polyhedron. Let $A \defeq \{a_i: i \in I \cup E\}$.
Let $B \defeq \{a_i: i \in T\}$ (where $T \subseteq I \cup E$) be a basis of $A$.

Let $D \defeq \{x \in \mathbb{R}^n: (a_i^Tx = 0, \forall i \in I \cup E)\}$.
Let $C \defeq \{a_i: i \in H\}$ (where $H$ is a new set of indices) be a basis of $D$.
Let $P' \defeq \{x \in P: (a_i^Tx = 0, \forall i \in H)\}$.

Then the following hold:
\begin{tightenum}
\item $|C| = n - \rank(A)$ and $B \cup C$ is a basis of $\mathbb{R}^n$.
\item $P = \{\xhat + d: \xhat \in P', d \in D\}$.
\end{tightenum}

Note that part 1 implies that $P'$ is a full-rank polyhedron.
Part 2 implies that any polyhedron $P$ can be decomposed into a full-rank polyhedron $P'$ and subspace $D$.
\end{theorem}

\section{BFS and BFD}

\begin{theorem}[Equiv of extreme point, vertex, BFS]
Let $P \defeq \{x \in \mathbb{R}^n: (a_i^Tx \ge b_i, \forall i \in I) \wedge (a_i^Tx = b_i, \forall i \in E)\}$
be a non-empty polyhedron. Then the following are equivalent for any $\xhat \in P$:
\begin{tightenum}
\item $\xhat$ is a vertex of $P$.
\item $\xhat$ is an extreme point of $P$.
\item $\xhat$ is a BFS of the system of constraints defining $P$.
\end{tightenum}
\end{theorem}

\begin{theorem}[Equiv of BFD and extreme direction]
Let $P \defeq \{x \in \mathbb{R}^n: (a_i^Tx \ge b_i, \forall i \in I) \wedge (a_i^Tx = b_i, \forall i \in E)\}$.
Let $C \defeq \{x \in \mathbb{R}^n: (a_i^Tx \ge 0, \forall i \in I) \wedge (a_i^Tx = 0, \forall i \in E)\}$.
$d$ is a BFD of $P$'s constraints iff it is an extreme direction of $C$.
\end{theorem}

\begin{theorem}[Condition for existence of BFS]
Let $P \defeq \{x \in \mathbb{R}^n: (a_i^Tx \ge b_i, \forall i \in I) \wedge (a_i^Tx = b_i, \forall i \in E)\}$
be a non-empty polyhedron. Let $A = \{a_i: i \in I \cup E\}$. Then the following are equivalent:
\begin{tightenum}
\item $P$ has a BFS.
\item $P$ doesn't contain a line.
\item $\rank(A) = n$.
\end{tightenum}
\end{theorem}

\begin{theorem}[Representation theorem]
Let $P$ be a polyhedron. Let $x^{(1)}, \ldots, x^{(p)}$ be its BFSes and
$d^{(1)}, \ldots, d^{(q)}$ be its BFDs, where $p \ge 1$. Then for any $\xhat \in P$,
$\exists \lambda \in \mathbb{R}^p_{\ge 0}$, $\exists \mu \in \mathbb{R}^q_{\ge 0}$ such that
$\sum_{i=1}^p \lambda_i = 1$ and
\[ \xhat = \sum_{i=1}^p \lambda_ix^{(i)} + \sum_{j=1}^q \mu_jd^{(j)}. \]
\end{theorem}
\begin{proof}[Proof sketch]
First handle the special case of polytopes by inducting on the number of tight constraints.
Reduce the special case of cones to the polytope case by
scaling each non-zero point of the cone $C$ to lie on a hyperplane $H$
such that $C \cap H$ is bounded.
For the general case, induct on the number of tight constraints, and use the result for cones.
\end{proof}

\begin{theorem}[Converse of representation theorem]
Let $x^{(1)}, \ldots, x^{(p))}, d^{(1)}, \ldots, d^{(q)}$ be points.
Let $P \defeq \{\sum_{i=1}^p \lambda_ix^{(i)} + \sum_{j=1}^q \mu_jd^{(j)}:
\lambda \in \mathbb{R}^p_{\ge 0}, \mu \in \mathbb{R}^q_{\ge 0}, \sum_{i=1}^p \lambda_i = 1\}$.
Then $P$ is a polyhedron.
\end{theorem}
\begin{proof}[Proof sketch]
Simple application of \nameref{thm:fme}.
\end{proof}

\section{Duality and Farkas' Lemma}

Content in this section is based on Prof. Karthik's notes: \href{\karthikLec{3}}{Lecture 3}.

\begin{theorem}[Fund thm of lin ineqs]
Let $a_1, \ldots, a_n, b \in \mathbb{R}^m$. Let $t \defeq \rank(\{b, a_1, \ldots, a_n\})$.
Then exactly one of these holds:
\begin{tightenum}
\item $b$ lies in the non-neg hull of $\{a_i: i \in [n]\}$.
\item $\exists c \in \mathbb{R}^m$ such that all of these hold:
    \begin{tightenum}
    \item $c^Tb < 0$.
    \item $c^Ta_i \ge 0$ for all $i$.
    \item $\rank(\{a_i: a_i^Tc = 0\}) = t-1$.
    \item If $\{b, a_1, \ldots, a_n\}$ are rational, then $c$ is rational.
    \end{tightenum}
\end{tightenum}
\end{theorem}

\section{Dimension and Faces}

Content in this section is based on Prof. Karthik's notes:
Lectures \href{\karthikLec{5}}{5} and \href{\karthikLec{6}}{6}.

\begin{definition}[Implicit equality and redundant constraint]
Let $P \defeq \{x \in \mathbb{R}^n: (a_i^Tx \ge b_i, \forall i \in I) \wedge (a_i^Tx = b_i, \forall i \in E)\}$.
For $i \in I$, $a_i^Tx \ge b_i$ is an implicit equality of $P$ if $a_i^Tx = b_i$ for all $x \in P$.
For $i \in I \cup E$, the $i\Th$ constraint is redundant if removing it changes the polyhedron.
\end{definition}

\begin{theorem}[$\dim$ from rank of constraints]
\label{thm:dim-from-rank}
Let $P \defeq \{x \in \mathbb{R}^n: (a_i^Tx \ge b_i, \forall i \in I) \wedge (a_i^Tx = b_i, \forall i \in E)\}$
be a non-empty polyhedron without any implicit equalities.
Let $A \defeq \{a_i: i \in E\}$. Then $\dim(P) = n - \rank(A)$.
\end{theorem}
\begin{proof}[Proof sketch]
Use a max-cardinality affindep subset of $P$ to construct a basis of $A$'s nullspace.
Use an interior point of $P$ and a basis of $A$'s nullspace to construct
a max-cardinality affindep subset of $P$.
\end{proof}

\begin{definition}[Face]
Let $P$ be a non-empty polyhedron. $F$ is a \emph{face} of $P$ if $F \neq \emptyset$
and $\exists c \in \mathbb{R}^n$, $\exists \beta \in \mathbb{R}$, such that
$F = \{x \in P: c^Tx = \beta\}$ and $(c^Tx \ge \beta, \forall x \in P)$.
(Note that $c$ can be 0, in which case $F = P$.)
A $F$ is called a \emph{proper face} of $P$ if $F$ is a face of $P$ and $F \neq P$.
\end{definition}

\begin{definition}[Tightening]
Let $P \defeq \{x \in \mathbb{R}^n: (a_i^Tx \ge b_i, \forall i \in I) \wedge (a_i^Tx = b_i, \forall i \in E)\}$
$F$ is a tightening of $P$ if $\exists T$ such that $E \subseteq T \subseteq I \cup E$ and
$F = \{x \in \mathbb{R}^n: (a_i^Tx = b_i, \forall i \in T) \wedge (a_i^Tx \ge b_i, \forall i \in I-T)\}$.
$T$ is called the set of tight constraints for $F$.
\end{definition}

\begin{lemma}[Tightening is face]
\label{thm:tightening-is-face}
If $F$ is a tightening of $P$, then $F$ is a face of $P$.
\end{lemma}
\begin{proof}[Proof sketch]
Pick $c \defeq \sum_{i \in T} a_i$ and $\beta \defeq \sum_{i \in T} b_i$.
Then $\forall x \in F$, $c^Tx = \beta$, and $\forall x \in P - F$, $c^Tx > \beta$.
\end{proof}

\begin{theorem}[Face is tightening]
\label{thm:face-is-tightening}
If $F$ is a face of $P$, then $F$ is a tightening of $P$.
\end{theorem}
\begin{proof}[Proof sketch]
Let $F \defeq \{x \in P: c^Tx = \beta\}$ be a face of $P$,
where $c^Tx \ge \beta$ for all $x \in P$.
Consider the LP $\min_{x \in P} c^Tx$.
Then $\opt(\LP) = \beta$ and $F = \argopt(\LP)$, since $F \neq \emptyset$.
Let $w^*$ be the optimal dual solution.
Let $H \defeq \{i \in I: w_i^* > 0\}$. Let $T \defeq E \cup H$.
Let $F' \defeq \{x: (a_i^Tx = b_i, \forall i \in T) \wedge (a_i^Tx \ge b_i, \forall i \in I-T)\}$.

Let $\xhat \in F'$. We can show that $(\xhat, w^*)$ is a pair of solutions
satisfying complementary slackness. Hence, $\xhat$ is optimal, so $\xhat \in \argopt(\LP) = F$.

For all $\xhat \in F$, $\xhat$ is optimal for LP,
and so $(\xhat, w^*)$ satisfies complementary slackness (by strong duality of LPs).
Hence, for $i \in H$, $a_i^T\xhat = b_i$. Hence, $\xhat \in F'$.
Hence, $F = F'$. Since $F'$ is a tightening, so is $F$.
\end{proof}

\begin{definition}[Facet]
A facet of $P$ is a maximal proper face of $P$, i.e., $F$ is a facet of $P$ if
for every proper face $F'$ of $P$, we have $F' \supseteq F \implies F' = F$.
\end{definition}

\begin{theorem}[Facet is 1-tightening]
\label{thm:facet-is-1tight}
Let $P \defeq \{x \in \mathbb{R}^n: (a_i^Tx \ge b_i, \forall i \in I) \wedge (a_i^Tx = b_i, \forall i \in E)\}$
be a non-empty polyhedron without implicit equalities.
If $F$ is a facet of $P$, then $\exists k \in I$ such that $F = \{x \in P: a_k^Tx = b_k\}$.
\end{theorem}
\begin{proof}
$T$ be the tight constraint indices for $F$. Pick any $k \in T$.
Let $F' \defeq \{x \in P: a_k^Tx = b_k\}$. Then $F \subseteq F' \subseteq P$.
Since $k$ is not an implicit equality, $F' \neq P$.
Since $F$ is a maximal face, $F = F'$.
\end{proof}

\begin{theorem}[1-tightening is facet]
\label{thm:1tight-is-facet}
Let $P \defeq \{x \in \mathbb{R}^n: (a_i^Tx \ge b_i, \forall i \in I) \wedge (a_i^Tx = b_i, \forall i \in E)\}$
be a non-empty polyhedron without implicit equalities.
Let $R \subseteq I$ be the set of irredundant constraints.
If $F = \{x \in P: a_k^Tx = b_k\}$ for some $k \in R$, then $F$ is a facet.
\end{theorem}
\begin{proof}[Proof sketch]
By \cref{thm:tightening-is-face}, $F$ is a face of $P$.
Since $k$ is not an implicit equality, $F \neq P$.
Use irredundancy of $k$ to find an interior point $\xhat$ of $F$
(pick an interior point $x^{(1)}$ of $P$,
pick a point $x^{(2)}$ that violates only the $k\Th$ constraint,
and interpolate between them).

Suppose $F$ is not a facet. Then it is contained by another proper face $F' \neq F$.
Let $T'$ be the tight constraints of $F'$. Since $\xhat \in F'$, $T' = E \cup \{k\}$, so $F = F'$.
\end{proof}

\begin{theorem}[Facet has dimension $\dim(P)-1$]
Let $F$ be a face of $P$. Then $F$ is a facet of $P$ iff $\dim(F) = \dim(P)-1$.
\end{theorem}
\begin{proof}[Proof sketch]
Let $E$ be the equality (both implicit and explicit) constraints.
Let $T$ be a set of tight constraints of $F$.
If $F$ is a facet, $|T-E| = 1$ by \cref{thm:facet-is-1tight}. Then use \cref{thm:dim-from-rank}.
Now let $\dim(F) = \dim(P)-1$. Then $\rank(T) = \rank(E)+1$.
Let $k \in T-E$ be linindep from $E$. Then $\Span(E\cup\{k\}) = \Span(T)$,
so $E \cup \{k\}$ is also a set of tight constraints for $F$.
By \cref{thm:1tight-is-facet}, $F$ is a facet.
\end{proof}

\begin{theorem}[Minimal face from subsystem]
Let $P \defeq \{x \in \mathbb{R}^n: (a_i^Tx \ge b_i, \forall i \in I) \wedge (a_i^Tx = b_i, \forall i \in E)\}$
be a non-empty polyhedron. Let $F \subseteq P$ and $F \neq \emptyset$.
Then $F$ is a minimal face of $P$ iff $\exists K$ such that $E \subseteq K \subseteq I \cup E$
and $F = \{x: (a_i^Tx = b_i, \forall i \in K)\}$.
\end{theorem}
\begin{proof}
(TODO)
\end{proof}

\begin{theorem}[Minimal face from rank]
Let $P \defeq \{x \in \mathbb{R}^n: (a_i^Tx \ge b_i, \forall i \in I) \wedge (a_i^Tx = b_i, \forall i \in E)\}$
be a non-empty polyhedron. Let $F \subseteq P$ and $F \neq \emptyset$.
Let $A \defeq \{a_i: i \in I \cup E\}$.
Then $F$ is a minimal face of $P$ iff $\dim(F) = n - \rank(A)$.
\end{theorem}
\begin{proof}
(TODO)
\end{proof}

\end{document}

\documentclass[a4paper, 12pt, fleqn]{article}

\usepackage{amsmath}
\usepackage{amsthm}
\usepackage{amssymb}
\usepackage[margin=1in]{geometry}
\usepackage{xcolor}
\usepackage{float}
\usepackage{url}
\usepackage{array}
\usepackage{comment}
\usepackage[bookmarksnumbered=true,hypertexnames=false]{hyperref}
\usepackage{algorithm, algpseudocode}
\usepackage[capitalize,sort]{cleveref}

\hypersetup{
    colorlinks,
    linkcolor={red!50!black},
    citecolor={red!50!black},
    urlcolor={blue!50!black}
}

% THEOREMS

\newtheorem{theorem}{Theorem}
\newtheorem{definition}{Definition}
\newtheorem{example}{Example}
\newtheorem{corollary}{Corollary}[theorem]
\newtheorem{lemma}[theorem]{Lemma}

% SHORTHANDS

\newcommand*{\Th}{^{\textrm{th}}}
\newcommand*{\WLoG}{Without loss of generality}
\newcommand*{\wLoG}{without loss of generality}
\newcommand*{\wrt}{with respect to}

% SYMBOLS

\let\eps\epsilon
\newcommand*{\defeq}{:=}

\newcommand*{\Ahat}{\widehat{A}}
\newcommand*{\Bhat}{\widehat{B}}
\newcommand*{\Chat}{\widehat{C}}
\newcommand*{\Jhat}{\widehat{J}}
\newcommand*{\Shat}{\widehat{S}}
\newcommand*{\Yhat}{\widehat{Y}}
\newcommand*{\bhat}{\widehat{b}}
\newcommand*{\what}{\widehat{w}}
\newcommand*{\xhat}{\widehat{x}}
\newcommand*{\yhat}{\widehat{y}}
\newcommand*{\zhat}{\widehat{z}}

\newcommand*{\Btild}{\widetilde{B}}
\newcommand*{\Ctild}{\widetilde{C}}
\newcommand*{\Jtild}{\widetilde{J}}
\newcommand*{\Stild}{\widetilde{S}}
\newcommand*{\Ytild}{\widetilde{Y}}
\newcommand*{\btild}{\widetilde{b}}
\newcommand*{\wtild}{\widetilde{w}}
\newcommand*{\xtild}{\widetilde{x}}
\newcommand*{\ytild}{\widetilde{y}}
\newcommand*{\ztild}{\widetilde{z}}

\newcommand*{\Fcal}{\mathcal{F}}
\newcommand*{\Tcal}{\mathcal{T}}

% MATH

\newcommand*{\floor}[1]{\left\lfloor #1 \right\rfloor}
\newcommand*{\smallfloor}[1]{\lfloor #1 \rfloor}
\newcommand*{\ceil}[1]{\left\lceil #1 \right\rceil}
\newcommand*{\smallceil}[1]{\lceil #1 \rceil}
\newcommand*{\abs}[1]{\left\lvert #1 \right\rvert}
\newcommand*{\smallabs}[1]{\lvert #1 \rvert}
\newcommand*{\norm}[1]{\left\lVert #1 \right\rVert}
\newcommand*{\smallnorm}[1]{\lVert #1 \rVert}
\newcommand*{\Z}{\mathbb{Z}}
\DeclareMathOperator*{\E}{E}
\DeclareMathOperator*{\Var}{Var}
\DeclareMathOperator*{\argmin}{argmin}
\DeclareMathOperator*{\argmax}{argmax}
\DeclareMathOperator{\poly}{poly}
\DeclareMathOperator{\opt}{opt}
\DeclareMathOperator{\support}{support}
\newcommand*{\OPT}{\mathrm{OPT}}


% INITIALIZATIONS

\newcommand{\initMinimal}{
\setlength{\parindent}{0pt}
\setlength{\parskip}{0.5em}
}
\newcommand{\initFromContents}{
\tableofcontents
\newpage
\initMinimal{}
}
\newcommand{\initAfterBeginDocument}{
\maketitle
\initFromContents{}
}
\newcommand{\addMyBib}{
\bibliographystyle{plainurl}
\bibliography{bibdb}
}

\author{Eklavya Sharma}
\date{\empty}


\title{Inverse of \texorpdfstring{$x \mapsto x\ln x$}{x -> x*ln(x)}}

\begin{document}

\maketitle
\initMinimal{}

We want to bound the inverse of $x \mapsto x\ln x$, i.e.,
if $y = x\ln x$, then we want to lower and upper bound $x$
by simple functions of $y$.

Our main result is that $x \in \Theta(y/\ln y)$.
We also prove tighter non-asymptotic bounds.

\section{Preliminaries}

\begin{lemma}[Log bounds \cite{theoremdep:log-bound}]
\label{thm:log-bound}
\[ \forall x \in \mathbb{R}_{> 0}, \frac{x-1}{x} \le \ln x \le x - 1. \]
\end{lemma}

\begin{lemma}
\label{thm:xy-lb}
Let $y = x\ln x$.
Then $y > 0 \iff x > 1$ and $y \ge e \iff x \ge e$.
\end{lemma}
\begin{proof}
\[ x \le 1 \implies \ln x \le 0 \implies y = x\ln x \le 0. \]
\[ x > 1 \implies \ln x > 0 \implies y = x\ln x > 0. \]
Therefore, $x \ge 1 \iff y \ge 0$.
\[ x < e \implies \ln x < 1 \implies y = x\ln x < e. \]
\[ x \ge e \implies \ln x \ge 1 \implies y = x\ln x \ge e. \]
Therefore, $x \ge e \iff y \ge e$.
\end{proof}

\begin{theorem}
\label{thm:refine}
Let $x \ge 1$ and $y = x\ln x$. Then
\[ 1 < \ell \le x \le u \implies \frac{y}{\ln u} \le x \le \frac{y}{\ln\ell}. \]
\end{theorem}
\begin{proof}
\begin{align*}
& \ell \le x \le u
\implies \ln\ell \le \ln x \le \ln u
\\ &\implies x\ln\ell \le y \le x\ln u
\implies \frac{y}{\ln u} \le x \le \frac{y}{\ln\ell}
\qedhere \end{align*}
\end{proof}
The above theorem is useful because it helps us \emph{refine} the bounds that we find.

\section{Bounds when \texorpdfstring{$y \ge e$}{y >= e}}

\begin{theorem}
\label{thm:lin-ub-e}
Let $y \ge e$ and $y = x\ln x$. Then $x \le y$.
\end{theorem}
\begin{proof}
By \cref{thm:xy-lb}, $y \ge e \iff x \ge e$.
$x \ge e \implies \ln x \ge 1 \implies y \ge x$.
\end{proof}

\begin{theorem}
Let $x \ge e$ and $y = x\ln x$. Then $x \ge y/(\ln y)$.
\end{theorem}
\begin{proof}
Set $u = y$ and use \cref{thm:lin-ub-e,thm:refine}.
\end{proof}

\begin{theorem}
Let $y > 1$ and $y = x\ln x$. Then
\[ x \le \frac{e+1}{e} \frac{y}{\ln y}. \]
\end{theorem}
\begin{proof}
By \cref{thm:xy-lb}, $x > 1$.
\[ \frac{x\ln y}{y}
= \frac{x(\ln x + \ln\ln x)}{x\ln x}
= 1 + \frac{\ln\ln x}{\ln x} \]
Let $t = \ln x$. Then $t > 0$ and $(x\ln y)/y = 1 + (\ln t)/t$.
Define $g(t) = (\ln t)/t$. Then
\[ g'(t) = \frac{1 - \ln t}{t^2} \]
$g'(t)$ is positive for $t < e$, negative for $t > e$ and $g'(e) = 0$.
Therefore, $g(t)$ is maximized at $t = e$, and the maximum value is $g(e) = 1/e$.
Therefore,
\[ \frac{x\ln y}{y} = 1 + g(\ln x) \le 1 + \frac{1}{e}
\implies x \le \frac{e+1}{e} \frac{y}{\ln y}  \qedhere \]
\end{proof}

\section{Bounds when \texorpdfstring{$y \ge 0$}{y >= 0}}

\begin{theorem}
\label{thm:lin-ub}
Let $y = x\ln x$. Then $x \le y + 1$.
\end{theorem}
\begin{proof}
$\frac{x-1}{x} \le \ln x \implies x-1 \le y$.
\end{proof}

Define
\[ \ell(y) = \begin{cases} y/\ln(y+1) & y \neq 0
\\ 1 & y = 0 \end{cases}. \]
Note that $\ell(y)$ is continuous over $y \in (-1, \infty)$.

\begin{theorem}
\label{thm:ell-lb}
Let $x \ge 1$ and $y = x\ln x$. Then $\ell(y) \le x$.
\end{theorem}
\begin{proof}
This is true for $x = 1$.
For $x > 1$, set $u = y+1$ and use \cref{thm:lin-ub,thm:refine}.
\end{proof}

\begin{theorem}
\label{thm:ell-ub}
Let $x \ge 1$ and $y = x\ln x$. Then $x \le 2\ell(y) - 1$.
\end{theorem}
\begin{proof}
This is true for $x = 1$, so let $x > 1$.
By \cref{thm:xy-lb}, $y > 0$.
\[ x \le 2\ell(y) - 1 \iff \frac{2y}{\ln(y+1)} \ge x+1
\iff \frac{2x\ln x}{x+1} \ge \ln(x\ln x + 1) \]
Define $g(x)$ as
\[ g(x) = \frac{2x\ln x}{x+1} - \ln(x\ln x + 1). \]
To prove that $x \le 2\ell(y) - 1$, we need to prove that $g(x) \ge 0$.

Note that $g(1) = 0$. If we prove that $g'(x) \ge 0$ for all $x \ge 1$,
then that would imply $g(x) \ge 0$ for all $x \ge 1$.
\begin{align*}
g'(x) &= 2\left( \frac{\ln x + 1}{x+1} - \frac{x\ln x}{(x+1)^2}\right)
    - \frac{\ln x + 1}{x\ln x + 1}
\\ &= \frac{x^2(\ln x - 1) + 2x(\ln x)^2 + \ln x + 1}{(x+1)^2(x\ln x + 1)}
\end{align*}
\begin{align*}
& x^2(\ln x - 1) + 2x(\ln x)^2 + \ln x + 1
\\ &\ge x^2\left(\frac{x-1}{x} - 1\right) + 2x\left(\frac{x-1}{x}\right)^2 + \frac{x-1}{x} + 1
    \tag{since $x > 1$ and by \cref{thm:log-bound}}
\\ &= x + \frac{1}{x} - 2 = \left(\sqrt{x} - \frac{1}{\sqrt{x}}\right)^2 \ge 0
\end{align*}
Hence, for $x \ge 1$, $g'(x) \ge 0 \implies g(x) \ge 0 \implies x \le 2\ell(y) - 1$.
\end{proof}

\begin{theorem}
Let $y = x\ln x$. Then $\ell(y) - 1 \le x$.
\end{theorem}
\begin{proof}(TODO)\end{proof}

\addMyBib{}

\end{document}

\documentclass[a4paper, 12pt, fleqn]{article}

\usepackage{amsmath,amsthm,amssymb}
\usepackage[margin=1in]{geometry}
\usepackage{xcolor}
\usepackage{array}
\usepackage{comment}
\usepackage{booktabs}
\usepackage[bookmarksnumbered=true,hypertexnames=false]{hyperref}
%\usepackage{algorithm, algpseudocode}
\usepackage[capitalize,sort]{cleveref}

\def\colorschemesepia{sepia}
\def\colorschemedark{dark}
\def\colorschemelight{light}

\ifx\colorscheme\undefined
\let\colorscheme\colorschemelight
\fi

\ifx\colorscheme\colorschemelight
\colorlet{textColor}{black}
\colorlet{bgColor}{white}
\fi

\ifx\colorscheme\colorschemesepia
\definecolor{textColor}{HTML}{433423}
\definecolor{bgColor}{HTML}{fbf0da}
\fi

\ifx\colorscheme\colorschemedark
\definecolor{textColor}{HTML}{bdc1c6}
\definecolor{bgColor}{HTML}{202124}
\definecolor{textBlue}{HTML}{8ab4f8}
\definecolor{textRed}{HTML}{f9968b}
\colorlet{bgRed}{red!40!black}
\colorlet{bgBlue}{red!40!black}
\else
\colorlet{textBlue}{blue!50!black}
\colorlet{textRed}{red!50!black}
\fi

\ifx\colorscheme\colorschemelight\else
\pagecolor{bgColor}
\color{textColor}
\fi

\hypersetup{colorlinks,linkcolor=textRed,citecolor=textRed,urlcolor=textBlue}
% THEOREMS

\newtheorem{theorem}{Theorem}
\newtheorem{definition}{Definition}
\newtheorem{example}{Example}
\newtheorem{corollary}{Corollary}[theorem]
\newtheorem{lemma}[theorem]{Lemma}

% SHORTHANDS

\newcommand*{\Th}{^{\textrm{th}}}
\newcommand*{\WLoG}{Without loss of generality}
\newcommand*{\wLoG}{without loss of generality}
\newcommand*{\wrt}{with respect to}

% SYMBOLS

\let\eps\epsilon
\newcommand*{\defeq}{:=}

\newcommand*{\Ahat}{\widehat{A}}
\newcommand*{\Bhat}{\widehat{B}}
\newcommand*{\Chat}{\widehat{C}}
\newcommand*{\Jhat}{\widehat{J}}
\newcommand*{\Shat}{\widehat{S}}
\newcommand*{\Yhat}{\widehat{Y}}
\newcommand*{\bhat}{\widehat{b}}
\newcommand*{\what}{\widehat{w}}
\newcommand*{\xhat}{\widehat{x}}
\newcommand*{\yhat}{\widehat{y}}
\newcommand*{\zhat}{\widehat{z}}

\newcommand*{\Btild}{\widetilde{B}}
\newcommand*{\Ctild}{\widetilde{C}}
\newcommand*{\Jtild}{\widetilde{J}}
\newcommand*{\Stild}{\widetilde{S}}
\newcommand*{\Ytild}{\widetilde{Y}}
\newcommand*{\btild}{\widetilde{b}}
\newcommand*{\wtild}{\widetilde{w}}
\newcommand*{\xtild}{\widetilde{x}}
\newcommand*{\ytild}{\widetilde{y}}
\newcommand*{\ztild}{\widetilde{z}}

\newcommand*{\Fcal}{\mathcal{F}}
\newcommand*{\Tcal}{\mathcal{T}}

% MATH

\newcommand*{\floor}[1]{\left\lfloor #1 \right\rfloor}
\newcommand*{\smallfloor}[1]{\lfloor #1 \rfloor}
\newcommand*{\ceil}[1]{\left\lceil #1 \right\rceil}
\newcommand*{\smallceil}[1]{\lceil #1 \rceil}
\newcommand*{\abs}[1]{\left\lvert #1 \right\rvert}
\newcommand*{\smallabs}[1]{\lvert #1 \rvert}
\newcommand*{\norm}[1]{\left\lVert #1 \right\rVert}
\newcommand*{\smallnorm}[1]{\lVert #1 \rVert}
\newcommand*{\Z}{\mathbb{Z}}
\DeclareMathOperator*{\E}{E}
\DeclareMathOperator*{\Var}{Var}
\DeclareMathOperator*{\argmin}{argmin}
\DeclareMathOperator*{\argmax}{argmax}
\DeclareMathOperator{\poly}{poly}
\DeclareMathOperator{\opt}{opt}
\DeclareMathOperator{\support}{support}
\newcommand*{\OPT}{\mathrm{OPT}}


\DeclareMathOperator{\boolone}{\mathbf{1}}
\DeclareMathOperator{\MGF}{MGF}
\DeclareMathOperator{\Cov}{Cov}
\DeclareMathOperator{\infOft}{io}
\DeclareMathOperator{\almEv}{ae}

\author{Eklavya Sharma}
\date{\empty}

\title{Basics of Probability}

\begin{document}

\maketitle
\setlength{\parskip}{0.2em}

\begin{definition}[Probability Space]
A probability space is a triple $(\Omega, \Fcal, P)$ where
\begin{itemize}
\item $\Omega$ is the sample space, also called the set of all outcomes.
\item $\Fcal$ is a $\sigma$-algebra over $\Omega$. $\Fcal$ is called the set of all events.
\item $P: \Fcal \mapsto [0, 1]$ is a measure over $(\Omega, \Fcal)$
    (i.e., $P$ is $\sigma$-additive) such that $P(\Omega) = 1$.
    $P$ is called the probability measure.
\end{itemize}
\end{definition}

\begin{theorem}[Inclusion-Exclusion Principle]
\[ \Pr\left(\bigcup_{i=1}^n A_i\right)
= \sum_{k=1}^n (-1)^{k+1} \sum_{1 \le i_1 < \cdots < i_k \le n} \Pr(A_{i_1} \cap \cdots \cap A_{i_k}). \]
\end{theorem}

\begin{theorem}
For randvars $X$ and $Y$, $\E(X + Y) = \E(X) + \E(Y)$.
\end{theorem}

\begin{theorem}
For independent randvars $X_1, \ldots, X_n$, $\E(X_1 \ldots X_n) = \E(X_1) \ldots \E(X_n)$.
\end{theorem}

\begin{theorem}
For a non-negative randvar $X$,
\[ \E(X) = \begin{cases} \displaystyle \sum_{i=0}^{\infty} \Pr(X > i) & \textrm{ if $X$ is discrete}
\\[1.2em] \displaystyle \int_0^{\infty} \Pr(X > x) dx & \textrm{ if $X$ is continuous} \end{cases}. \]
\end{theorem}

\begin{definition}
\begin{align*}
\Cov(X, Y) &\defeq \E((X-\E(X))(Y-\E(Y))) = \E(XY) - \E(X)\E(Y)
\\ \Var(X) &\defeq \Cov(X, X) = \E((X-\E(X))^2) = \E(X^2) - \E(X)^2
\end{align*}
\end{definition}

\begin{theorem}
\[ \Var\left(\sum_{i=1}^n X_i\right) = \sum_{i=1}^n \Var(X_i) + 2\sum_{1 \le i < j \le n} \Cov(X_i, X_j). \]
\end{theorem}

\begin{theorem}
Let $\MGF_t(X) \defeq \E(e^{tX})$. Then $\MGF_t$ uniquely determines $X$'s CDF.
\end{theorem}

\begin{theorem}[Change of variables]
Let $X \in \mathbb{R}^n$ be a continuous random vector.
Let $g: \mathbb{R}^n \mapsto \mathbb{R}^n$ be a bijective function
having continuous partial derivatives.
Then $f_{g(X)}(y) = f_X(x)|J_g(x)|^{-1}$, where $x \defeq g^{-1}(y)$
and $J_g$ is the Jacobian of $g$
(i.e., $J_g(x)[i,j] \defeq \partial g(x)_i/\partial x_j$).
\end{theorem}

\begin{definition}
Let $A = [A_1, A_2, \ldots]$ be an infinite sequence of events. Then
\begin{align*}
\infOft(A) &= \lim_{m \to \infty} \bigcup_{i=m}^{\infty} A_i
    = \bigcap_{m=1}^{\infty} \bigcup_{i=m}^{\infty} A_i
& \almEv(A) &= \lim_{m \to \infty} \bigcap_{i=m}^{\infty} A_i
    = \bigcup_{m=1}^{\infty} \bigcap_{i=m}^{\infty} A_i.
\end{align*}
$\infOft(A)$ are the outcomes in $\Omega$ for which infinitely many events in $A$ happen.
$\almEv(A)$ are the outcomes in $\Omega$ for which all except finitely many events in $A$ happen.
\end{definition}

\begin{lemma}[Borel-Cantelli]
$\sum_{i=1}^{\infty} \Pr(A_i) \neq \infty \implies \Pr(\infOft(A)) = 0$.
\end{lemma}

\begin{lemma}
(Events in $A$ are independent and
$\sum_{i=1}^{\infty} \Pr(A_i) = \infty$) $\implies \Pr(\infOft(A)) = 1$.
\end{lemma}

\section{Probability Distributions}

\begin{table}[!ht]
\centering
\caption{Discrete Probability Distributions}
\defaultaddspace=0.8em
\abovetopsep=4pt
\begin{tabular}{lccccc}
\toprule Distribution
    & $\Pr(X = x)$
    % & $\Pr(X \le x)$
    & $\E(X)$
    & $\Var(X)$
    & $\MGF_t(X)$
\\ \midrule $\opname{Bernouilli}(p)$
    & $p^x(1-p)^{1-x}$
    % & --
    & $p$
    & $p(1-p)$
    & $pe^t + 1-p$
\\[\defaultaddspace] $\opname{Binomial}(n, p)$
    & $\displaystyle \binom{n}{x}p^x(1-p)^{n-x}$
    % & --
    & $np$
    & $np(1-p)$
    & $(pe^t + 1-p)^n$
\\[\defaultaddspace] $\opname{Geometric}(p)$
    & $(1-p)^{x-1}p$
    % & $1 - (1-p)^x$
    & $1/p$
    & $\displaystyle \frac{1-p}{p^2}$
    & $\displaystyle \frac{pe^t}{1-(1-p)e^t}$
\\[\defaultaddspace] $\opname{NegBinom}(n, p)$
    & $\displaystyle \binom{x-1}{n-1}p^n(1-p)^{x-n}$
    % & --
    & $n/p$
    & $\displaystyle \frac{n(1-p)}{p^2}$
    & $\displaystyle \left(\frac{pe^t}{1-(1-p)e^t}\right)^n$
\\[\defaultaddspace] $\opname{Poisson}(\lambda)$
    & $e^{-\lambda}\lambda^x/x!$
    % & --
    & $\lambda$
    & $\lambda$
    & $\displaystyle \exp(\lambda(e^t-1))$
% \\[\defaultaddspace] $\opname{NegBinom}(n, p)$
\\ \bottomrule
\end{tabular}
\label{table:disc-distr}
\end{table}

\begin{table}[!ht]
\centering
\caption{Continuous Probability Distributions}
\defaultaddspace=0.8em
\abovetopsep=4pt
\begin{tabular}{lccccc}
\toprule Distribution
    & $f_X(x)$
    % & $\Pr(X \le x)$
    & $\E(X)$
    & $\Var(X)$
    & $\MGF_t(X)$
\\ \midrule $\opname{Uniform}(a, b)$
    & $\displaystyle \frac{\mathbf{1}(a \le x \le b)}{b-a}$
    % & $1 - e^{\lambda x}$
    & $\displaystyle \frac{a+b}{2}$
    & $\displaystyle \frac{(b-a)^2}{12}$
    & $\displaystyle \frac{e^{bt} - e^{at}}{(b-a)t}$
\\[\defaultaddspace] $\opname{Exponential}(\lambda)$
    & $\lambda e^{-\lambda x}$
    % & $1 - e^{\lambda x}$
    & $1/\lambda$
    & $1/\lambda^2$
    & $\lambda/(\lambda - t)$
\\[\defaultaddspace] $\opname{Gamma}(n, \lambda)$
    & $\displaystyle \frac{(\lambda x)^{n-1}}{(n-1)!} \lambda e^{-\lambda x}$
    % & --
    & $n/\lambda$
    & $n/\lambda^2$
    & $\displaystyle \left(1-\frac{t}{\lambda}\right)^{-n}$
\\[\defaultaddspace] $\opname{Normal}(\mu, \sigma^2)$
    & $\displaystyle \frac{1}{\sigma\sqrt{2\pi}}e^{-\frac{1}{2}\left(\frac{x-\mu}{\sigma}\right)^2}$
    % & --
    & $\mu$
    & $\sigma^2$
    & $\exp(\mu t + \sigma^2t^2/2)$
\\ \bottomrule
\end{tabular}
\label{table:cont-distr}
\end{table}

\begin{theorem}[Poisson approximates Binomial]
Let $\lambda \in \mathbb{R}_{\ge 0}$ and $k \in \mathbb{Z}_{\ge 0}$ be constants.
Let $X_n \sim \opname{Binom}(n, \lambda/n)$.
Then $\lim_{n \to \infty} \Pr(X_n = k) = e^{-\lambda}\lambda^k/k!$.
\end{theorem}

\begin{theorem}[Binomial over Poisson]
Let $N \sim \opname{Poisson}(\lambda)$ and $M \mid N \sim \opname{Binom}(N, p)$.
Then $M \sim \opname{Poisson}(\lambda p)$.
\end{theorem}

\begin{theorem}[Poisson decomposition]
Let $X_1, X_2, \ldots$ be IID randvars, where $\Pr(X_j = i) = p_i$ for $i \in [k]$ and all $j$,
and $\sum_{i=1}^k p_i = 1$.
Let $N \sim \opname{Poisson}(\lambda)$ where $\{N, X_1, X_2, \ldots\}$ is independent.
Let $N_i \defeq \sum_{j=1}^N \boolone(X_j = i)$. Then
$\{N_1, \ldots, N_k\}$ is independent and $N_i \mid N \sim \opname{Binom}(N, p_i)$.
\end{theorem}

\begin{theorem}[Scaling normal]
$X \sim N(\mu, \sigma^2) \implies aX + b \sim N(a\mu + b, a^2\sigma^2)$.
\end{theorem}

\begin{theorem}[Competing exponentials]
Let $X_1, \ldots, X_n$ be independent ranvars, where $X_i \sim \opname{Expo}(\lambda_i)$.
Let $Z \defeq \min_{i=1}^n X_i$ and $E$ be the event $X_1 < X_2 < \ldots < X_n$.
Let $\beta \defeq \lambda_1 + \ldots + \lambda_n$. Then
\begin{itemize}
\item $\Pr(X_i = Z) = \lambda_i/\beta$.
\item $Z \sim \opname{Expo}(\beta)$.
\item $E$ and $Z$ are independent.
\end{itemize}
\end{theorem}

\subsection{Sum of Random Variables}

\begin{theorem}[Convolution]
\[ f_{X+Y}(z) = \begin{cases}
\displaystyle \sum_{y \in D} f_{X, Y}(z - y, y) = \sum_{x \in D} f_{X, Y}(x, z-x)
    & \textrm{ discrete}
\\[1.2em] \displaystyle
\int_{-\infty}^{\infty} f_{X, Y}(z - y, y) dy = \int_{-\infty}^{\infty} f_{X, Y}(x, z-x) dx
    & \textrm{ continuous}
\end{cases}. \]
\end{theorem}

\begin{theorem}
Let $X_1, \ldots, X_n$ be independent. Then
$\MGF_t\left(\sum_{i=1}^n X_i\right) = \prod_{i=1}^n \MGF_t(X_i)$.
\end{theorem}

\begin{theorem}
Let $X_1, \ldots, X_n$ be independent. Let $Y \defeq \sum_{i=1}^n X_i$. Then
\begin{itemize}
\item $X_i \sim \opname{Bernouilli}(p)
    \implies Y \sim \opname{Binomial}(n, p)$.
\item $X_i \sim \opname{Poisson}(\lambda_i) \implies
    Y \sim \opname{Poisson}\left(\sum_{i=1}^n \lambda_i\right)$.
\item $X_i \sim \opname{Exponential}(\lambda) \implies
    Y \sim \opname{Gamma}(n, \lambda)$.
\item $X_i \sim \opname{Geometric}(p) \implies
    Y \sim \opname{NegBinom}(n, p)$.
\end{itemize}
\end{theorem}

\section{Inequalities and Limits}

\begin{theorem}[Markov]
For non-negative randvar $X$, $\Pr(X \ge a) \le \E(X)/a$.
\end{theorem}

\begin{theorem}[Chebyshev]
$\displaystyle \Pr(|X-\E(X)| \ge a) \le \frac{\Var(X)}{a^2}$.
\end{theorem}

\begin{theorem}[One-sided Chebyshev]
\begin{align*}
\Pr(X-\E(X) \ge a) &\le \frac{\Var(X)}{\Var(X)+a^2}
& \Pr(X-\E(X) \le -a) &\le \frac{\Var(X)}{\Var(X)+a^2}
\end{align*}
\end{theorem}

\begin{theorem}[Strong law of large lumbers]
Let $X_1, X_2, \ldots$ be IID randvars having mean $\mu$.
Let $Y_n \defeq \frac{1}{n} \sum_{i=1}^n X_i$. Let
\[ E \defeq \left\{\omega \in \Omega: \lim_{n \to \infty} Y_n(\omega) = \mu\right\}. \]
Then $\Pr(E) = 1$.
\end{theorem}

\begin{definition}
Let $Z$ be a random variable and $S \defeq [X_1, X_2, \ldots]$ be an infinite sequence
of random variables. We say that $S$ converges to $Z$ if
$\lim_{n \to \infty} F_{X_n}(x) = F_Z(x)$ for all $x \in \mathbb{R}$ where $F_Z$ is continuous.
\end{definition}

\begin{theorem}[Central Limit Theorem]
Let $X_1, X_2, \ldots$ be IID randvars having mean $\mu$ and variance $\sigma^2$.
Let $Y_n \defeq \sqrt{n}\left(\frac{1}{n}\sum_{i=1}^n X_i - \mu\right)$.
Then $[Y_1, Y_2, \ldots]$ converges to $N(0, \sigma^2)$.
\end{theorem}

\begin{theorem}[Jensen's inequality]
If $X$ is a random variable and $f$ is a convex function, then $f(\E(X)) \le \E(f(X))$.
\end{theorem}

\begin{theorem}[Cauchy-Schwarz inequality]
For random variables $X$ and $Y$, $|\E(XY)|^2 \le \E(X^2)\E(Y^2)$
and $|\Cov(X,Y)|^2 \le \Var(X)\Var(Y)$.
\end{theorem}

\section{Conditional Probability}

\begin{theorem}
Let $X$ and $Y$ be randvars (either of them can be discrete or continuous).
Let $f_X$ and $f_Y$ be their distribution functions (either PMF or PDF), respectively.
Let $f_{X,Y}$ be their joint distribution function.
Let $g_x$ be the distribution function of $Y$ conditioned on $X=x$.
Then $g_x(y) = f_{X,Y}(x, y) / f_X(x)$.
We denote $g_x(y)$ by $f_{Y \mid X}(y \mid x)$.
\end{theorem}

\begin{definition}
Let $X$ and $Y$ be randvars and $A$ be an event.
Let $g(x) \defeq \Pr(A \mid X=x)$ and $h(x) \defeq \E(Y \mid X=x)$.
Then $\Pr(A \mid X) \defeq g(X)$ and $\E(Y \mid X) \defeq h(X)$.
\end{definition}

\begin{theorem}
$\E(\Pr(A \mid X)) = \Pr(A)$ and $\E(\E(Y \mid X)) = \E(Y)$.
\end{theorem}

\begin{theorem}
$\Var(Y) = \E(\Var(Y \mid X)) + \Var(\E(Y \mid X))$.
\end{theorem}

\section{Binomial Coefficient}

The binomial coefficient $\binom{n}{k}$ is the number of subsets of $\{1, 2, \ldots, n\}$ of size $k$,
where $n \in \mathbb{Z}_{\ge 0}$ and $k \in \mathbb{Z}$.

\begin{itemize}
\item $\displaystyle \binom{n}{k} = \binom{n}{n-k} = \begin{cases}
    0 & \textrm{ if } k < 0 \textrm{ or } k > n
    \\ \frac{n!}{k!(n-k)!} & \textrm{ if } 0 \le k \le n \end{cases}$.
\item Additive recursion: For $n \ge 1$, $\displaystyle \binom{n}{k}
    = \binom{n-1}{k} + \binom{n-1}{k-1}
    = \binom{n+1}{k+1} - \binom{n}{k+1}$.
\item Decrement: For $n \ge 1$, $\displaystyle \binom{n}{k}
    = \frac{n}{k} \binom{n-1}{k-1}
    = \frac{n}{n-k} \binom{n-1}{k}
    = \frac{n-k+1}{k} \binom{n}{k-1}$.
\item Sum 1: $\displaystyle \sum_{i=k}^n \binom{i}{k}\binom{n}{i}x^i = \binom{n}{k}x^k(1+x)^{n-k}$.
Set $k=0$ to get $\displaystyle \sum_{i=0}^n \binom{n}{i}x^i = (1+x)^n$.
\item Sum 2: $\displaystyle \sum_{i=0}^p \binom{m}{i}\binom{n}{p-i} = \binom{m+n}{p}$.
\item Sum 3: $\displaystyle \sum_{i=k}^{n-b} \binom{i}{k}\binom{n-i}{b} = \binom{n+1}{k+b+1}$.
Set $b=0$ to get $\displaystyle \sum_{i=k}^{n} \binom{i}{k} = \binom{n+1}{k+1}$.
\end{itemize}

\section{Other useful results}

\[ \forall x \in \mathbb{R},\quad e^x \ge 1 + x. \]
\[ \forall x > 0, \quad \frac{x-1}{x} \le \ln x \le x-1. \]
\[ \forall n \ge 1, \quad \left(\sum_{i=1}^n \frac{1}{i}\right) - \ln n \in [1/n, 1]. \]
\[ \textrm{Stirling's approximation: For } n \ge 1,\quad
    \frac{n!}{n^{n+\frac{1}{2}}e^{-n}} \in [\sqrt{2\pi}, e]. \]
\[ \forall a \in \mathbb{\mathbb{Z}}, \forall b \in \mathbb{Z}_{>0},\quad
    \ceil{\frac{a}{b}} = \floor{\frac{a-1}{b}}+1 \quad\textrm{and}\quad
    \floor{\frac{a}{b}} = \ceil{\frac{a+1}{b}}-1. \]

\begin{theorem}[Generalization of Geometric series]
For $0 \le a \le b$,
\[ \sum_{i=0}^{\infty} \binom{b+i}{a}p^i
    = \frac{1}{1-p}\sum_{i=0}^a \binom{b}{i}\left(\frac{p}{1-p}\right)^{a-i}. \]
On setting $b = a$, we get
\[ \sum_{i=0}^{\infty} \binom{a+i}{a}p^i = \frac{1}{(1-p)^{a+1}}. \]
\end{theorem}

%\addMyBib{}

\end{document}

\documentclass[a4paper, 12pt, fleqn]{article}

\usepackage{amsmath,amsthm,amssymb}
\usepackage[margin=1in]{geometry}
\usepackage{xcolor}
\usepackage{array}
\usepackage{comment}
\usepackage{booktabs}
\usepackage[bookmarksnumbered=true,hypertexnames=false]{hyperref}
%\usepackage{algorithm, algpseudocode}
\usepackage[capitalize,sort]{cleveref}

%\def\colorscheme{dark}
\def\colorschemesepia{sepia}
\def\colorschemedark{dark}
\def\colorschemelight{light}

\ifx\colorscheme\undefined
\let\colorscheme\colorschemelight
\fi

\ifx\colorscheme\colorschemelight
\colorlet{textColor}{black}
\colorlet{bgColor}{white}
\fi

\ifx\colorscheme\colorschemesepia
\definecolor{textColor}{HTML}{433423}
\definecolor{bgColor}{HTML}{fbf0da}
\fi

\ifx\colorscheme\colorschemedark
\definecolor{textColor}{HTML}{bdc1c6}
\definecolor{bgColor}{HTML}{202124}
\definecolor{textBlue}{HTML}{8ab4f8}
\definecolor{textRed}{HTML}{f9968b}
\colorlet{bgRed}{red!40!black}
\colorlet{bgBlue}{red!40!black}
\else
\colorlet{textBlue}{blue!50!black}
\colorlet{textRed}{red!50!black}
\fi

\ifx\colorscheme\colorschemelight\else
\pagecolor{bgColor}
\color{textColor}
\fi

\hypersetup{colorlinks,linkcolor=textRed,citecolor=textRed,urlcolor=textBlue}
% THEOREMS

\newtheorem{theorem}{Theorem}
\newtheorem{definition}{Definition}
\newtheorem{example}{Example}
\newtheorem{corollary}{Corollary}[theorem]
\newtheorem{lemma}[theorem]{Lemma}

% SHORTHANDS

\newcommand*{\Th}{^{\textrm{th}}}
\newcommand*{\WLoG}{Without loss of generality}
\newcommand*{\wLoG}{without loss of generality}
\newcommand*{\wrt}{with respect to}

% SYMBOLS

\let\eps\epsilon
\newcommand*{\defeq}{:=}

\newcommand*{\Ahat}{\widehat{A}}
\newcommand*{\Bhat}{\widehat{B}}
\newcommand*{\Chat}{\widehat{C}}
\newcommand*{\Jhat}{\widehat{J}}
\newcommand*{\Shat}{\widehat{S}}
\newcommand*{\Yhat}{\widehat{Y}}
\newcommand*{\bhat}{\widehat{b}}
\newcommand*{\what}{\widehat{w}}
\newcommand*{\xhat}{\widehat{x}}
\newcommand*{\yhat}{\widehat{y}}
\newcommand*{\zhat}{\widehat{z}}

\newcommand*{\Btild}{\widetilde{B}}
\newcommand*{\Ctild}{\widetilde{C}}
\newcommand*{\Jtild}{\widetilde{J}}
\newcommand*{\Stild}{\widetilde{S}}
\newcommand*{\Ytild}{\widetilde{Y}}
\newcommand*{\btild}{\widetilde{b}}
\newcommand*{\wtild}{\widetilde{w}}
\newcommand*{\xtild}{\widetilde{x}}
\newcommand*{\ytild}{\widetilde{y}}
\newcommand*{\ztild}{\widetilde{z}}

\newcommand*{\Fcal}{\mathcal{F}}
\newcommand*{\Tcal}{\mathcal{T}}

% MATH

\newcommand*{\floor}[1]{\left\lfloor #1 \right\rfloor}
\newcommand*{\smallfloor}[1]{\lfloor #1 \rfloor}
\newcommand*{\ceil}[1]{\left\lceil #1 \right\rceil}
\newcommand*{\smallceil}[1]{\lceil #1 \rceil}
\newcommand*{\abs}[1]{\left\lvert #1 \right\rvert}
\newcommand*{\smallabs}[1]{\lvert #1 \rvert}
\newcommand*{\norm}[1]{\left\lVert #1 \right\rVert}
\newcommand*{\smallnorm}[1]{\lVert #1 \rVert}
\newcommand*{\Z}{\mathbb{Z}}
\DeclareMathOperator*{\E}{E}
\DeclareMathOperator*{\Var}{Var}
\DeclareMathOperator*{\argmin}{argmin}
\DeclareMathOperator*{\argmax}{argmax}
\DeclareMathOperator{\poly}{poly}
\DeclareMathOperator{\opt}{opt}
\DeclareMathOperator{\support}{support}
\newcommand*{\OPT}{\mathrm{OPT}}


\author{Eklavya Sharma}
\date{\empty}

\title{Parameter Estimation}

\begin{document}

\maketitle
\setlength{\parskip}{0.2em}

Our aim is to find out something about a distribution by observing a sample.

\begin{definition}[Sample]
For a distribution $D$, a \emph{sample of size $n$} from $D$
is the sequence $[X_1, X_2, \ldots, X_n]$ of $n$ IID random variables,
each having distribution $D$.
\end{definition}

\paragraph{Notation:}
For a random variable $X$ having distribution $D$
and any function $g$, define $\E(g(D)) \defeq \E(g(X))$.
(Hence, $\Var(D) \defeq \Var(X)$.)

\section{Bias and Variance of Estimators}

\begin{definition}[Sample mean and variance]
Let $[X_1, \ldots, X_n]$ be a sample.
\begin{enumerate}
\item The mean of the sample is defined as
    $\displaystyle \Xbar \defeq \frac{1}{n}\sum_{i=1}^n X_i$.
\item The variance of the sample is defined as
    $\displaystyle V_X \defeq \frac{1}{n-1}\sum_{i=1}^n (X_i-\Xbar)^2$.
\item The standard-deviation of the sample is defined as $S_X \defeq \sqrt{V_X}$.
\end{enumerate}
\end{definition}

\begin{theorem}
Let $\Xbar$ be the mean of a sample from $D$.
Then $\E(\Xbar) = \E(D)$ and $\Var(\Xbar) = \Var(D)/n$.
\end{theorem}

\begin{claim}
\label{thm:svar-expand}
Let $\Xbar$ and $S^2$ be the mean and variance, respectively, of sample $[X_1, \ldots, X_n]$.
Let $a$ be any random variable (or a constant). Then
\[ S^2 = \frac{1}{n-1}\left(\sum_{i=1}^n (X_i-a)^2 - n(\Xbar-a)^2\right). \]
(Note that setting $a = \Xbar$ gives the definition of $S^2$.)
\end{claim}

\begin{theorem}
Let $V$ be the variance of sample $[X_1, \ldots, X_n]$ from $D$.
Let $\mu \defeq \E(D)$ and $\sigma^2 \defeq \Var(D)$.
Then $\E(V) = \sigma^2$ and
$\Var(V) = \frac{\E((D-\mu)^4)}{n} - \frac{\sigma^4(n-3)}{n(n-1)}$.
\end{theorem}
\begin{proof}
\begin{align*}
\E(V) &= \frac{1}{n-1}\left(\sum_{i=1}^n \E((X_i-\mu)^2) - n\E((\Xbar-\mu)^2)\right)
    \tag{by \cref{thm:svar-expand}}
\\ &= \frac{1}{n-1}\left(\sum_{i=1}^n \Var(X_i) - n\Var(\Xbar)\right) = \sigma^2.
\end{align*}
The expression for $\Var(V)$ is from \cite{se.math.73080}.
\end{proof}

\section{Distribution of Estimators}

\begin{definition}
Let $Z$ be a random variable and $S \defeq [X_1, X_2, \ldots]$ be an infinite sequence
of random variables. We say that $S$ converges to $Z$ if
$\lim_{n \to \infty} F_{X_n}(x) = F_Z(x)$ for all $x \in \mathbb{R}$ where $F_Z$ is continuous.
\end{definition}

\begin{theorem}[Central Limit Theorem]
Let $X_1, X_2, \ldots$ be IID randvars having mean $\mu$ and variance $\sigma^2$.
Let $\Xbar_n \defeq \frac{1}{n}\sum_{i=1}^n X_i$.
Let $Y_n \defeq \sqrt{n/\sigma}(\Xbar_n - \mu)$.
Then $[Y_1, Y_2, \ldots]$ converges to $N(0, 1)$.
\end{theorem}

\begin{lemma}[Scaling normal]
\label{thm:scaling-normal}
Let $X \sim N(\mu, \sigma)$. Then for any constants $a$ and $b$,
$aX + b \sim N(a\mu + b, |b|\sigma)$.
\end{lemma}

\begin{lemma}[\cite{wikipedia.sum-of-normal}]
\label{thm:sum-of-ind-normal}
Let $X$ and $Y$ be independent randvars where
$X \sim N(\mu_X, \sigma_X)$ and $Y \sim N(\mu_Y, \sigma_Y)$.
Then $X + Y \sim N(\mu_X + \mu_Y, \sqrt{\sigma_X^2 + \sigma_Y^2})$.
\end{lemma}

\begin{theorem}
\label{thm:normal-smean-svar}
Let $[X_1, \ldots, X_n]$ be a sample from $N(\mu, \sigma)$.
Let $\Xbar$ and $S^2$ be the mean and variance of the sample. Then
\begin{enumerate}
\item\label{item:normal-smean-svar:smean}$\Xbar \sim N(\mu, \sigma/\sqrt{n})$.
\item\label{item:normal-smean-svar:svar}$\frac{n-1}{\sigma^2}S^2 \sim \chi^2(n-1)$.
\item\label{item:normal-smean-svar:indep}$\Xbar$ and $S^2$ are independent.
\end{enumerate}
Here $\chi^2(n-1)$ is the
\href{https://en.wikipedia.org/wiki/Chi-squared_distribution}{Chi-Squared distribution}
with $n-1$ degrees of freedom.
\end{theorem}
\begin{proof}
Part \ref{item:normal-smean-svar:smean} follows from \cref{thm:scaling-normal,thm:sum-of-ind-normal}.

\cite{duke.stat.mean-var-indep} proves parts \ref{item:normal-smean-svar:svar}
and \ref{item:normal-smean-svar:indep}.
Alternatively, \cite{se.math.50598} proves part \ref{item:normal-smean-svar:indep}
and \cite{psu.stat414.svar} proves part \ref{item:normal-smean-svar:svar}.
\end{proof}

\section{Distribution of Statistical Scores}

\begin{definition}
Let $Z \sim N(0, 1)$ and $U \sim \chi^2(r)$ be independent randvars.
Let $T \defeq Z/\sqrt{U/r}$. Then $T$'s distribution is called the
\emph{Student's $t$ distribution} with $r$ degrees of freedom.
\end{definition}

\begin{lemma}[$t$ distribution is symmetric]
Let $T \sim t(r)$. Then $T$ and $-T$ have the same distribution.
\end{lemma}
\begin{proof}
Let $Z \sim N(0, 1)$ and $U \sim \chi^2(r)$ be independent randvars
and $T \defeq Z/\sqrt{U/r}$. Then $T \sim t(r)$.
Since $-Z \sim N(0, 1)$, so $-T = (-Z)/\sqrt{U/r} \sim t(r)$.
\end{proof}

\begin{lemma}[Implications of symmetry]
Let $X$ be a continuous random variable such that $X$ and $-X$ have the same distribution.
Then, $\forall x \in \mathbb{R}$, we get $F_X(x) + F_X(-x) = 1$,
and $\forall \alpha \in [0, 1]$, we get $F_X^{-1}(\alpha) + F_X^{-1}(1-\alpha) = 0$.
\end{lemma}
\begin{proof}
$F_X(-x) = F_{-X}(-x) = \Pr(-X \le -x) = \Pr(X \ge x)
= 1 - F_X(x)$.

Let $x = F_X^{-1}(\alpha)$. Then
\\ $-F_X^{-1}(1-\alpha) = -F_X^{-1}(1 - F_X(x))
= -F_X^{-1}(F_X(-x)) = x = F_X^{-1}(\alpha)$.
\end{proof}

\begin{theorem}
Let $\Xbar$ and $S^2$ be the mean and variance of a sample from $N(\mu, \sigma)$. Then
\[ \frac{\Xbar - \mu}{S/\sqrt{n}} \sim t(n-1). \]
\end{theorem}
\begin{proof}[Proof sketch]
Use \cref{thm:normal-smean-svar} and
$\displaystyle \frac{\frac{\Xbar - \mu}{\sigma/\sqrt{n}}}{\sqrt{\frac{(n-1)S^2/\sigma^2}{n-1}}}
= \frac{\Xbar - \mu}{S/\sqrt{n}}$.
\end{proof}

\section{Distribution of Paired Statistical Scores}

\begin{theorem}
Let $\Xbar$ and $S_X^2$ be the mean and variance of a sample $[X_1, \ldots, X_n]$
from distribution $N(\mu_X, \sigma)$.
Let $\Ybar$ and $S_Y^2$ be the mean and variance of sample $[Y_1, \ldots, Y_m]$
from distribution $N(\mu_Y, \sigma)$. The two samples are independent.
Then for
\begin{align*}
S_p^2 &\defeq \frac{(n-1)S_X^2 + (m-1)S_Y^2}{n+m-2},
& T &\defeq \frac{(\Xbar-\Ybar) - (\mu_X-\mu_Y)}{S_p\sqrt{\frac{1}{n} + \frac{1}{m}}},
\end{align*}
we have $T \sim t(n+m-2)$. ($S_p^2$ is called \emph{pooled sample variance}.)
\end{theorem}
\begin{proof}[Proof sketch]
$\Xbar, \Ybar, S_X, S_Y$ are independent by
\cref{thm:normal-smean-svar}.\ref{item:normal-smean-svar:indep}.
\begin{align*}
& \Xbar \sim N(\mu_X, \sigma/\sqrt{n}) \quad\textrm{and}\quad \Ybar \sim N(\mu_Y, \sigma/\sqrt{m})
    \tag{by \cref{thm:normal-smean-svar}.\ref{item:normal-smean-svar:smean}}
\\ &\implies \frac{(\Xbar - \Ybar) - (\mu_X - \mu_Y)}{\sigma\sqrt{\frac{1}{n} + \frac{1}{m}}} \sim N(0, 1).
    \tag{by \cref{thm:scaling-normal,thm:sum-of-ind-normal}}
\end{align*}
\begin{align*}
& (n-1)S_X^2/\sigma^2 \sim \chi^2(n-1) \quad\textrm{and}\quad (m-1)S_Y^2/\sigma^2 \sim \chi^2(m-1)
    \tag{by \cref{thm:normal-smean-svar}.\ref{item:normal-smean-svar:svar}}
\\ &\implies (n+m-2)S_p^2/\sigma^2 \sim \chi^2(n+m-2).
\qedhere \end{align*}
\end{proof}

\begin{lemma}
\label{thm:svar-lincomb}
For $i \in \{1, \ldots, k\}$, let $\mathbf{X}_i \defeq [X_{i,1}, \ldots, X_{i,n_i}]$
be a sample from $N(\mu_i, \sigma_i)$. The samples are independent.
Let $a_1, \ldots, a_k$ be non-negative constants.
Let $S_i^2$ be the variance of $\mathbf{X}_i$. Let
\begin{align*}
r &\defeq \frac{\left(\sum_{i=1}^k a_iS_i^2\right)^2}{\sum_{i=1}^k \frac{(a_iS_i^2)^2}{n_i-1}}
& L &\defeq \frac{r}{\sum_{i=1}^k a_i\sigma_i^2}\sum_{i=1}^k a_iS_i^2.
\end{align*}
Then $L$ is approximately distributed $\chi^2(r)$.
\end{lemma}
\begin{proof}
The meaning of \emph{approximate} and the `proof' can be found at \cite{se.math.3189589,welch}.
\end{proof}

\begin{theorem}
Let $\Xbar$ and $S_X^2$ be the mean and variance of a sample $[X_1, \ldots, X_n]$
from distribution $N(\mu_X, \sigma_X)$.
Let $\Ybar$ and $S_Y^2$ be the mean and variance of sample $[Y_1, \ldots, Y_m]$
from distribution $N(\mu_Y, \sigma_Y)$.
The samples $[X_1, \ldots, X_n]$ and $[Y_1, \ldots, Y_m]$ are independent.
Then for
\[ r \defeq \frac{(S_X^2/n + S_Y^2/m)^2}{\frac{(S_X^2/n)^2}{n-1} + \frac{(S_Y^2/m)^2}{m-1}}
\qquad\textrm{and}\qquad T \defeq \frac{(\Xbar-\Ybar) - (\mu_X-\mu_Y)}{\sqrt{S_X^2/n + S_Y^2/m}}, \]
$T$ approximately follows $t(r)$.
\end{theorem}
\begin{proof}[Proof sketch]
$T = Z/(\sqrt{L/r})$, where
\begin{align*}
Z &\defeq \frac{(\Xbar-\Ybar) - (\mu_X-\mu_Y)}{\sqrt{\sigma_X^2/n + \sigma_Y^2/m}} \sim N(0, 1),
& L &\defeq \frac{r}{\sigma_X^2/n + \sigma_Y^2/m}\left(\frac{S_X^2}{n} + \frac{S_Y^2}{m}\right),
\end{align*}
and $L$ approximately follows $\chi^2(r)$ by \cref{thm:svar-lincomb}.
\end{proof}

\begin{definition}
Let $X$ and $Y$ be independent randvars, where $X \sim \chi^2(u)$ and $Y \sim \chi^2(v)$.
Then the distribution of $\displaystyle \frac{X/u}{Y/v}$ is called the F distribution
with parameters $u$ and $v$.
\end{definition}

\begin{lemma}
Let $R$ be an $F$ distribution with parameters $u$ and $v$.
Then $R^{-1}$ is an $F$ distribution with parameters $v$ and $u$.
Furthermore, $\forall x \in \mathbb{R}_{> 0}$, we get $F_R(x) + F_{R^{-1}}(x^{-1}) = 1$,
and $\forall \alpha \in [0, 1]$, we get $F_R^{-1}(\alpha)F_{R^{-1}}^{-1}(1-\alpha) = 1$.
\end{lemma}
\begin{proof}
$F_{R^{-1}}(x^{-1}) = \Pr(R^{-1} \le x^{-1}) = \Pr(R \ge x) = 1 - F_R(x)$.

Let $x \defeq F_R^{-1}(\alpha)$. Then
\\ $F_{R^{-1}}^{-1}(1-\alpha) = F_{R^{-1}}^{-1}(1 - F_R(x))
= F_{R^{-1}}^{-1}(F_{R^{-1}}(x^{-1})) = x^{-1} = 1/F_R^{-1}(\alpha)$.
\end{proof}

\bibliographystyle{plainurl}
\bibliography{bibdb.bib}

\end{document}

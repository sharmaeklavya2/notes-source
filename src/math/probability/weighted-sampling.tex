\documentclass[a4paper, 12pt, fleqn]{article}

\usepackage{amsmath,amsthm,amssymb}
\usepackage[margin=1in]{geometry}
\usepackage{xcolor}
\usepackage{array}
\usepackage{comment}
\usepackage{booktabs}
\usepackage[bookmarksnumbered=true,hypertexnames=false]{hyperref}
\usepackage{algorithm, algpseudocode}
\usepackage[capitalize,sort]{cleveref}

\def\colorschemesepia{sepia}
\def\colorschemedark{dark}
\def\colorschemelight{light}

\ifx\colorscheme\undefined
\let\colorscheme\colorschemelight
\fi

\ifx\colorscheme\colorschemelight
\colorlet{textColor}{black}
\colorlet{bgColor}{white}
\fi

\ifx\colorscheme\colorschemesepia
\definecolor{textColor}{HTML}{433423}
\definecolor{bgColor}{HTML}{fbf0da}
\fi

\ifx\colorscheme\colorschemedark
\definecolor{textColor}{HTML}{bdc1c6}
\definecolor{bgColor}{HTML}{202124}
\definecolor{textBlue}{HTML}{8ab4f8}
\definecolor{textRed}{HTML}{f9968b}
\colorlet{bgRed}{red!40!black}
\colorlet{bgBlue}{red!40!black}
\else
\colorlet{textBlue}{blue!50!black}
\colorlet{textRed}{red!50!black}
\fi

\ifx\colorscheme\colorschemelight\else
\pagecolor{bgColor}
\color{textColor}
\fi

\hypersetup{colorlinks,linkcolor=textRed,citecolor=textRed,urlcolor=textBlue}
% THEOREMS

\newtheorem{theorem}{Theorem}
\newtheorem{definition}{Definition}
\newtheorem{example}{Example}
\newtheorem{corollary}{Corollary}[theorem]
\newtheorem{lemma}[theorem]{Lemma}
\newtheorem{claim}[theorem]{Claim}

% SHORTHANDS

\newcommand*{\Th}{^{\textrm{th}}}
\newcommand*{\WLoG}{Without loss of generality}
\newcommand*{\wLoG}{without loss of generality}
\newcommand*{\wrt}{with respect to}
\let\opname\operatorname

% SYMBOLS

\let\eps\epsilon
\newcommand*{\defeq}{:=}

\newcommand*{\alphahat}{\widehat{\alpha}}
\newcommand*{\betahat}{\widehat{\beta}}
\newcommand*{\Ahat}{\widehat{A}}
\newcommand*{\Bhat}{\widehat{B}}
\newcommand*{\Chat}{\widehat{C}}
\newcommand*{\Jhat}{\widehat{J}}
\newcommand*{\Shat}{\widehat{S}}
\newcommand*{\Yhat}{\widehat{Y}}
\newcommand*{\bhat}{\widehat{b}}
\newcommand*{\shat}{\widehat{s}}
\newcommand*{\what}{\widehat{w}}
\newcommand*{\xhat}{\widehat{x}}
\newcommand*{\yhat}{\widehat{y}}
\newcommand*{\zhat}{\widehat{z}}

\newcommand*{\Btild}{\widetilde{B}}
\newcommand*{\Ctild}{\widetilde{C}}
\newcommand*{\Jtild}{\widetilde{J}}
\newcommand*{\Stild}{\widetilde{S}}
\newcommand*{\Ytild}{\widetilde{Y}}
\newcommand*{\btild}{\widetilde{b}}
\newcommand*{\stild}{\widetilde{s}}
\newcommand*{\wtild}{\widetilde{w}}
\newcommand*{\xtild}{\widetilde{x}}
\newcommand*{\ytild}{\widetilde{y}}
\newcommand*{\ztild}{\widetilde{z}}

\newcommand*{\Fcal}{\mathcal{F}}
\newcommand*{\Tcal}{\mathcal{T}}

\newcommand*{\ebar}{\overline{e}}
\newcommand*{\Xbar}{\overline{X}}
\newcommand*{\xbar}{\overline{x}}
\newcommand*{\Ybar}{\overline{Y}}
\newcommand*{\ybar}{\overline{y}}

% MATH

\newcommand*{\floor}[1]{\left\lfloor #1 \right\rfloor}
\newcommand*{\smallfloor}[1]{\lfloor #1 \rfloor}
\newcommand*{\ceil}[1]{\left\lceil #1 \right\rceil}
\newcommand*{\smallceil}[1]{\lceil #1 \rceil}
\newcommand*{\abs}[1]{\left\lvert #1 \right\rvert}
\newcommand*{\smallabs}[1]{\lvert #1 \rvert}
\newcommand*{\norm}[1]{\left\lVert #1 \right\rVert}
\newcommand*{\smallnorm}[1]{\lVert #1 \rVert}
\newcommand*{\Z}{\mathbb{Z}}
\DeclareMathOperator*{\E}{E}
\DeclareMathOperator*{\Var}{Var}
\DeclareMathOperator*{\argmin}{argmin}
\DeclareMathOperator*{\argmax}{argmax}
\DeclareMathOperator{\poly}{poly}
\DeclareMathOperator{\opt}{opt}
\DeclareMathOperator{\argopt}{argopt}
\DeclareMathOperator{\support}{support}
\newcommand*{\OPT}{\mathrm{OPT}}


\author{Eklavya Sharma}
\date{\empty}

\title{How Much Randomness is Needed to Sample from a Discrete Distribution?}

\DeclareMathOperator{\sampleFromReal}{\mathtt{sampleFromReal}}
\DeclareMathOperator{\sampleFromDie}{\mathtt{sampleFromDie}}
\DeclareMathOperator{\throwDie}{\mathtt{throwDie}}
\DeclareMathOperator{\unif}{unif}
\DeclareMathOperator{\len}{len}
\newcommand*{\Null}{\mathtt{null}}
\newcommand*{\boolone}{:=}

\begin{document}

\maketitle
\setlength{\parskip}{0.2em}

There is a discrete distribution $\Dcal$ of support $S \subseteq \mathbb{Z}$.
We want to sample from $\Dcal$, and our only source of randomness is a
(potentially unfair) $k$-sided die ($k \ge 2$), where the probability of each face is known.

For $i \in \{1, \ldots, k\}$, let $q_i$ be the probability that the die shows $i$.
\WLoG{} assume $0 < q_1 \le q_2 \le \ldots \le q_k$.
For $i \in \{0, \ldots, k\}$, let $Q_i \defeq \sum_{j=1}^i q_j$.

\section{Algorithm and Notation}

Let $Z$ be a random variable from $\Dcal$.
For any $i \in S$, let $p_i \defeq \Pr(Z = i)$
and let $I_i$ be the half-open interval $(\Pr(Z < i), \Pr(Z \le i)]$.

\begin{lemma}
$(0, 1) \subseteq \bigcup_{i \in S} I_i \subseteq [0, 1]$.
\end{lemma}
\begin{proof}
(TODO)
\end{proof}

\begin{algorithm}[htb]
\caption{$\sampleFromDie$: Sample from $\Dcal$, where $\throwDie \sim \unif(\{1, 2, \ldots, k\}$).}
\begin{algorithmic}[1]
\State Set $\ell = 0$ and $r = 1$.
\For{$t \in \mathbb{Z}_{\ge 0}$}
    \If{$(\ell, r] \subseteq I_i$ for some $i \in S$}
        \State \Return ($i$, $t$)
    \EndIf
    \State $v = \throwDie()$
    \State $\ell, r = \ell + Q_{v-1}(r - \ell), \ell + Q_v(r - \ell)$
\EndFor
\end{algorithmic}
\end{algorithm}

Let $(Y, T)$ $\sampleFromDie$'s output.
Here $Y$ is the random sample from $\Dcal$ we are looking for
and $T$ is the number of times $\throwDie$ is called.

For $t \in \mathbb{Z}_{\ge 1}$, let $v_t$ be the value of the $t\Th$ die roll.
For $t \in \mathbb{Z}_{\ge 0}$, let $\ell_t$ and $r_t$ be the value of $\ell$ and $r$
in $\sampleFromDie$ after $t$ iterations have completed.
Let $J_t \defeq (\ell_t, r_t]$.
(The algorithm may terminate before the $t\Th$ iteration completes.
However, we look at the hypothetical scenario where we continue running the algorithm
instead of terminating at return statements.
Hence, $v_t$, $\ell_t$, and $r_t$ are well-defined for all $t$.)

Let $\Jcal_t$ be the support of $J_t$.

\section{Correctness}

For any interval $I$, let $\len(I)$ be its length.

\begin{lemma}
For all $t \in \mathbb{Z}_{\ge 0}$ and $W_t \in \Jcal_t$, we have $\Pr(J_t = W_t) = \len(W_t)$.
\end{lemma}
\begin{proof}
(TODO)
\end{proof}

\begin{theorem}
Let $Y$ be $\sampleFromDie$'s output. Then $\Pr(Y = i) = p_i$ for all $i \in S$.
\end{theorem}
\begin{proof}
(TODO)
\end{proof}

\section{Running Time for Bounded \texorpdfstring{$|S|$}{|S|}}

\WLoG{}, assume $S = \{1, 2, \ldots, n\}$.
For $i \in \{0, 1, \ldots, n\}$, let $s_i \defeq \Pr(Z \le i)$.
For any proposition $P$, let $\boolone(P)$ be 1 if $P$ is true and 0 if $P$ is false.

\begin{lemma}
For all $t \in \mathbb{Z}_{\ge 0}$ and $W_t \in \Jcal_t$, we have $\len(W_t) \in [q_1^t, q_k^t]$.
\end{lemma}
\begin{proof}
(TODO)
\end{proof}

\begin{lemma}
$\Pr(T > t) \le \min(q_k^t(n-1), 1)$.
\end{lemma}
\begin{proof}
Let $C \defeq \{s_1, \ldots, s_{n-1}\}$.
Then $T > t \iff C \cap J_t \neq \emptyset$.
For any $W \in \Jcal_t$, let $n_W \defeq |C \cap W|$. Then
\begin{align*}
\Pr(T > t) &= \Pr(C \cap J_t \neq \emptyset)
\\ &= \sum_{W \in \Jcal_t} \Pr(J_t = W)\boolone(n_W \ge 1)
\\ &\le \sum_{W \in \Jcal_t} q_k^t n_W
\\ &= q_k^t \sum_{W \in \Jcal_t} |C \cap W|
\\ &= q_k^t (n-1).
\qedhere \end{align*}
\end{proof}

\begin{corollary}
Let $\delta \in (0, 1)$ and
$\displaystyle t \defeq \ceil{\frac{\log(n-1) + \log(1/\delta)}{\log(1/q_k)}}$.
Then $\Pr(T > t) \le \delta$.
\end{corollary}

\begin{theorem}
Let $\displaystyle h \defeq \ceil{\frac{\log(n-1)}{\log(1/q_k)}}$. Then
$\displaystyle \E(T) \le h + \frac{(n-1)q_k^h}{1 - q_k} \le h + \frac{1}{1 - q_k}$.
\end{theorem}
\begin{proof}
For any $t \in \mathbb{Z}_{\ge 0}$,
\begin{align*}
t \ge h
&\iff t \ge \frac{\log(n-1)}{\log(1/q_k)}
\iff t\log(1/q_k) \ge \log(n-1)
\\ &\iff q_k^{-t} \ge n-1
\iff (n-1)q_k^t \le 1.
\end{align*}
Hence,
\begin{align*}
\E(T) &= \sum_{t=0}^{\infty} \Pr(T > t)
\le \sum_{t=0}^{\infty} \min(1, q_k^t(n-1))
\\ &\le h + \sum_{t=h}^{\infty} q_k^t(n-1)
= h + (n-1)q_k^h \sum_{t=0}^{\infty} q_k^t
\\ &= h + \frac{(n-1)q_k^h}{1 - q_k}
\le h + \frac{1}{1 - q_k}.
\qedhere \end{align*}
\end{proof}

If we have a fair $n$-sided die, then we get $\E(T) \le 2$.

\section{Entropy-Based Bound}

All intervals in $\Jcal_t$ are said to have \emph{depth} $t$.

\begin{definition}
Let $\Jcal \defeq \bigcup_{t=0}^{\infty} J_t$. An interval $W \in \Jcal$ is said to be
\emph{maximal in} $I_i$ if $W \subseteq I_i$ and for all $W' \in \Jcal$,
we have $W' \supsetneq W \implies W' \nsubseteq I_i$.

A \emph{die-decomposition} of $I_i$ is the set of all sets in $\Jcal$ that are maximal in $I_i$.
\end{definition}

\begin{lemma}
For any $i \in S$, let $\Kcal_i$ be a die-decomposition of $I_i$.
Then $\bigcup_{W \in \Kcal_i} W = I_i$.
\end{lemma}
\begin{proof}
(TODO)
\end{proof}

\begin{lemma}
Let $(Y, T)$ be the output of $\sampleFromDie$. Then $(\ell_T, r_T) \in \Kcal_Y$.
\end{lemma}
\begin{proof}
(TODO)
\end{proof}

\begin{lemma}
For any $i \in S$ and $t \in \mathbb{Z}_{\ge 0}$, we get
\[ \sum_{W \in \Kcal_i \cap \Jcal_t} \len(W) \le (2 - q_1 - q_k)q_k^{t-1}. \]
\end{lemma}
\begin{proof}
(TODO)
\end{proof}

\begin{lemma}
\[ \E(T) \le \frac{2-q_1-q_k}{1-q_k}\sum_{i \in S} p_i^{\frac{\log(1/q_k)}{\log(1/q_1)}}
    \left(\ceil{\frac{\log(1/p_i)}{\log(1/q_1)}} + \frac{q_k}{1-q_k}\right). \]
\end{lemma}
\begin{proof}
Let $\beta_{i,t} \defeq \sum_{W \in \Kcal_i \cap \Jcal_t} \len(W)$.
Then $\beta_{i,t} \le \min(p_i, (2-q_1-q_k)q_k^{t-1})$.
$\beta_{i,t} > 0 \iff |\Kcal_i \cap \Jcal_t| > 0 \implies q_1^t \le p_i$.
Let $h_i \defeq \ceil{\frac{\log(1/p_i)}{\log(1/q_1)}}$. Then
\[ q_1^t \le p_i \iff t\log(1/q_1) \ge \log(1/p_i) \iff t \ge h_i. \]
Hence, $t < h_i \implies \beta_{i,t} = 0$.

\begin{align*}
\E(T) &= \sum_{i \in S} \sum_{t=0}^{\infty} \E(T | J_T \in \Kcal_i \cap \Jcal_t)\Pr(J_T \in \Kcal_i \cap \Jcal_t)
\\ &= \sum_{i \in S} \sum_{t=0}^{\infty} t\Pr(J_T \in \Kcal_i \cap \Jcal_t)
\\ &= \sum_{i \in S} \sum_{t=0}^{\infty} t\beta_{i,t}
= \sum_{i \in S} \sum_{t=h_i}^{\infty} t\beta_{i,t}
\\ &\le (2-q_1-q_k)\sum_{i \in S} \sum_{t=h_i}^{\infty} tq_k^{t-1}
\\ &= \frac{2-q_1-q_k}{1-q_k}\sum_{i \in S} q_k^{h_i-1}\left(h_i + \frac{q_k}{1-q_k}\right)
\\ &= \frac{2-q_1-q_k}{1-q_k}\sum_{i \in S} p_i^{\frac{\log(1/q_k)}{\log(1/q_1)}}\left(h_i + \frac{q_k}{1-q_k}\right)
\qedhere \end{align*}
\end{proof}

Suppose $q_1 = q_k = 1/k$. Then we get
\begin{align*}
\E(T) &\le \frac{2(1-1/k)}{1-1/k}\sum_{i \in S} p_i\left(\ceil{\log_k(1/p_i)} + \frac{k}{k-1}\right)
\\ &= 2\sum_{i \in S} p_i\ceil{\log_k(1/p_i)} + \frac{2k}{k-1}.
\end{align*}

%\bibliographystyle{plainurl}
%\bibliography{bibdb.bib}

\end{document}

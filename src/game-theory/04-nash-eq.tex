\documentclass[a4paper, 12pt, fleqn]{article}

\usepackage{amsmath}
\usepackage{amsthm}
\usepackage{amssymb}
\usepackage[margin=1in]{geometry}
\usepackage{xcolor}
\usepackage{float}
\usepackage{url}
\usepackage{array}
\usepackage{comment}
\usepackage[bookmarksnumbered=true,hypertexnames=false]{hyperref}
\usepackage{algorithm, algpseudocode}
\usepackage[capitalize,sort]{cleveref}

\hypersetup{
    colorlinks,
    linkcolor={red!50!black},
    citecolor={red!50!black},
    urlcolor={blue!50!black}
}

% THEOREMS

\newtheorem{theorem}{Theorem}
\newtheorem{definition}{Definition}
\newtheorem{example}{Example}
\newtheorem{corollary}{Corollary}[theorem]
\newtheorem{lemma}[theorem]{Lemma}

% SHORTHANDS

\newcommand*{\Th}{^{\textrm{th}}}
\newcommand*{\WLoG}{Without loss of generality}
\newcommand*{\wLoG}{without loss of generality}
\newcommand*{\wrt}{with respect to}

% SYMBOLS

\let\eps\epsilon
\newcommand*{\defeq}{:=}

\newcommand*{\Ahat}{\widehat{A}}
\newcommand*{\Bhat}{\widehat{B}}
\newcommand*{\Chat}{\widehat{C}}
\newcommand*{\Jhat}{\widehat{J}}
\newcommand*{\Shat}{\widehat{S}}
\newcommand*{\Yhat}{\widehat{Y}}
\newcommand*{\bhat}{\widehat{b}}
\newcommand*{\what}{\widehat{w}}
\newcommand*{\xhat}{\widehat{x}}
\newcommand*{\yhat}{\widehat{y}}
\newcommand*{\zhat}{\widehat{z}}

\newcommand*{\Btild}{\widetilde{B}}
\newcommand*{\Ctild}{\widetilde{C}}
\newcommand*{\Jtild}{\widetilde{J}}
\newcommand*{\Stild}{\widetilde{S}}
\newcommand*{\Ytild}{\widetilde{Y}}
\newcommand*{\btild}{\widetilde{b}}
\newcommand*{\wtild}{\widetilde{w}}
\newcommand*{\xtild}{\widetilde{x}}
\newcommand*{\ytild}{\widetilde{y}}
\newcommand*{\ztild}{\widetilde{z}}

\newcommand*{\Fcal}{\mathcal{F}}
\newcommand*{\Tcal}{\mathcal{T}}

% MATH

\newcommand*{\floor}[1]{\left\lfloor #1 \right\rfloor}
\newcommand*{\smallfloor}[1]{\lfloor #1 \rfloor}
\newcommand*{\ceil}[1]{\left\lceil #1 \right\rceil}
\newcommand*{\smallceil}[1]{\lceil #1 \rceil}
\newcommand*{\abs}[1]{\left\lvert #1 \right\rvert}
\newcommand*{\smallabs}[1]{\lvert #1 \rvert}
\newcommand*{\norm}[1]{\left\lVert #1 \right\rVert}
\newcommand*{\smallnorm}[1]{\lVert #1 \rVert}
\newcommand*{\Z}{\mathbb{Z}}
\DeclareMathOperator*{\E}{E}
\DeclareMathOperator*{\Var}{Var}
\DeclareMathOperator*{\argmin}{argmin}
\DeclareMathOperator*{\argmax}{argmax}
\DeclareMathOperator{\poly}{poly}
\DeclareMathOperator{\opt}{opt}
\DeclareMathOperator{\support}{support}
\newcommand*{\OPT}{\mathrm{OPT}}


% INITIALIZATIONS

\newcommand{\initMinimal}{
\setlength{\parindent}{0pt}
\setlength{\parskip}{0.5em}
}
\newcommand{\initFromContents}{
\tableofcontents
\newpage
\initMinimal{}
}
\newcommand{\initAfterBeginDocument}{
\maketitle
\initFromContents{}
}
\newcommand{\addMyBib}{
\bibliographystyle{plainurl}
\bibliography{bibdb}
}

\author{Eklavya Sharma}
\date{\empty}

\usepackage{diagbox}

\title{Nash Equilibrium}

\begin{document}

\maketitle
\initMinimal{}

\begin{definition}
A strategy profile $s^*$ for game $(N, (S_i)_{i \in N}, (u_i)_{i \in N})$
is called a \emph{pure strategy Nash equilibrium} (PSNE) iff
\[ \forall i \in N, u_i(s^*_i, s^*_{-i}) = \max_{a \in S_i} u_i(a, s^*_{-i}) \]
\end{definition}
Equivalently, this means that for every player, unilateral deviations
do not increase utility.

\section{Examples}

\subsection{Coordination game}

\begin{tabular}{|c|c|c|}
\hline
\diagbox{1}{2} & A & B
\\ \hline
A & 100, 100 & 0, 0
\\ \hline
B & 0, 0 & 10, 10
\\ \hline
\end{tabular}

Here $(A, A)$ and $(B, B)$ are PSNE
but $(A, B)$ and $(B, A)$ are not PSNE.

\subsection{Prisoner's dilemma}

\begin{tabular}{|c|c|c|}
\hline
\diagbox{1}{2} & C & B
\\ \hline
C & $-2, -2$ & $-10, -1$
\\ \hline
B & $-1, -10$ & $-5, -5$
\\ \hline
\end{tabular}

Here $(B, B)$ is the only PSNE.

\section{Properties of PSNE}

\begin{theorem}
A very weak DSE is also a PSNE.
\end{theorem}
\begin{proof}
\begin{align*}
& s^* \textrm{ is a very weak DSE}
\\ &\iff \forall i \in N, \; \forall s_i \in S_i, \; \forall s_{-i} \in S_{-i},
    \; u_i(s^*_i, s_{-i}) \ge u_i(s_i, s_{-i})
\\ &\implies \forall i \in N, \; \forall s_i \in S_i,
    \; u_i(s^*_i, s^*_{-i}) \ge u_i(s_i, s^*_{-i})
\\ &\iff s^* \textrm{ is a PSNE}
\qedhere \end{align*}
\end{proof}

\begin{theorem}
If a game contains a strong DSE, then that is the only PSNE.
\end{theorem}
\begin{proof}
Assume the game contains a strong DSE $s^*$ and a PSNE $t^*$ such that $s^* \neq t^*$.
Since $s^* \neq t^*$, there exists a player $i$ such that $s^*_i \neq t^*_i$.
\begin{align*}
& s^* \textrm{ is a strong DSE}
\\ &\implies s^*_i \textrm{ is a strongly dominant strategy for player } i
\\ &\iff \forall s_i \in S_i - \{s_i^*\}, \; \forall s_{-i} \in S_{-i}, \;
    u_i(s_i^*, s_{-i}) > u_i(s_i, s_{-i})
\\ &\implies u_i(s_i^*, t_{-i}^*) > u_i(t_i^*, t_{-i}^*)
\end{align*}
\begin{align*}
& t^* \textrm{ is a PSNE}
\\ &\implies \forall s_i \in S_i, \;
    u_i(t_i^*, t_{-i}^*) \ge u_i(s_i, t_{-i}^*)
\\ &\implies u_i(t_i^*, t_{-i}^*) \ge u_i(s_i^*, t_{-i}^*)
\end{align*}
This is a contradiction, so we cannot have a strong DSE
and a different PSNE together in a game.
\end{proof}

%\addMyBib{}

\end{document}

\documentclass[a4paper, 12pt, fleqn]{article}

\usepackage{amsmath}
\usepackage{amsthm}
\usepackage{amssymb}
\usepackage[margin=1in]{geometry}
\usepackage{xcolor}
\usepackage{float}
\usepackage{url}
\usepackage{array}
\usepackage{comment}
\usepackage[bookmarksnumbered=true,hypertexnames=false]{hyperref}
\usepackage{algorithm, algpseudocode}
\usepackage[capitalize,sort]{cleveref}

\hypersetup{
    colorlinks,
    linkcolor={red!50!black},
    citecolor={red!50!black},
    urlcolor={blue!50!black}
}

% THEOREMS

\newtheorem{theorem}{Theorem}
\newtheorem{definition}{Definition}
\newtheorem{example}{Example}
\newtheorem{corollary}{Corollary}[theorem]
\newtheorem{lemma}[theorem]{Lemma}

% SHORTHANDS

\newcommand*{\Th}{^{\textrm{th}}}
\newcommand*{\WLoG}{Without loss of generality}
\newcommand*{\wLoG}{without loss of generality}
\newcommand*{\wrt}{with respect to}

% SYMBOLS

\let\eps\epsilon
\newcommand*{\defeq}{:=}

% MATH

\newcommand*{\floor}[1]{\left\lfloor #1 \right\rfloor}
\newcommand*{\smallfloor}[1]{\lfloor #1 \rfloor}
\newcommand*{\ceil}[1]{\left\lceil #1 \right\rceil}
\newcommand*{\smallceil}[1]{\lceil #1 \rceil}
\newcommand*{\abs}[1]{\left\lvert #1 \right\rvert}
\newcommand*{\smallabs}[1]{\lvert #1 \rvert}
\newcommand*{\norm}[1]{\left\lVert #1 \right\rVert}
\newcommand*{\smallnorm}[1]{\lVert #1 \rVert}
\newcommand*{\Z}{\mathbb{Z}}
\DeclareMathOperator*{\E}{E}
\DeclareMathOperator*{\Var}{Var}
\DeclareMathOperator*{\argmin}{argmin}
\DeclareMathOperator*{\argmax}{argmax}
\DeclareMathOperator{\poly}{poly}
\DeclareMathOperator{\opt}{opt}
\newcommand*{\OPT}{\mathrm{OPT}}

% INITIALIZATIONS

\newcommand{\initMinimal}{
\setlength{\parindent}{0pt}
\setlength{\parskip}{0.5em}
}
\newcommand{\initFromContents}{
\tableofcontents
\newpage
\initMinimal{}
}
\newcommand{\initAfterBeginDocument}{
\maketitle
\initFromContents{}
}
\newcommand{\addMyBib}{
\bibliographystyle{plainurl}
\bibliography{bibdb}
}

\author{Eklavya Sharma}
\date{\empty}

\usepackage{diagbox}
\usepackage{tikz}
\usetikzlibrary{arrows.meta}

\title{Extensive Form Games}

\begin{document}

\maketitle
\initMinimal{}

We will look at another model for games, called \emph{extensive form games}.
We will then show how it can be reduced to a strategic form game.

An extensive form game captures games where players move sequentially.
Such a game is usually visualized as a tree.

An extensive form game $\Gamma$ is represented as the tuple
$(N, (A_i)_{i \in N}, H, P, (I_i)_{i \in N}, (u_i)_{i \in N})$ where
\begin{itemize}
\item $N$ is the set of players.
\item $A_i$ is the set of actions available to player $i$.
\item A history is a sequence of actions that can be played during the game.
    The state of the game can be represented by the history at that point in time.
\item $H$ is the set of terminal histories, i.e.,
    when the game reaches a state in $H$, the game ends.
    No history in $H$ is a prefix of another history in $H$.
\item $S_H$ is the set of proper subhistories of $H$. Formally, let
    $\operatorname{prefix}(h)$ denote the set of proper prefixes of history $h \in H$.
    Then $S_H \defeq \bigcup_{h \in H} \operatorname{prefix}(h)$.
\item $P: S_H \mapsto N$ is the player function, i.e., when the game is at state $h \in S_H$,
    the player $P(h)$ is supposed to make a move.
\item For $i \in N$, let $H_i$ be the states where $i$ is supposed to make a move,
    i.e., $H_i \defeq \{h \in S_H: P(h) = i\}$. Then $I_i$ is a partition of $H_i$.
    Sets in $I_i$ are called the information sets of player $i$.
    Intuitively, for each $X \in I_i$, player $i$ cannot distinguish the states in $X$.
\item For each $i \in N$, $u_i: H \mapsto \mathbb{R}$ is the utility function of player $i$.
    This means that when the game reaches state $h \in H$,
    each player $i$ will get utility $u_i(h)$.
\item For history $h \in S_H$, where $P(h) = i$, let $C(h) \subseteq A_i$ be the
    actions available to player $i$, i.e., $a \in C(h) \iff h + a \in S_H \cup H$.
    For each $i \in N$ and each $X \in I_i$, $C(h)$ should be the same for each $h \in X$.
    Intuitively, the actions available to player $i$ should be the same
    for all histories in an information set. Denote these actions as $C(X)$.
\end{itemize}

\begin{definition}
A game is said to be a perfect information game
iff for each player $i$, all information sets are singletons.
\end{definition}

\begin{definition}
For a player $i$, $s_i: I_i \mapsto A_i$ is called a \emph{strategy} iff
for each $J \in I_i$, we have $s_i(J) \in C(J)$.
\end{definition}
Intuitively, a strategy is a plan about which action to take for each information set.

\begin{definition}
Let $s_i$ be a strategy of player $i$.
Let $s \defeq (s_i)_{i \in N}$.
Then $s$ is called a \emph{strategy profile} for the game.
Let $S_i$ be all possible strategies for player $i$.
Let $S \defeq S_1 \times S_2 \times \ldots \times S_n$, where $n \defeq |N|$.
Then $S$ is called the \emph{strategy profile collection} for the game.

For a strategy profile $s$, the outcome of the game, denoted by $O(s)$,
is the terminal history reached when the game ends.
For a player $i$ and strategy profile $s$, $u_i(s) \defeq u_i(O(s))$.

The \emph{strategic form equivalent} of $\Gamma$
is the game $(N, (S_i)_{i \in N}, (u_i)_{i \in N})$.
\end{definition}

%\addMyBib{}

\end{document}

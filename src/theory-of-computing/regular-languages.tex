\documentclass[a4paper, 12pt, fleqn]{article}

\usepackage{amsmath}
\usepackage{amsthm}
\usepackage{amssymb}
\usepackage[margin=1in]{geometry}
\usepackage{xcolor}
\usepackage{float}
\usepackage{url}
\usepackage{array}
\usepackage{comment}
\usepackage[bookmarksnumbered=true,hypertexnames=false]{hyperref}
\usepackage{algorithm, algpseudocode}
\usepackage[capitalize,sort]{cleveref}

\hypersetup{
    colorlinks,
    linkcolor={red!50!black},
    citecolor={red!50!black},
    urlcolor={blue!50!black}
}

% THEOREMS

\newtheorem{theorem}{Theorem}
\newtheorem{definition}{Definition}
\newtheorem{example}{Example}
\newtheorem{corollary}{Corollary}[theorem]
\newtheorem{lemma}[theorem]{Lemma}

% SHORTHANDS

\newcommand*{\Th}{^{\textrm{th}}}
\newcommand*{\WLoG}{Without loss of generality}
\newcommand*{\wLoG}{without loss of generality}
\newcommand*{\wrt}{with respect to}

% SYMBOLS

\let\eps\epsilon
\newcommand*{\defeq}{:=}

\newcommand*{\Ahat}{\widehat{A}}
\newcommand*{\Bhat}{\widehat{B}}
\newcommand*{\Chat}{\widehat{C}}
\newcommand*{\Jhat}{\widehat{J}}
\newcommand*{\Shat}{\widehat{S}}
\newcommand*{\Yhat}{\widehat{Y}}
\newcommand*{\bhat}{\widehat{b}}
\newcommand*{\what}{\widehat{w}}
\newcommand*{\xhat}{\widehat{x}}
\newcommand*{\yhat}{\widehat{y}}
\newcommand*{\zhat}{\widehat{z}}

\newcommand*{\Btild}{\widetilde{B}}
\newcommand*{\Ctild}{\widetilde{C}}
\newcommand*{\Jtild}{\widetilde{J}}
\newcommand*{\Stild}{\widetilde{S}}
\newcommand*{\Ytild}{\widetilde{Y}}
\newcommand*{\btild}{\widetilde{b}}
\newcommand*{\wtild}{\widetilde{w}}
\newcommand*{\xtild}{\widetilde{x}}
\newcommand*{\ytild}{\widetilde{y}}
\newcommand*{\ztild}{\widetilde{z}}

\newcommand*{\Fcal}{\mathcal{F}}
\newcommand*{\Tcal}{\mathcal{T}}

% MATH

\newcommand*{\floor}[1]{\left\lfloor #1 \right\rfloor}
\newcommand*{\smallfloor}[1]{\lfloor #1 \rfloor}
\newcommand*{\ceil}[1]{\left\lceil #1 \right\rceil}
\newcommand*{\smallceil}[1]{\lceil #1 \rceil}
\newcommand*{\abs}[1]{\left\lvert #1 \right\rvert}
\newcommand*{\smallabs}[1]{\lvert #1 \rvert}
\newcommand*{\norm}[1]{\left\lVert #1 \right\rVert}
\newcommand*{\smallnorm}[1]{\lVert #1 \rVert}
\newcommand*{\Z}{\mathbb{Z}}
\DeclareMathOperator*{\E}{E}
\DeclareMathOperator*{\Var}{Var}
\DeclareMathOperator*{\argmin}{argmin}
\DeclareMathOperator*{\argmax}{argmax}
\DeclareMathOperator{\poly}{poly}
\DeclareMathOperator{\opt}{opt}
\DeclareMathOperator{\support}{support}
\newcommand*{\OPT}{\mathrm{OPT}}


% INITIALIZATIONS

\newcommand{\initMinimal}{
\setlength{\parindent}{0pt}
\setlength{\parskip}{0.5em}
}
\newcommand{\initFromContents}{
\tableofcontents
\newpage
\initMinimal{}
}
\newcommand{\initAfterBeginDocument}{
\maketitle
\initFromContents{}
}
\newcommand{\addMyBib}{
\bibliographystyle{plainurl}
\bibliography{bibdb}
}

\author{Eklavya Sharma}
\date{\empty}


\title{Regular Languages}

\begin{document}

\initAfterBeginDocument{}


\section{Strings and Languages}

Basic definitions:

\begin{itemize}
\item Alphabet
\item String, Empty string ($e$), Length of a string ($|w|$)
\item Concatenation of 2 strings
\item Suffix, Prefix, Substring
\item Reversal of a string (denoted as $w^R$)
\item Language
\item Lexicographical ordering of strings
\item Shortlex ordering of strings
\end{itemize}

Basic theorems:

\begin{itemize}
\item \begin{theorem}$(xy)^R = y^Rx^R$\end{theorem}
\end{itemize}

Operations on languages:

\begin{itemize}
\item Complement of language $L$ on alphabet $\Sigma$:
    $\overline{L} = \Sigma^* - L$.
\item Union and intersection of languages.
\item Concatenation of languages:
    $L_1L_2 = \{xy: x \in L_1 \wedge y \in L_2\}$.
\item Power of a language:
    $L^0 = \{e\} \wedge L^k = L^{k-1}L$.
\item Kleene-star of a language $L$:
    $L^* = \bigcup\limits_{i \ge 0} L^i$.
\end{itemize}

\subsection{Finite representation of languages}

Every string in $\Sigma^*$ can be mapped to a number. That number is its rank in the shortlex order.
Therefore, $\Sigma^*$ is countably infinite.
Since every language is a subset of $\Sigma^*$, every language is countable.

The set of all languages is $2^{\Sigma^*}$.
It can be proven using diagonalization that there is no bijection from $\Sigma^*$ to $2^{\Sigma^*}$.
Therefore, all languages cannot be represented as a finite string.

\section{Regular expressions}

Let $\Sigma$ be an alphabet.
A regular expression $R$ is a string over
$\Sigma \cup \{\phi, \cup, \texttt{(}, \texttt{)}, ^* \}$
which represents a language over $\Sigma$. $L(R)$ is the language represented by $R$.

\begin{itemize}
\item $\phi$ is a regular expression, where $L(\phi) = \{\}$.
\item $\forall a \in \Sigma$, $a$ is a regular expression where $L(a) = \{a\}$.
\item If $R$ is a regular expression, then so is $\texttt{(}R\texttt{)}$,
    where $L(\texttt{(}R\texttt{)}) = L(R)$.
\item If $R_1$ and $R_2$ are regular expressions, then so is $R_1 \cup R_2$,
    where $L(R_1 \cup R_2) = L(R_1) \cup L(R_2)$.
\item If $R_1$ and $R_2$ are regular expressions, then so is $R_1 R_2$,
    where $L(R_1 R_2) = L(R_1) L(R_2)$.
\item If $R$ is a regular expression, then so is $R^*$,
    where $L(R^*) = L(R)^*$.
\end{itemize}

A language $L'$ is defined to be regular iff there exists a regular expression $R$
such that $L(R) = L'$.

\subsection{Other closure results}

\begin{theorem}
If $L$ is a regular language, then the following are regular:

\begin{itemize}
\item $\operatorname{pref}(L)$: set of all prefixes of all strings in $L$.
\item $\operatorname{suff}(L)$: set of all suffixes of all strings in $L$.
\item $\operatorname{subseq}(L)$: set of all subsequences of all strings in $L$.
\item $L^R$: reverse of all strings in $L$.
\end{itemize}
\end{theorem}

They can all be proven to be regular by finding a regular expression for them.

Let $h: \Sigma \mapsto \Delta^*$ be a function.
Extend $h$ to $\Sigma^* \mapsto \Delta^*$ like this:
\[ h(e) = e \wedge h(aw) = h(a)h(w) \textrm{ where } a \in \Sigma \textrm{ and } w \in \Sigma^* \]
This extension is called an homomorphism. Also define $h(L) = \{h(w): w \in L\}$.

\begin{theorem}$\forall x, y \in \Sigma^*, h(xy) = h(x)h(y)$\end{theorem}
\begin{theorem}If $L$ is regular then $h(L)$ is regular.\end{theorem}

\section{Finite automata}

\subsection[DFA]{Deterministic Finite Automaton (DFA)}

A DFA is a quintuple $M = (Q, \Sigma, \delta, s, F)$ where

\begin{itemize}
\item $Q$ is the set of states.
\item $\Sigma$ is the alphabet.
\item $\delta: Q \times \Sigma \mapsto Q$ is the transition function.
\item $s \in Q$ is the start state.
\item $F \subseteq Q$ are the final states.
\end{itemize}

We can extend $\delta$ to $Q \times \Sigma^* \mapsto Q$ like this:
$\delta(q, e) = q$ and $\delta(q, aw) = \delta(\delta(q, a), w)$,
where $a \in \Sigma$ and $w \in \Sigma^*$.
Intuitively, $\delta(q, w)$ is the state reached by running $M$ on $w$.

\begin{theorem}[Can be proved using induction]
\[ \delta(q, xy) = \delta(\delta(q, x), y)$ where $x, y \in \Sigma^* \]
\end{theorem}

We define $L(M)$ as follows:
$ w \in L(M) \iff \delta(s, w) \in F$.

Automata $M_1$ and $M_2$ are equivalent iff $L(M_1) = L(M_2)$.

\subsection[NFA]{Non-deterministic Finite Automaton (NFA)}

Let $\Sigma? = \Sigma \cup \{e\}$.

An NFA is a quntuple $M = (Q, \Sigma, \Delta, s, F)$.
This is similar to a DFA, except that $\Delta$ is a subset of $Q \times (\Sigma?) \times Q$.

$E(q)$ is the set of states reachable from $q$ by following only $e$~transitions.
For $K \subseteq Q$, $E(K) = \cup_{q \in K} E(q)$.

$w \in L(M) \iff$ there is a path from $s$ to a final state which consumes $w$.

Define $\vdash_M$ as:
\begin{itemize}
\item $\forall w \in \Sigma^*, (p, w) \vdash_M (q, w) \iff (p, e, q) \in \Delta$.
\item $\forall a \in \Sigma, \forall w \in \Sigma^*, (p, aw) \vdash_M (q, w) \iff (p, a, q) \in \Delta$.
\end{itemize}

\begin{theorem}$(p, x) \vdash_M (q, y) \iff (p, xz) \vdash_M (q, yz)$\end{theorem}

Define $\vdash_M^*$ as the reflexive-transitive closure of $\vdash_M$.

We define $L(M)$ as follows:
\[ w \in L(M) \iff (\exists f \in F, (s, w) \vdash_M^* (f, e)) \]

\begin{theorem}
If $h$ is a homomorphism, then $h(\Delta) = \{(p, h(a), q) \in \Delta \}$ is an NFA
accepting $h(L)$.
\end{theorem}

\subsection{NFA to DFA}

\begin{theorem}
Let $M = (K, \Sigma, \Delta, s, F)$ be an NFA.
Then a DFA $M' = (2^K, \Sigma, \delta, E(s), F')$ exists such that $L(M) = L(M')$ where
\[ F' = \{Q \subseteq K: Q \cap F \neq \{\}\} \]
\[ \delta(Q, a) = \bigcup \{E(p): q \in Q \wedge (q, a, p) \in \Delta \} \]
\end{theorem}

This can be proven by using this lemma:
\begin{lemma}
\[ \forall w \in \Sigma^*, \forall p, q \in K, \]
\[ (q, w) \vdash_M^* (p, e) \iff \left(\exists P \supseteq \{p\}, (E(q), w) \vdash_{M'}^* (P, e) \right) \]
\end{lemma}
This lemma can be proved by induction over $|w|$.

\section{Finite automata and regular expressions}

\subsection{Regular expression to finite automaton}

\begin{theorem}
Let $M$, $M_1$ and $M_2$ be finite automata.
Then it is possible to construct finite automata which accept:

\begin{itemize}
\item $L(M_1) \cup L(M_2)$
\item $L(M_1)L(M_2)$
\item $L(M)^*$
\item $\overline{L(M)}$
\item $L(M_1) \cap L(M_2)$
\end{itemize}
\end{theorem}

This proves closure properties and the fact that
a finite automata can be constructed for a regular language.

Intersection and union of DFAs can be constructed
directly by taking the cross-product of states.

\subsection{Finite automaton to regular expression}

To construct a regular expression for an NFA,
we consider a generalization of NFAs, called GNFAs.
A GNFA can have an edge which is a regular expression.

Take an NFA and add a new start and final state, so that
there is no transition into a start state and a single transition out of the start state and
there is a single final state with only a single transition into it and no transition out of it.

Now iteratively remove nodes from the GNFA till only the start and final states remain.
The edge between them is labeled with the required regular expression.

\section{Pumping Lemma}

\begin{theorem}
Let $L$ be a regular language.
Then there exists a positive integer $n$, called a `pumping length' of $L$, such that
\[ \forall w \in L \cap \Sigma^n\Sigma^*,
\exists (x, y, z) \textrm{ such that } \]
\[ w = xyz \wedge y \neq e \wedge |xy| \le n \wedge
(\forall i \ge 0, xy^iz \in L) \]
\end{theorem}

This can be proven by using the fact that if a DFA has $n$ states,
then a string of length $\ge n$ will encounter some state $q$ more than once.
The cycle from $q$ to $q$ can be pumped as many times as we want.

The pumping lemma can be used to prove that a language is irregular,
by showing that for every pumping length, there is a string which cannot be pumped.

Another way of showing that a language is irregular is by showing that its
intersection with a regular language is irregular
(This works because regular languages are closed under intersection).

\end{document}

\documentclass[a4paper,12pt,fleqn]{article}

\usepackage{amsmath,amsthm,amssymb}
\usepackage[margin=1in]{geometry}
\usepackage{xcolor}
\usepackage{array}
\usepackage{comment}
%\usepackage{booktabs}
\usepackage[bookmarksnumbered=true,hypertexnames=false]{hyperref}
%\usepackage{algorithm, algpseudocode}
\usepackage[capitalize,sort]{cleveref}

%\def\colorscheme{dark}
\def\colorschemesepia{sepia}
\def\colorschemedark{dark}
\def\colorschemelight{light}

\ifx\colorscheme\undefined
\let\colorscheme\colorschemelight
\fi

\ifx\colorscheme\colorschemelight
\colorlet{textColor}{black}
\colorlet{bgColor}{white}
\fi

\ifx\colorscheme\colorschemesepia
\definecolor{textColor}{HTML}{433423}
\definecolor{bgColor}{HTML}{fbf0da}
\fi

\ifx\colorscheme\colorschemedark
\definecolor{textColor}{HTML}{bdc1c6}
\definecolor{bgColor}{HTML}{202124}
\definecolor{textBlue}{HTML}{8ab4f8}
\definecolor{textRed}{HTML}{f9968b}
\colorlet{bgRed}{red!40!black}
\colorlet{bgBlue}{red!40!black}
\else
\colorlet{textBlue}{blue!50!black}
\colorlet{textRed}{red!50!black}
\fi

\ifx\colorscheme\colorschemelight\else
\pagecolor{bgColor}
\color{textColor}
\fi

\hypersetup{colorlinks,linkcolor=textRed,citecolor=textRed,urlcolor=textBlue}
% THEOREMS

\newtheorem{theorem}{Theorem}
\newtheorem{definition}{Definition}
\newtheorem{example}{Example}
\newtheorem{corollary}{Corollary}[theorem]
\newtheorem{lemma}[theorem]{Lemma}

% SHORTHANDS

\newcommand*{\Th}{^{\textrm{th}}}
\newcommand*{\WLoG}{Without loss of generality}
\newcommand*{\wLoG}{without loss of generality}
\newcommand*{\wrt}{with respect to}

% SYMBOLS

\let\eps\epsilon
\newcommand*{\defeq}{:=}

\newcommand*{\Ahat}{\widehat{A}}
\newcommand*{\Bhat}{\widehat{B}}
\newcommand*{\Chat}{\widehat{C}}
\newcommand*{\Jhat}{\widehat{J}}
\newcommand*{\Shat}{\widehat{S}}
\newcommand*{\Yhat}{\widehat{Y}}
\newcommand*{\bhat}{\widehat{b}}
\newcommand*{\what}{\widehat{w}}
\newcommand*{\xhat}{\widehat{x}}
\newcommand*{\yhat}{\widehat{y}}
\newcommand*{\zhat}{\widehat{z}}

\newcommand*{\Btild}{\widetilde{B}}
\newcommand*{\Ctild}{\widetilde{C}}
\newcommand*{\Jtild}{\widetilde{J}}
\newcommand*{\Stild}{\widetilde{S}}
\newcommand*{\Ytild}{\widetilde{Y}}
\newcommand*{\btild}{\widetilde{b}}
\newcommand*{\wtild}{\widetilde{w}}
\newcommand*{\xtild}{\widetilde{x}}
\newcommand*{\ytild}{\widetilde{y}}
\newcommand*{\ztild}{\widetilde{z}}

\newcommand*{\Fcal}{\mathcal{F}}
\newcommand*{\Tcal}{\mathcal{T}}

% MATH

\newcommand*{\floor}[1]{\left\lfloor #1 \right\rfloor}
\newcommand*{\smallfloor}[1]{\lfloor #1 \rfloor}
\newcommand*{\ceil}[1]{\left\lceil #1 \right\rceil}
\newcommand*{\smallceil}[1]{\lceil #1 \rceil}
\newcommand*{\abs}[1]{\left\lvert #1 \right\rvert}
\newcommand*{\smallabs}[1]{\lvert #1 \rvert}
\newcommand*{\norm}[1]{\left\lVert #1 \right\rVert}
\newcommand*{\smallnorm}[1]{\lVert #1 \rVert}
\newcommand*{\Z}{\mathbb{Z}}
\DeclareMathOperator*{\E}{E}
\DeclareMathOperator*{\Var}{Var}
\DeclareMathOperator*{\argmin}{argmin}
\DeclareMathOperator*{\argmax}{argmax}
\DeclareMathOperator{\poly}{poly}
\DeclareMathOperator{\opt}{opt}
\DeclareMathOperator{\support}{support}
\newcommand*{\OPT}{\mathrm{OPT}}


\author{Eklavya Sharma}
\date{\empty}

\title{Linear Programming Counterexamples}

\begin{document}

\maketitle
\setlength{\parskip}{0.2em}

\section{Polyhedra}

\begin{example}[Square Pyramid]
\label{ex:pyramid-1}
Let
\begin{align*}
P &\defeq \{(x, y, z): \max(|x|, |y|) \le z, z \le 1\}
\\ &= \{(x, y, z): x \le z, -x \le z, y \le z, -y \le z, z \le 1\}.
\end{align*}
Then $P$ is an (inverted) square pyramid.
The base of the pyramid is $[-1, 1]^2 \times \{1\}$.
Its vertices are $(0, 0, 0)$, $(-1, -1, 1)$, $(-1, 1, 1)$, $(1, -1, 1)$, $(1, 1, 1)$.
\end{example}

\begin{example}[Square Pyramid]
\label{ex:pyramid-2}
Let
\[ P \defeq \{(x, y, z): x \le z, y \le z, x \ge 0, y \ge 0, z \ge 0\}. \]
Then $P$ is an (inverted) square pyramid.
The base of the pyramid is $[0, 1]^2 \times \{1\}$.
Its vertices are $(0, 0, 0)$, $(0, 0, 1)$, $(0, 1, 1)$, $(1, 0, 1)$, $(1, 1, 1)$.
\end{example}

\begin{example}[Introducing degeneracy by changing RHS to 0]
Let $P \defeq \{x: (a_i^Tx = b_i \forall i \in E) \textrm{ and } (a_i^Tx \le b_i \forall i \in E)\}$.
Let $D \defeq \{x: (a_i^Tx = 0 \forall i \in E) \textrm{ and } (a_i^Tx \le 0 \forall i \in E)\}$.
Then $D$ is the set of directions of $P$. If $P$ is bounded, then $D = \{0\}$.
Now all bases of $D$ correspond to the same point.
If $P$ has multiple bases, then $0$ is a degenerate point of $D$.
Furthermore, the simplex method can be made to run on $P$ and $D$ with the same pivots.
\end{example}

\begin{example}[Non-extreme point with $n$ active constraints]
Let $P \defeq \{(x, y): x + y \ge 1, x + y \le 1, x \ge 0, y \ge 0\}$.
No constraint is redundant.
$(1,0)$ and $(0,1)$ are degenerate BFSes of $P$.
Their midpoint, $(1/2, 1/2)$, is not an extreme point, but has 2 active constraints
(which are linearly dependent).
\end{example}

\subsection{Degeneracy vs Redundancy}

\begin{example}[Redundancy $\nRightarrow$ Degeneracy]
Let $P \defeq \{(x, y): 0 \le x \le 1, 0 \le y \le 1\}$.
Then every extreme-point is non-degenerate.
Adding the constraint $y \le 2$ doesn't add degeneracy but adds redundancy.
\end{example}

\begin{example}[Degeneracy $\nRightarrow$ Redundancy]
Let $P \defeq \{(x, y): y \le x, y \le -x, y \ge 0\}$ (so $P = \{(0, 0)\}$).
Then $(0, 0)$ is a degenerate extreme point, but no constraint is redundant.
\end{example}

\begin{example}[Degeneracy $\nRightarrow$ Redundancy]
Let $P$ be a square pyramid (c.f.~\cref{ex:pyramid-1}).
Then there is a degenerate extreme point but no constraint is redundant.
\end{example}

\section{Simplex Method}

\begin{example}
Let $b \in \mathbb{R}_{\ge 0}$ and $a \in \mathbb{R}^n$, where $0 < a_1 < a_2 < \ldots < a_n$.
Consider the LP \[ \max_{x \ge 0}\quad a^Tx \quad\textrm{where}\quad \sum_{i=1}^n x_i \le b. \]
Clearly, the optimal solution is $[0, 0, \ldots, 0, b]$.
In standard form, the LP becomes
\[ \min_{x \ge 0}\quad \sum_{i=1}^n (-a_i)x_i \quad\textrm{where}\quad \sum_{i=1}^{n+1} x_i = b. \]
Suppose our initial basis is $\{x_{n+1}\}$. For ease of notation, let $x_0 \defeq x_{n+1}$.

If we run the simplex method with Bland's rule (variable of lowest index enters basis),
then there will be $n$ iterations, where in the $i\Th$ iteration,
$x_i$ enters the basis and $x_{i-1}$ leaves the basis.
Hence, we visit each of the $n+1$ bases.
If $b > 0$, each basis corresponds to a unique BFS.
If $b = 0$, all bases correspond to the same BFS $\mathbf{0}$.

If we run the simplex method with Dantzig's rule (variable of most negative reduced cost enters basis),
then there will be just 1 iteration where $x_n$ enters and $x_{n+1}$ leaves.
\end{example}

\begin{example}
For the following LP, where $b \in \mathbb{R}_{\ge 0}$, the optimal solution is $(b, 0)$.
\[ \max_{x \ge 0, y \ge 0}\quad 2x+3y \quad\textrm{where}\quad x + 2y \le b. \]
Bland's rule takes 1 iteration but Dantzig's rule takes 2 iterations.
There are 3 bases. When $b > 0$, each base corresponds to a different BFS.
When $b = 0$, all bases correspond to the same BFS $\mathbf{0}$.
\end{example}

\begin{example}[Unique basis and degeneracy]
Let $P \defeq \{(x, y, z): x + y = 2b, x - y = 0, x \ge 0, y \ge 0, z \ge 0\}$,
be a standard form polyhedron. Then $\{x, y\}$ is the only basis,
and the corresponding solution is $(b,b,0)$.
The solution is degenerate iff $b = 0$.
\end{example}

%\addMyBib{}

\end{document}

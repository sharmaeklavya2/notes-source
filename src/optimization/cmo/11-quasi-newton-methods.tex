\documentclass[a4paper, 12pt, fleqn]{article}

\usepackage{amsmath}
\usepackage{amsthm}
\usepackage{amssymb}
\usepackage[margin=1in]{geometry}
\usepackage{xcolor}
\usepackage{float}
\usepackage{url}
\usepackage{array}
\usepackage{comment}
\usepackage[bookmarksnumbered=true,hypertexnames=false]{hyperref}
\usepackage{algorithm, algpseudocode}
\usepackage[capitalize,sort]{cleveref}

\hypersetup{
    colorlinks,
    linkcolor={red!50!black},
    citecolor={red!50!black},
    urlcolor={blue!50!black}
}

% THEOREMS

\newtheorem{theorem}{Theorem}
\newtheorem{definition}{Definition}
\newtheorem{example}{Example}
\newtheorem{corollary}{Corollary}[theorem]
\newtheorem{lemma}[theorem]{Lemma}

% SHORTHANDS

\newcommand*{\Th}{^{\textrm{th}}}
\newcommand*{\WLoG}{Without loss of generality}
\newcommand*{\wLoG}{without loss of generality}
\newcommand*{\wrt}{with respect to}

% SYMBOLS

\let\eps\epsilon
\newcommand*{\defeq}{:=}

% MATH

\newcommand*{\floor}[1]{\left\lfloor #1 \right\rfloor}
\newcommand*{\smallfloor}[1]{\lfloor #1 \rfloor}
\newcommand*{\ceil}[1]{\left\lceil #1 \right\rceil}
\newcommand*{\smallceil}[1]{\lceil #1 \rceil}
\newcommand*{\abs}[1]{\left\lvert #1 \right\rvert}
\newcommand*{\smallabs}[1]{\lvert #1 \rvert}
\newcommand*{\norm}[1]{\left\lVert #1 \right\rVert}
\newcommand*{\smallnorm}[1]{\lVert #1 \rVert}
\newcommand*{\Z}{\mathbb{Z}}
\DeclareMathOperator*{\E}{E}
\DeclareMathOperator*{\Var}{Var}
\DeclareMathOperator*{\argmin}{argmin}
\DeclareMathOperator*{\argmax}{argmax}
\DeclareMathOperator{\poly}{poly}
\DeclareMathOperator{\opt}{opt}
\newcommand*{\OPT}{\mathrm{OPT}}

% INITIALIZATIONS

\newcommand{\initMinimal}{
\setlength{\parindent}{0pt}
\setlength{\parskip}{0.5em}
}
\newcommand{\initFromContents}{
\tableofcontents
\newpage
\initMinimal{}
}
\newcommand{\initAfterBeginDocument}{
\maketitle
\initFromContents{}
}
\newcommand{\addMyBib}{
\bibliographystyle{plainurl}
\bibliography{bibdb}
}

\author{Eklavya Sharma}
\date{\empty}

\input{src/optimization/cmo/common.texlib}

\title{CMO: Quasi-Newton Methods}

\begin{document}

\maketitle
\initMinimal{}

\tableofcontents

\section{Quasi-Newton method template}

Newton's method's update rule:
\[ x_{k+1} = x_k - \hessian_f^{-1}(x_k)\grad_f(x_k) \]
This method is not useful, because it requires inverting the hessian,
which can be prohibitively computationally expensive for high-dimensional data.

We will therefore try to model the change in the hessian's inverse,
and approximate the hessian's inverse instead of calculating it exactly.

Let $g_k = \grad_f(x_k)$, $\delta_k = x_{k+1} - x_k$ and $\gamma_k = g_{k+1} - g_k$.
\[ \grad_f(x_{k+1}) \approx \grad_f(x_k) + \hessian_f(x_k)(x_{k+1} - x_k)
\tag{by differentiating Taylor series} \]
\[ \implies \delta_k \approx \hessian_f^{-1}(x_k)\gamma_k \]

This inspires us to use an update rule of this form:
\[ x_{k+1} = x_k - A_kg_k \]
and apply the following constraint on $A_k$:
\begin{equation} \label{eqn:qn-cond} \delta_k = A_{k+1}\gamma_k \end{equation}
This constraint is called the `Quasi-Newton condition'.

Also, we must ensure that $A_k$ is symmetric and positive (semi)definite.

Note that the Quasi-Newton condition is $d$ equations,
whereas there are $d^2$ entries in $A_k$.
We therefore have a lot of slack in terms of how to update $A_k$.

In all Quasi-Newton methods described next, we choose $A_0$
as any matrix which is symmetric and positive (semi)definite.
Generally, the identity matrix is used.
Then we use $A_k$, $\delta_k$ and $\gamma_k$ to obtain $A_{k+1}$ via an update rule,
like `rank-1 update', `rank-2 update' or `BFGS'.

\section{Rank-1 update}

Here we impose a condition of the form $A_{k+1} = A_k + cuu^T$,
where $c \in \mathbb{R}$ and $u \in \mathbb{R}^d$
(Note that $\operatorname{rank}(uu^T) = 1$).

It's easy to see that $A_{k+1}$ is symmetric for all $c$
and positive definite for $c \ge 0$.

To get concrete values of $c$ and $u$,
we'll plug the rank-1 update condition into the Quasi-Newton condition (\ref{eqn:qn-cond}).
\[ \delta_k = (A_k + cuu^T)\gamma_k \implies (cu^T\gamma_k)u = \delta_k - A_k\gamma_k \]
Therefore, $u$ is parallel to $\delta_k - A_k\gamma_k$. Let $u = \delta_k - A_k\gamma_k$.
Then
\[ u = (cu^T\gamma_k)u \implies cu^T\gamma_k = 1
\implies c = \frac{1}{u^T\gamma_k} = \frac{1}{\delta_k^T\gamma_k - \gamma_k^TA_k\gamma_k} \]

With these specific values of $u$ and $c$,
the rank-1 update condition will satisfy all required conditions
(symmetry, positive definiteness and Quasi-Newton condition) if $c \ge 0$.

Unfortunately, it has not yet been proved or disproved whether $c \ge 0$.

\subsection{Analysis for quadratic function}

Let $f(x) = \frac{1}{2}x^TQx - b^Tx$, where $Q$ is symmetric and positive definite.
Then $\grad_f(x) = Qx - b \implies \gamma_k = Q\delta_k$.

\begin{lemma} \label{thm:rank-1:ind}
\[ \forall i \in [0, k], A_{k+1}\gamma_i = \delta_i \]
\end{lemma}
\begin{proof}[Proof by induction on $k$]
\[ P(l): \forall i \in [0, l-1], A_l\gamma_i = \delta_i \]
We have to prove $P(l)$ for all $l \ge 1$.

\textbf{Base case}:
Since $A_1$ was constructed to follow the Quasi-Newton condition,
$\delta_0 = A_1\gamma_0 \implies P(1)$.

\textbf{Inductive step}:
Assume $P(l)$ is true. We'll prove $P(l+1)$.

Let $i \in [0, l-1]$.
\begin{align*}
A_{l+1}\gamma_i &= \left( A_l + \frac{uu^T}{u^T\gamma_l} \right)\gamma_i
\tag{here $u = \delta_l - A_l\gamma_l$}
\\ &= \delta_i + \frac{u^T\gamma_i}{u^T\gamma_l}u
\tag{$A_l\gamma_i = \delta_i$ by induction hypothesis}
\end{align*}
\begin{align*}
u^T\gamma_i &= (\delta_l - A_l\gamma_l)^T\gamma_i
\\ &= \delta_l^T\gamma_i - \gamma_l^TA_l\gamma_i
\\ &= \delta_l^T\gamma_i - \gamma_l^T\delta_i \tag{by induction hypothesis}
\\ &= \delta_l^TQ\delta_i - \delta_l^TQ\delta_i \tag{$\forall j, \gamma_j = Q\delta_j$}
\\ &= 0
\end{align*}
Therefore, $A_{l+1}\gamma_i = \delta_i$ for all $i \in [0, l-1]$.
Since $A_{l+1}$ was constructed to follow the Quasi-Newton condition,
$A_{l+1}\gamma_l = \delta_l$. Therefore, $P(l+1)$ holds true.
\end{proof}

\begin{lemma}
If all $\delta_i$ were orthonormal, then $A_d = Q^{-1}$.
\end{lemma}
\begin{proof}
By lemma \ref{thm:rank-1:ind},
\[ \forall i \in [0, d-1], \delta_i = A_d\gamma_i = A_dQ\delta_i \]
Therefore, $(1, \delta_i)$ is an eigenpair for $A_dQ$.

Let $P$ be the matrix whose $i^{\textrm{th}}$ columns is $\delta_i$.
$P$ exists because real symmetric matrices are orthogonally diagonalizable
and $A_dQ$ is real and symmetric.
Then $A_dQ = PIP^T = I \implies A_d = Q^{-1}$.
\end{proof}

\begin{lemma} \label{thm:delta-linindep-convergence}
If all $\delta_i$ are linearly independent, then $A_d = Q^{-1}$.
\end{lemma}
\begin{proof}
Let $\Delta = \{\delta_0, \ldots, \delta_{d-1}\}$.
Since $\Delta \subseteq \mathbb{R}^d$,
$|\Delta| = d = \dim(\mathbb{R}^d)$
and $\Delta$ is linearly independent, $\Delta$ is a basis of $\mathbb{R}^d$.

Let $x \in \mathbb{R}^d$. Let $x = \sum_{i=0}^{d-1} c_i\delta_i$. Then
\[ A_dQx
= \sum_{i=0}^{d-1} A_dQ(c_i\delta_i)
= \sum_{i=0}^{d-1} c_i(A_d\gamma_i)
= \sum_{i=0}^{d-1} c_i\delta_i
= x \]
Therefore, $\forall x \in \mathbb{R}^d, (A_dQ)x = x$, so $A_dQ = I$.

Note that the proof is not specific to rank-1 updates.
Its correctness relies only on the Quasi-Newton condition and $f$ being quadratic.
\end{proof}

Since $A_d = Q^{-1}$, the $(d+1)^{\textrm{th}}$ iteration
would be identical to Newton's method.
So the rank-1 update method will converge to the minimum in at most $d+1$ iterations.

\subsection{Unresolved questions}

\begin{itemize}
\item $A_k$ is positive definite when $c \ge 0$. Is $c \ge 0$?
\item Is $\{\delta_0, \delta_1, \ldots\}$ linearly independent?
\end{itemize}

\section{Rank-2 update}

\[ A_{k+1} = A_k + cuu^T + bvv^T \]
It's easy to see that $A_{k+1}$ is symmetric iff $A_k$ is symmetric.

By Quasi-Newton condition, we get
\[ \delta_k = A_{k+1}\gamma_k
\implies (cu^T\gamma_k)u + (bv^T\gamma_k)v = \delta_k - A_k\gamma_k \]
Let $u = \delta_k$ and $v = A_k\gamma_k$. Then
\begin{align*}
c &= \frac{1}{u^T\gamma_k} = \frac{1}{\delta_k^T\gamma_k}
& b &= \frac{-1}{v^T\gamma_k} = \frac{-1}{\gamma_k^TA_k\gamma_k}
\end{align*}
\[ A_{k+1} = A_k + \frac{\delta_k\delta_k^T}{\delta_k^T\gamma_k}
- \frac{A_k\gamma_k\gamma_k^TA_k}{\gamma_k^TA_k\gamma_k} \]

\subsection{Analysis for quadratic function}

Let $f(x) = \frac{1}{2}x^TQx - b^Tx$. Then $\gamma_k = Q\delta_k$.

\begin{lemma}[Symmetric square root of a matrix]
If $A$ is a symmetric and positive definite matrix,
then $\exists L$ such that $A = L^2$ and $L$ is symmetric, positive semidefinite and invertible.
\end{lemma}
\begin{proof}
Since $A$ is real and symmetric, it is orthogonally diagonalizable.
So there is a matrix $P$ and a diagonal matrix $D$ such that $A = PDP^T$ and $PP^T = P^TP = I$.
Since $A$ is positive definite, all diagonal entries of $D$ are positive.
Therefore, $\sqrt{D}$ exists.
Also, all entries of $\sqrt{D}$ are positive, so $\sqrt{D}^{-1}$ exists.
Let $L = P\sqrt{D}P^T$. Then $L$ is symmetric and $L^2 = A$.
\[ u^TL^u = u^T(P\sqrt{D}P^T)u = (P^Tu)^T\sqrt{D}(P^Tu) \ge 0 \]
Therefore, $L$ is also positive semidefinite. Also,
\[ L(P\sqrt{D}^{-1}P^T) = P\sqrt{D}P^TP\sqrt{D}^{-1}P^T = I \]
Therefore, $L^{-1} = P\sqrt{D}^{-1}P^T$.
\end{proof}

\begin{theorem} \label{thm:rank-2:pd}
Let $A_k$ be symmetric and positive definite.
Then $A_{k+1}$ is positive definite.
\end{theorem}
\begin{proof}
\begin{equation} \label{eqn:c-pos}
c = \frac{1}{\delta_k^T\gamma_k} = \frac{1}{\delta_k^TQ\delta_k} > 0
\end{equation}
We'll now prove that $A_{k+1} - cuu^T$ is positive semidefinite.
Let $w \in \mathbb{R}^d - \{0\}$.
\begin{align*}
& w^T(A_{k+1} - cuu^T)w
\\ &= w^T(A_k + bvv^T)w
\\ &= w^TA_kw - \frac{(w^TA_k\gamma_k)^2}{\gamma_k^TA_k\gamma_k}
\end{align*}
Since $A_k$ is symmetric and positive definite, it has a symmetric and invertible square root $L$.
\begin{align*}
& w^T(A_{k+1} - cuu^T)w
\\ &= w^TL^TLw - \frac{(w^TL^TL\gamma_k)^2}{\gamma_k^TL^TL\gamma_k}
\\ &= \|Lw\|^2 - \frac{((Lw)^T(L\gamma_k))^2}{\|L\gamma_k\|^2}
\\ &\ge 0 \tag{by Cauchy-Schwarz inequality}
\end{align*}
Therefore, $A_{k+1} - cuu^T$ is positive semidefinite.
Since $cuu^T$ is also positive semidefinite, $A_{k+1}$ is also positive semidefinite.

The Cauchy-Schwarz inequality is tight iff the vectors are parallel or anti-parallel.
Therefore, $A_{k+1} - cuu^T = 0 \iff Lw = \alpha L\gamma_k$ for some $\alpha \in \mathbb{R}$.
Since $L$ is invertible, this is equivalent to $w = \alpha \gamma_k$.

Assume $A_{k+1}$ is not positive definite.
$\exists w \in \mathbb{R}^d - \{0\}, w^TA_{k+1}w = 0$.
\begin{align*}
& w^TA_{k+1}w = 0
\\ &\implies w^T(A_{k+1} - cuu^T)w + w^T(cuu^T)w = 0
\\ &\implies w^T(A_{k+1} - cuu^T)w = 0 \wedge w^T(cuu^T)w = 0
\\ &\implies (\alpha \gamma_k)^T(cuu^T)(\alpha \gamma_k) = 0
\\ &\implies c\alpha^2 (\gamma_k^T\delta_k)^2 = 0 \tag{$u = \delta_k$}
\\ &\implies \alpha^2 (\delta_k^TQ\delta_k) = 0 \tag{$\gamma_k = Q\delta_k$ and \ref{eqn:c-pos}}
\end{align*}
This is not possible because
$\delta_k^TQ\delta_k > 0$ (because $Q$ is positive definite)
and $\alpha \neq 0$ (because $w \neq 0$).
Therefore, we have a contradiction.
Therefore, $A_{k+1}$ is positive definite.
\end{proof}

\begin{lemma}[Proof omitted (probably beyond scope of course)] \label{thm:rank-2:ind}
\[ \forall k \ge 1, \forall i \in [0, k-1], A_k\gamma_i = \delta_i \wedge \delta_k^TQ\delta_i = 0 \]
\end{lemma}

Let $\Delta = \{\delta_0, \delta_1, \ldots\}$.
Lemma \ref{thm:rank-2:ind} states that $\Delta$ is $Q$-conjugate.
This implies that $\Delta$ is linearly independent.
By lemma \ref{thm:delta-linindep-convergence},
we get that rank-2 updates converge to minimum in $d+1$ iterations.

\section{BFGS}

Instead of modeling the change in hessian's inverse,
we'll now model the change in the hessian.
But we need to do it in a way such that the change in the inverse is also easy to compute.

Let $B_k$ be an approximation to the hessian and $A_k$ be an approximation to the inverse of the hessian.
Then $\gamma_k = B_{k+1}\delta_k$ and $\delta_k = A_{k+1}\gamma_k$.

We'll chose the update rule as
\[ B_{k+1} = B_k + cuu^T + bvv^T \]
This will make sure that $B_k$ is symmetric implies $B_{k+1}$ is symmetric.

Applying the Quasi-Newton condition, we get
\[ \gamma_k = B_{k+1}\delta_k
\implies \gamma_k - B_k\delta_k = (cu^T\delta_k)u + (bv^T\delta_k)v \]
Let $u = \gamma_k$ and $v = B_k\delta_k$.
\begin{align*} c &= \frac{1}{u^T\delta_k} = \frac{1}{\gamma_k^T\delta_k}
& d &= \frac{-1}{v^T\delta_k} = \frac{-1}{\delta_k^TB_k\delta_k} \end{align*}
\[ B_{k+1} = B_k + \frac{\gamma_k^T\gamma_k}{\gamma_k^T\delta_k}
- \frac{B_k\delta_k\delta_k^TB_k}{\delta_k^TB_k\delta_k} \]

Similar to theorem \ref{thm:rank-2:pd}, we can prove that
$B_{k+1}$ is positive definite for quadratic functions.
This implies that $A_{k+1}$ is also symmetric and positive definite
for quadratic functions.

To invert $B_{k+1}$, we'll use the Sherman-Morrison formula.

\begin{theorem}[Sherman-Morrison formula]
Let $A$ be an invertible matrix. Then $A + uv^T$ is invertible iff $1 + v^TA^{-1}u \neq 0$. Also,
\[ (A + uv^T)^{-1} = A^{-1} - \frac{A^{-1}uv^TA^{-1}}{1 + v^TA^{-1}u} \]
\end{theorem}

Applying the formula twice, we get
\[ A_{k+1} = A_k + \frac{\delta_k\delta_k^T}{\delta_k^T\gamma_k}
\left( 1 + \frac{\gamma_k^TA_k\gamma_k}{\delta_k^T\gamma_k} \right)
- \frac{A_k\gamma_k\delta_k^T +  \delta_k\gamma_k^TA_k}{\delta_k^T\gamma_k} \]

\section{Broyden Family}

Let's explore this update rule:
\[ A_{k+1} = A_k + a\frac{\delta_k\delta_k^T}{\delta^T\gamma_k}
+ c\frac{A_k\gamma_k\gamma_k^TA_k}{\gamma_k^TA\gamma_k}
- b\frac{A_k\gamma_k\delta_k^T + \delta_k\gamma_k^TA_k}{\delta_k^T\gamma_k} \]
Applying the Quasi-Newton condition, we get
\[ \delta_k - A_k\gamma_k
= \left(a - b\frac{\gamma_k^TA_k\gamma_k}{\delta_k^T\gamma_k}\right)\delta_k + (c - b)A_k\gamma_k \]
Equating coefficients of $\delta_k$ and $\gamma_k$, we get
\begin{align*} a &= 1 + b\frac{\gamma_k^TA_k\gamma_k}{\delta_k^T\gamma_k} & c &= b-1 \end{align*}
On rearranging, we get
\[ A_{k+1} = \left( A_k + \frac{\delta_k\delta_k^T}{\delta_k^T\gamma_k}
- \frac{A_k\gamma_k\gamma_k^TA_k}{\gamma_k^TA_k\gamma_k} \right)
+ b(\gamma_k^TA_k\gamma_k)w_kw_k^T\]
where
\[ w = \frac{\delta_k}{\delta_k^T\gamma_k} - \frac{A_k\gamma_k}{\gamma_k^TA_k\gamma_k} \]
This update rule is called the Broyden Family.
Note that the first term is the same as the rank-2 update.

Define the following 2 functions:
\[ \operatorname{rank-2}(A, \delta, \gamma) = A + \frac{\delta\delta^T}{\delta^T\gamma}
- \frac{A\gamma\gamma^TA}{\gamma^TA\gamma} \]
\[ \operatorname{bfgs}(A, \delta, \gamma) = A + \frac{\delta\delta^T}{\delta^T\gamma}
\left( 1 + \frac{\gamma^TA\gamma}{\delta^T\gamma} \right)
- \frac{A\gamma\delta^T + \delta\gamma^TA}{\delta^T\gamma} \]
The Broyden family can also be rewritten as
\[ A_{k+1} = (1-b)\operatorname{rank-2}(A_k, \delta_k, \gamma_k) + b\operatorname{bfgs}(A_k, \delta_k, \gamma_k) \]

\end{document}

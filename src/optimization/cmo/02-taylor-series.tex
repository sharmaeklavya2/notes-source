\documentclass[a4paper, 12pt, fleqn]{article}

\usepackage{amsmath}
\usepackage{amsthm}
\usepackage{amssymb}
\usepackage[margin=1in]{geometry}
\usepackage{xcolor}
\usepackage{float}
\usepackage{url}
\usepackage{array}
\usepackage{comment}
\usepackage[bookmarksnumbered=true,hypertexnames=false]{hyperref}
\usepackage{algorithm, algpseudocode}
\usepackage[capitalize,sort]{cleveref}

\hypersetup{
    colorlinks,
    linkcolor={red!50!black},
    citecolor={red!50!black},
    urlcolor={blue!50!black}
}

% THEOREMS

\newtheorem{theorem}{Theorem}
\newtheorem{definition}{Definition}
\newtheorem{example}{Example}
\newtheorem{corollary}{Corollary}[theorem]
\newtheorem{lemma}[theorem]{Lemma}

% SHORTHANDS

\newcommand*{\Th}{^{\textrm{th}}}
\newcommand*{\WLoG}{Without loss of generality}
\newcommand*{\wLoG}{without loss of generality}
\newcommand*{\wrt}{with respect to}

% SYMBOLS

\let\eps\epsilon
\newcommand*{\defeq}{:=}

% MATH

\newcommand*{\floor}[1]{\left\lfloor #1 \right\rfloor}
\newcommand*{\smallfloor}[1]{\lfloor #1 \rfloor}
\newcommand*{\ceil}[1]{\left\lceil #1 \right\rceil}
\newcommand*{\smallceil}[1]{\lceil #1 \rceil}
\newcommand*{\abs}[1]{\left\lvert #1 \right\rvert}
\newcommand*{\smallabs}[1]{\lvert #1 \rvert}
\newcommand*{\norm}[1]{\left\lVert #1 \right\rVert}
\newcommand*{\smallnorm}[1]{\lVert #1 \rVert}
\newcommand*{\Z}{\mathbb{Z}}
\DeclareMathOperator*{\E}{E}
\DeclareMathOperator*{\Var}{Var}
\DeclareMathOperator*{\argmin}{argmin}
\DeclareMathOperator*{\argmax}{argmax}
\DeclareMathOperator{\poly}{poly}
\DeclareMathOperator{\opt}{opt}
\newcommand*{\OPT}{\mathrm{OPT}}

% INITIALIZATIONS

\newcommand{\initMinimal}{
\setlength{\parindent}{0pt}
\setlength{\parskip}{0.5em}
}
\newcommand{\initFromContents}{
\tableofcontents
\newpage
\initMinimal{}
}
\newcommand{\initAfterBeginDocument}{
\maketitle
\initFromContents{}
}
\newcommand{\addMyBib}{
\bibliographystyle{plainurl}
\bibliography{bibdb}
}

\author{Eklavya Sharma}
\date{\empty}

\input{src/optimization/cmo/common.texlib}

\title{CMO 2: Taylor Series}

\begin{document}

\maketitle
\initMinimal{}
\tableofcontents

\section{Univariate Taylor Series}

Let $f: [a,b] \mapsto \mathbb{R}$.
Let $x, y \in [a, b]$.

Suppose $f$ is differentiable $k$ times. Then for some $z \in (x, y)$,
\[ f(y) = \sum_{i=0}^{k-1}f^{(i)}(x)\frac{(y-x)^i}{i!} + f^{(k)}(z)\frac{(y-x)^k}{k!} \]

$C^k$ is the set of all functions which are $k$-times differentiable
and whose $k^{\textrm{th}}$ derivative is continuous.

When $f^{(k)} \in C^k$,
\[ f(y) = \sum_{i=0}^k f^{(i)}(x)\frac{(y-x)^i}{i!} + o(1)\frac{(y-x)^k}{k!} \]

Therefore, we can ignore the last term if $x$ is close to $y$.

\section{Multivariate Calculus}

\begin{definition} Let $f: \mathbb{R}^m \mapsto \mathbb{R}^n$ be a function and $y = f(x)$.
Then the Jacobian of $y$ w.r.t. $x$ is an $n$ by $m$ matrix where
\[ \left(\frac{\partial y}{\partial x}\right)_{i, j} = \frac{\partial y_i}{\partial x_j} \]
\end{definition}

\begin{theorem}[Chain rule]
\label{thm:chain-rule}
Let $y = f(x)$ and $z = g(y)$. Then
\[ \frac{\partial z}{\partial x}
= \left( \frac{\partial z}{\partial y} \right) \left( \frac{\partial y}{\partial x} \right) \]
\end{theorem}

\begin{definition} For $f: \mathbb{R}^d \mapsto \mathbb{R}$,
the gradient of $f$, denoted as $\grad_f$, is a $d$-dimensional vector defined as
\[ \grad_f(x) = \left[ \frac{\partial f(x)}{\partial x_i} \right]_{i=1}^d \]
\end{definition}

For multivariate functions, $f \in C^1$ iff $\grad_f$ exists and all components are continuous.
Note that \href{https://calculus.subwiki.org/wiki/Derivative_of_differentiable_function_need_not_be_continuous}
{differentiability does not imply $C_1$}.

\begin{definition} For $f: \mathbb{R}^d \mapsto \mathbb{R}$, the hessian of $f$,
denoted as $\hessian_f$, is a $d$ by $d$ matrix defined as
\[ \hessian_f(x)_{i, j} = \frac{\partial^2 f(x)}{\partial x_i \partial x_j} \]
\end{definition}

For multivariate function, $f \in C^2$ iff $\hessian_f$ exists and all its entries are continuous.

\begin{theorem}[Proof omitted]
When $f \in C^2$, $\hessian_f$ is symmetric.
\end{theorem}

\section{Multivariate Taylor Series}

Let $g(t) = f(x + tu)$, where $t \in \mathbb{R}$ and $x, u \in \mathbb{R}^d$.

\begin{theorem}
\[ g'(t) = \grad_f(x+tu)^Tu \tag{when $f \in C^1$, by chain rule} \]
\[ g''(t) = u^T \hessian_f(x+tu) u \tag{when $f \in C^2$} \]
\end{theorem}

\begin{theorem} $f \in C^1 \implies g \in C^1$ \end{theorem}

\begin{theorem} If $f \in C^1$ and $y$ is close to $x$,
\[ f(y) = f(x) + \grad_f(x)^T(y-x) + o(\|y-x\|) \]
\end{theorem}
\begin{proof}
Let $g(t) = f(x+tu)$ and let $u = y - x$ be small.
By applying univariate Taylor series on $g$ at 0, we get
\begin{align*}
& g(1) = g(0) + g'(\alpha) \textrm{, where } \alpha \in [0, 1]
\\ &\Rightarrow f(x+u) = f(x) + \grad_f(x + \alpha u)^Tu
\\ &\Rightarrow f(x+u) = f(x) + (\grad_f(x) + o(1))^Tu  \tag{$\grad_f$ is continuous and $u$ is small}
\\ &\Rightarrow f(y) = f(x) + \grad_f(x)^T(y-x) + o(\|y-x\|)
\end{align*}
\end{proof}

\begin{theorem} If $f \in C^2$ and $y$ is close to $x$,
\[ f(y) = f(x) + \grad_f(x)^T(y-x) + \frac{1}{2}(y-x)^T\hessian_f(x)(y-x) + o(\|y-x\|^2) \]
\end{theorem}
\begin{proof}Similar to previous theorem.\end{proof}

\end{document}

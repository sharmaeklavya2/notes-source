\documentclass[a4paper,12pt,fleqn]{article}

\usepackage{amsmath,amsthm,amssymb}
\usepackage[margin=1in]{geometry}
\usepackage{xcolor}
\usepackage{array}
\usepackage{comment}
\usepackage{enumitem}
%\usepackage{booktabs}
\usepackage[bookmarksnumbered=true,hypertexnames=false]{hyperref}
%\usepackage{algorithm, algpseudocode}
\usepackage[capitalize,sort]{cleveref}

\def\colorschemesepia{sepia}
\def\colorschemedark{dark}
\def\colorschemelight{light}

\ifx\colorscheme\undefined
\let\colorscheme\colorschemelight
\fi

\ifx\colorscheme\colorschemelight
\colorlet{textColor}{black}
\colorlet{bgColor}{white}
\fi

\ifx\colorscheme\colorschemesepia
\definecolor{textColor}{HTML}{433423}
\definecolor{bgColor}{HTML}{fbf0da}
\fi

\ifx\colorscheme\colorschemedark
\definecolor{textColor}{HTML}{bdc1c6}
\definecolor{bgColor}{HTML}{202124}
\definecolor{textBlue}{HTML}{8ab4f8}
\definecolor{textRed}{HTML}{f9968b}
\colorlet{bgRed}{red!40!black}
\colorlet{bgBlue}{red!40!black}
\else
\colorlet{textBlue}{blue!50!black}
\colorlet{textRed}{red!50!black}
\fi

\ifx\colorscheme\colorschemelight\else
\pagecolor{bgColor}
\color{textColor}
\fi

\hypersetup{colorlinks,linkcolor=textRed,citecolor=textRed,urlcolor=textBlue}
% THEOREMS

\newtheorem{theorem}{Theorem}
\newtheorem{definition}{Definition}
\newtheorem{example}{Example}
\newtheorem{corollary}{Corollary}[theorem]
\newtheorem{lemma}[theorem]{Lemma}
\newtheorem{claim}[theorem]{Claim}

% SHORTHANDS

\newcommand*{\Th}{^{\textrm{th}}}
\newcommand*{\WLoG}{Without loss of generality}
\newcommand*{\wLoG}{without loss of generality}
\newcommand*{\wrt}{with respect to}
\let\opname\operatorname

% SYMBOLS

\let\eps\epsilon
\newcommand*{\defeq}{:=}

\newcommand*{\alphahat}{\widehat{\alpha}}
\newcommand*{\betahat}{\widehat{\beta}}
\newcommand*{\Ahat}{\widehat{A}}
\newcommand*{\Bhat}{\widehat{B}}
\newcommand*{\Chat}{\widehat{C}}
\newcommand*{\Jhat}{\widehat{J}}
\newcommand*{\Shat}{\widehat{S}}
\newcommand*{\Yhat}{\widehat{Y}}
\newcommand*{\bhat}{\widehat{b}}
\newcommand*{\shat}{\widehat{s}}
\newcommand*{\what}{\widehat{w}}
\newcommand*{\xhat}{\widehat{x}}
\newcommand*{\yhat}{\widehat{y}}
\newcommand*{\zhat}{\widehat{z}}

\newcommand*{\Btild}{\widetilde{B}}
\newcommand*{\Ctild}{\widetilde{C}}
\newcommand*{\Jtild}{\widetilde{J}}
\newcommand*{\Stild}{\widetilde{S}}
\newcommand*{\Ytild}{\widetilde{Y}}
\newcommand*{\btild}{\widetilde{b}}
\newcommand*{\stild}{\widetilde{s}}
\newcommand*{\wtild}{\widetilde{w}}
\newcommand*{\xtild}{\widetilde{x}}
\newcommand*{\ytild}{\widetilde{y}}
\newcommand*{\ztild}{\widetilde{z}}

\newcommand*{\Fcal}{\mathcal{F}}
\newcommand*{\Tcal}{\mathcal{T}}

\newcommand*{\ebar}{\overline{e}}
\newcommand*{\Xbar}{\overline{X}}
\newcommand*{\xbar}{\overline{x}}
\newcommand*{\Ybar}{\overline{Y}}
\newcommand*{\ybar}{\overline{y}}

% MATH

\newcommand*{\floor}[1]{\left\lfloor #1 \right\rfloor}
\newcommand*{\smallfloor}[1]{\lfloor #1 \rfloor}
\newcommand*{\ceil}[1]{\left\lceil #1 \right\rceil}
\newcommand*{\smallceil}[1]{\lceil #1 \rceil}
\newcommand*{\abs}[1]{\left\lvert #1 \right\rvert}
\newcommand*{\smallabs}[1]{\lvert #1 \rvert}
\newcommand*{\norm}[1]{\left\lVert #1 \right\rVert}
\newcommand*{\smallnorm}[1]{\lVert #1 \rVert}
\newcommand*{\Z}{\mathbb{Z}}
\DeclareMathOperator*{\E}{E}
\DeclareMathOperator*{\Var}{Var}
\DeclareMathOperator*{\argmin}{argmin}
\DeclareMathOperator*{\argmax}{argmax}
\DeclareMathOperator{\poly}{poly}
\DeclareMathOperator{\opt}{opt}
\DeclareMathOperator{\argopt}{argopt}
\DeclareMathOperator{\support}{support}
\newcommand*{\OPT}{\mathrm{OPT}}


\newenvironment*{tightenum}{\begin{enumerate}[noitemsep]}{\end{enumerate}}
\DeclareMathOperator{\LP}{LP}

\author{Eklavya Sharma}
\date{\empty}

\title{Linear Programming Duality}

\begin{document}

\maketitle
\setlength{\parskip}{0.2em}

\section{Definition}

Let $A \in \mathbb{R}^{m \times n}$, $I \subseteq [m]$, $J \subseteq [n]$.
Define the linear program $P \defeq \LP(A, b, c, I, J)$ as
\[ \min_{x \in \mathbb{R}^n: x_J \ge 0} c^Tx \quad\textrm{where}\quad
((Ax)_i \ge b_i, \forall i \in I) \textrm{ and } ((Ax)_i = b_i, \forall i \in [m]-I) \]
Then the dual of $P$ is $D \defeq \LP(-A^T, -c, -b, J, I)$, i.e.,
\[ \max_{y \in \mathbb{R}^m: y_I \ge 0} b^Ty \quad\textrm{where}\quad
((A^Ty)_j \le c_j, \forall j \in J) \textrm{ and } ((A^Ty)_j = c_j, \forall j \in [n]-J) \]
$P$ is called the \emph{primal} LP.

\begin{lemma}
Dual of dual is primal.
\end{lemma}

On setting $J = [n]$ and $I = [m]$, we get that the
following are duals of each other
\begin{align*}
P &: \min_{x \ge 0} c^Tx \textrm{ where } Ax \ge b
& D&: \max_{y \ge 0} b^Ty \textrm{ where } A^Ty \le c
\end{align*}

\begin{theorem}[Weak duality]
Let $\LP(A, b, c, I, J)$ and $\LP(-A^T, -c, -b, J, I)$ be the primal and dual LPs, respectively.
Then for any feasible solution $x$ to the primal and any feasible solution $y$ to the dual,
we have $b^Ty \le y^T\!Ax \le c^Tx$.
\end{theorem}

\begin{corollary}[Complementary slackness]
$\xhat$ and $\yhat$ are optimal solutions to $\LP(A, b, c, I, J)$ and $\LP(-A^T, -c, -b, J, I)$, respectively,
iff $\xhat^T(c - A^T\yhat) = 0$ and $\yhat^T(A\xhat - b) = 0$.
\end{corollary}

\begin{theorem}[Strong duality]
The primal LP has an optimal solution iff the dual LP has an optimal solution.
If $x^*$ is an optimal solution to the primal LP and $y^*$ is an optimal solution to the dual LP,
then $x^*$ and $y^*$ have the same objective value.
\end{theorem}

\section{Method for Easy Application}

Adapted from S\'ebastien Lahaie's \href{http://www.cs.columbia.edu/coms6998-3/lpprimer.pdf}{notes}.

\begin{enumerate}
\item Express the problem in standard form:
    \begin{tightenum}
    \item Express as a minimization problem.
    \item Write each (non-trivial) constraint as $f(x) \le 0$ or $f(x) = 0$.
    \item Write each trivial constraint as $x \ge 0$ (i.e., if $x \le 0$, replace $x$ by $-x$):
    \end{tightenum}
\[ \min_{x \in \mathbb{R}^n: x_J \ge 0} c^Tx \quad\textrm{where}\quad
(b_i - (Ax)_i \le 0, \forall i \in I) \textrm{ and } (b_i - (Ax)_i = 0, \forall i \in [m]-I). \]
\item Add dual variables:
    \begin{tightenum}
    \item Create a non-negative dual variable $y_i$ for each inequality constraint $f_i(x) \le 0$.
    \item Create an unrestricted dual variable $y_i$ for each equality constraint $f_i(x) = 0$.
    \item Remove the constraint and add the term $y_if_i(x)$ to the objective.
    \item Maximize over dual variables.
    \end{tightenum}
\[ \max_{y \in \mathbb{R}^m: y_I \ge 0}\; \min_{x \in \mathbb{R}^n: x_J \ge 0} c^Tx + y^T(b - Ax). \]
\item Rearrange terms to express objective as an affine function of primal variables:
\[ \max_{y \in \mathbb{R}^m: y_I \ge 0}\; \min_{x \in \mathbb{R}^n: x_J \ge 0} b^Ty + x^T(c - A^Ty). \]
\item For each term $x_jg_j(y)$ in the objective, remove the term and add the constraint
    \begin{tightenum}
    \item $g_j(y) \ge 0$ if $x_j$ is non-negative.
    \item $g_j(y) = 0$ if $x_j$ is unrestricted.
    \end{tightenum}
\[ \max_{y \in \mathbb{R}^m: y_J \ge 0} b^Ty \quad\textrm{where}\quad
(c_j - (A^Ty)_j \ge 0, \forall j \in J) \textrm{ and } (c_j - (A^Ty)_j = 0, \forall j \in [n]-J). \]
\item Rearrange into suitable form
\[ \max_{y \in \mathbb{R}^m: y_J \ge 0} b^Ty \quad\textrm{where}\quad
((A^Ty)_j \le c_j, \forall j \in J) \textrm{ and } ((A^Ty)_j = c_j, \forall j \in [n]-J). \]
\end{enumerate}

\end{document}

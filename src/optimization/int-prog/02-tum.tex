\documentclass[a4paper,12pt,fleqn]{article}

\usepackage{amsmath,amsthm,amssymb}
\usepackage[margin=1in]{geometry}
\usepackage{xcolor}
\usepackage{array}
\usepackage{comment}
\usepackage{enumitem}
\usepackage[bookmarksnumbered=true,hypertexnames=false]{hyperref}
\usepackage[capitalize,sort]{cleveref}

\def\colorschemesepia{sepia}
\def\colorschemedark{dark}
\def\colorschemelight{light}

\ifx\colorscheme\undefined
\let\colorscheme\colorschemelight
\fi

\ifx\colorscheme\colorschemelight
\colorlet{textColor}{black}
\colorlet{bgColor}{white}
\fi

\ifx\colorscheme\colorschemesepia
\definecolor{textColor}{HTML}{433423}
\definecolor{bgColor}{HTML}{fbf0da}
\fi

\ifx\colorscheme\colorschemedark
\definecolor{textColor}{HTML}{bdc1c6}
\definecolor{bgColor}{HTML}{202124}
\definecolor{textBlue}{HTML}{8ab4f8}
\definecolor{textRed}{HTML}{f9968b}
\colorlet{bgRed}{red!40!black}
\colorlet{bgBlue}{red!40!black}
\else
\colorlet{textBlue}{blue!50!black}
\colorlet{textRed}{red!50!black}
\fi

\ifx\colorscheme\colorschemelight\else
\pagecolor{bgColor}
\color{textColor}
\fi

\hypersetup{colorlinks,linkcolor=textRed,citecolor=textRed,urlcolor=textBlue}
% THEOREMS

\newtheorem{theorem}{Theorem}
\newtheorem{definition}{Definition}
\newtheorem{example}{Example}
\newtheorem{corollary}{Corollary}[theorem]
\newtheorem{lemma}[theorem]{Lemma}

% SHORTHANDS

\newcommand*{\Th}{^{\textrm{th}}}
\newcommand*{\WLoG}{Without loss of generality}
\newcommand*{\wLoG}{without loss of generality}
\newcommand*{\wrt}{with respect to}

% SYMBOLS

\let\eps\epsilon
\newcommand*{\defeq}{:=}

\newcommand*{\Ahat}{\widehat{A}}
\newcommand*{\Bhat}{\widehat{B}}
\newcommand*{\Chat}{\widehat{C}}
\newcommand*{\Jhat}{\widehat{J}}
\newcommand*{\Shat}{\widehat{S}}
\newcommand*{\Yhat}{\widehat{Y}}
\newcommand*{\bhat}{\widehat{b}}
\newcommand*{\what}{\widehat{w}}
\newcommand*{\xhat}{\widehat{x}}
\newcommand*{\yhat}{\widehat{y}}
\newcommand*{\zhat}{\widehat{z}}

\newcommand*{\Btild}{\widetilde{B}}
\newcommand*{\Ctild}{\widetilde{C}}
\newcommand*{\Jtild}{\widetilde{J}}
\newcommand*{\Stild}{\widetilde{S}}
\newcommand*{\Ytild}{\widetilde{Y}}
\newcommand*{\btild}{\widetilde{b}}
\newcommand*{\wtild}{\widetilde{w}}
\newcommand*{\xtild}{\widetilde{x}}
\newcommand*{\ytild}{\widetilde{y}}
\newcommand*{\ztild}{\widetilde{z}}

\newcommand*{\Fcal}{\mathcal{F}}
\newcommand*{\Tcal}{\mathcal{T}}

% MATH

\newcommand*{\floor}[1]{\left\lfloor #1 \right\rfloor}
\newcommand*{\smallfloor}[1]{\lfloor #1 \rfloor}
\newcommand*{\ceil}[1]{\left\lceil #1 \right\rceil}
\newcommand*{\smallceil}[1]{\lceil #1 \rceil}
\newcommand*{\abs}[1]{\left\lvert #1 \right\rvert}
\newcommand*{\smallabs}[1]{\lvert #1 \rvert}
\newcommand*{\norm}[1]{\left\lVert #1 \right\rVert}
\newcommand*{\smallnorm}[1]{\lVert #1 \rVert}
\newcommand*{\Z}{\mathbb{Z}}
\DeclareMathOperator*{\E}{E}
\DeclareMathOperator*{\Var}{Var}
\DeclareMathOperator*{\argmin}{argmin}
\DeclareMathOperator*{\argmax}{argmax}
\DeclareMathOperator{\poly}{poly}
\DeclareMathOperator{\opt}{opt}
\DeclareMathOperator{\support}{support}
\newcommand*{\OPT}{\mathrm{OPT}}


\author{Eklavya Sharma}
\date{\empty}

\newcommand*{\karthikLec}[1]{http://karthik.ise.illinois.edu/courses/ie511/lectures-sp-21/lecture-#1.pdf}
\newenvironment*{tightenum}{\begin{enumerate}[noitemsep]}{\end{enumerate}}

\title{Integer Programming: Total Unimodularity}

\begin{document}

\maketitle
\setlength{\parskip}{0.2em}

Based on lecture notes by Prof.~Karthik (\href{\karthikLec{10}}{Lecture 10}) and Prof.~Eteasmi.

\section{Definition and Motivation}

\begin{definition}[Integral matrix]
$A \in \mathbb{R}^{m \times n}$ is \emph{integral} iff
each entry in $A$ is an integer.
\end{definition}

\begin{definition}[Total Unimodularity]
A matrix $A \in \mathbb{R}^{m \times n}$ is \emph{totally unimodular} (TU) iff
for every square submatrix $B$ of $A$, we have $\det(B) \in \{-1, 0, 1\}$.
\end{definition}

\begin{lemma}[Integral inverse]
Let $A$ be TU. Then for every square submatrix $B$ of $A$, $B^{-1}$ is integral.
\end{lemma}
\begin{proof}[Proof sketch]
If $A$ is TU, then $B$ is also TU. Let $B \in \mathbb{R}^{n \times n}$. Then
\[ (B^{-1})[i, j] = \frac{(-1)^{i+j}\det(C_{i,j})}{\det(B)}
\quad\textrm{where}\quad C_{i,j} = B[[n]-\{j\}, [n]-\{i\}]. \]
$\det(B), \det(C_{i,j}) \in \{-1, 0, 1\}$ because $B$ is TU.
\end{proof}

\begin{theorem}[TU polyhedron]
Let $P \defeq \{x \in \mathbb{R}^n: (a_i^Tx = b_i, \forall i \in E) \wedge (a_i^Tx \ge b_i, \forall i \in I)\}$
be a non-empty polyhedron, where $b_i \in \mathbb{Z}$ for all $i \in I \cup E$.
Let $A$ be a matrix whose rows are $\{a_i^T: i \in I \cup E\}$.
If $A$ is TU, then $P$ is integral.
\end{theorem}
\begin{proof}[Proof sketch]
We need to show that every minimal face of $P$ contains an integral vector.
Every minimal face $F$ is given by $\{x: Bx = c\}$, which is a subsystem of $Ax = b$.
Find a basis $U$ of the columns of $B$. Then $U$ would be a full-rank square submatrix of $B$.
Use $U^{-1}c$ to construct an integral point in $F$.
\end{proof}

\begin{theorem}[Hoffman-Kruskal]
Let $A \in \mathbb{Z}^{m \times n}$. $A$ is TU iff
$\{x: Ax \le b\}$ is integral for all $b \in \mathbb{Z}^m$.
\end{theorem}

\section{TUity-Preserving Operations on Matrices}

\begin{lemma}
Let $A \in \mathbb{R}^{m \times n}$.
\begin{tightenum}
\item\label{item:TUop:neg}$A$ is TU iff $-A$ is TU.
\item\label{item:TUop:trn}$A$ is TU iff $A^T$ is TU.
\item\label{item:TUop:rearrange}If $B$ is obtained by rearranging the rows or columns of $A$,
    then $A$ is TU iff $B$ is TU.
\item\label{item:TUop:mul}If $B$ is obtained by multiplying a row or column of $A$ by a scalar $\alpha \in \{-1, 0, 1\}$,
    then $A$ is TU $\implies$ $B$ is TU.
\item\label{item:TUop:stack}$A$ is TU iff $[A, I]$ is TU.
\item\label{item:TUop:pivot}If $A'$ is obtained by pivoting $A$ at $(i, j)$, then $A'$ is TU if $A$ is TU.
\item\label{item:TUop:inv}If $A$ is invertible, then $A$ is TU iff $A^{-1}$ is TU.
\end{tightenum}
\end{lemma}
\ref{item:TUop:neg}, \ref{item:TUop:trn}, \ref{item:TUop:rearrange}, \ref{item:TUop:mul} are trivial to prove.
\ref{item:TUop:inv} is a corollary of \ref{item:TUop:stack} and \ref{item:TUop:pivot},
since we can obtain $[I, A^{-1}]$ by repeatedly pivoting $[A, I]$.
\begin{proof}[Proof sketch of \ref{item:TUop:stack}]
For any square submatrix containing a few rows and columns from $I$,
repeatedly pivot on elements of $I$ till we get a submatrix of $A$.
\end{proof}
\begin{proof}[Proof of \ref{item:TUop:pivot}]
Let $J \subseteq [m]$ and $K \subseteq [n]$. Let $B \defeq A[J, K]$ and $B' \defeq A'[J, K]$.
Then $\det(B) \in \{-1, 0, 1\}$ because $A$ is TU. We will show that $\det(B') \in \{-1, 0, 1\}$.

If $i \in J$, then $B'$ can be obtained by performing row operations on $B$.
Hence, $\det(B') = \det(B) \in \{-1, 0, 1\}$.
If $i \not\in J$ and $j \in K$, then $B'$ has a zero column, so $\det(B') = 0$.

Suppose $i \not\in J$ and $j \not\in K$.
Let $J' \defeq \{i\} \cup J$ and $K' \defeq \{j\} \cup K$.
Let $C \defeq A[J', K']$ and $C' \defeq A'[J', K']$.
Then $C'$ can be obtained by performing row operations on $C$.
Hence, $\det(C') = \det(C) \in \{-1, 0, 1\}$. Also,
\[ C' = \begin{bmatrix}1 & A'[i, K] \\ \textbf{0} & B'\end{bmatrix}. \]
Hence, $\det(C') = \det(B')$. Hence, $\det(B') \in \{-1, 0, 1\}$.
\end{proof}

\section{Conditions for TUity}

\begin{lemma}[Sufficient condition]
\label{thm:tu-2-part}
Let $A \in \{-1, 0, 1\}^{m \times n}$. Then $A$ is TU if
each column of $A$ contains at most two non-0 elements
and $\exists M \subseteq [m]$ (subset of rows) such that
every column $j$ with two non-0 entries satisfies
\begin{equation}
\label{eqn:partition-sum}
\sum_{i \in M} A[i,j] = \sum_{i \not\in M} A[i,j].
\end{equation}
\end{lemma}
\begin{proof}[Proof sketch]
Let $B$ be the smallest submatrix of $A$ such that $\det(B) \not\in \{-1, 0, 1\}$.
Every column of $B$ has exactly two non-0 elements,
else we can construct a smaller counterexample.
\Cref{eqn:partition-sum} implies that rows of $B$
are linearly dependent, and so $\det(B) = 0$.
\end{proof}

\begin{lemma}[Characterization]
\label{thm:tu-char}
Let $A \in \{-1, 0, 1\}^{m \times n}$. $A$ is TU iff
$\forall J \subseteq [m]$, $\exists K \subseteq J$,
\[ \abs{\sum_{i \in K} A[i,j] - \sum_{i \in J-K} A[i,j]} \le 1 \qquad \forall j \in [n]. \]
\end{lemma}

\section{Examples}

\begin{lemma}[Interval matrix]
A matrix $A \in \{0, 1\}^{m \times n}$ is called an \emph{interval matrix}
if in each column, all ones are in consecutive positions.
An interval matrix is TU.
\end{lemma}
\begin{proof}[Proof sketch]
Use \cref{thm:tu-char} with $K$ as alternate rows of $J$.
\end{proof}

\begin{lemma}[Directed incidence matrix]
Let $G \defeq (V, E)$ be a directed graph.
The incidence matrix of $G$ is defined as the matrix $A \in \{-1, 0, 1\}^{|V| \times |E|}$, where
\[ A[w, (u, v)] \defeq \begin{cases}0 & \textrm{ if } w \not\in \{u, v\}
\\ -1 & \textrm{ if } w = u
\\ 1 & \textrm{ if } w = v
\end{cases}. \]
Then $A$ is TU.
\end{lemma}
\begin{proof}[Proof sketch]
Use \cref{thm:tu-2-part} with $M = \emptyset$.
\end{proof}

\begin{lemma}
\label{thm:cycle-det}
Let $A \in \{0, 1\}^{n \times n}$, where $A[i, j] = 1$ iff $j \in \{i+1, i+1-n\}$.
Then $\det(A)$ is 0 if $n$ is even and $2$ if $n$ is odd.
\end{lemma}
\begin{proof}[Proof sketch]
Apply two row operations to the determinant to reduce to
an $(n-2) \times (n-2)$ matrix of the same structure.
\end{proof}

\begin{lemma}[Undirected incidence matrix]
Let $G \defeq (V, E)$ be an undirected graph.
The incidence matrix of $G$ is defined as the matrix $A \in \{0, 1\}^{|V| \times |E|}$,
where $A[v, e]$ is 1 iff $v$ is an endpoint of $e$.
Then $A$ is TU iff $G$ is bipartite.
\end{lemma}
\begin{proof}[Proof sketch]
If $G$ is bipartite with vertex partitions $L$ and $R$,
use \cref{thm:tu-2-part} with $M = L$.
If $G$ is not bipartite, it contains a cycle $C$ of odd length.
Let $V'$ and $E'$ be the vertices and edges in $C$.
Then $\det(A[V', E']) = 2$, by \cref{thm:cycle-det}.
\end{proof}

\end{document}

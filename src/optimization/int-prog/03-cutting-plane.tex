\documentclass[a4paper,12pt,fleqn]{article}

\usepackage{amsmath,amsthm,amssymb}
\usepackage[margin=1in]{geometry}
\usepackage{xcolor}
\usepackage{array}
\usepackage{comment}
\usepackage{enumitem}
\usepackage[bookmarksnumbered=true,hypertexnames=false]{hyperref}
\usepackage[capitalize,sort]{cleveref}

\def\colorschemesepia{sepia}
\def\colorschemedark{dark}
\def\colorschemelight{light}

\ifx\colorscheme\undefined
\let\colorscheme\colorschemelight
\fi

\ifx\colorscheme\colorschemelight
\colorlet{textColor}{black}
\colorlet{bgColor}{white}
\fi

\ifx\colorscheme\colorschemesepia
\definecolor{textColor}{HTML}{433423}
\definecolor{bgColor}{HTML}{fbf0da}
\fi

\ifx\colorscheme\colorschemedark
\definecolor{textColor}{HTML}{bdc1c6}
\definecolor{bgColor}{HTML}{202124}
\definecolor{textBlue}{HTML}{8ab4f8}
\definecolor{textRed}{HTML}{f9968b}
\colorlet{bgRed}{red!40!black}
\colorlet{bgBlue}{red!40!black}
\else
\colorlet{textBlue}{blue!50!black}
\colorlet{textRed}{red!50!black}
\fi

\ifx\colorscheme\colorschemelight\else
\pagecolor{bgColor}
\color{textColor}
\fi

\hypersetup{colorlinks,linkcolor=textRed,citecolor=textRed,urlcolor=textBlue}
% THEOREMS

\newtheorem{theorem}{Theorem}
\newtheorem{definition}{Definition}
\newtheorem{example}{Example}
\newtheorem{corollary}{Corollary}[theorem]
\newtheorem{lemma}[theorem]{Lemma}
\newtheorem{claim}[theorem]{Claim}

% SHORTHANDS

\newcommand*{\Th}{^{\textrm{th}}}
\newcommand*{\WLoG}{Without loss of generality}
\newcommand*{\wLoG}{without loss of generality}
\newcommand*{\wrt}{with respect to}
\let\opname\operatorname

% SYMBOLS

\let\eps\epsilon
\newcommand*{\defeq}{:=}

\newcommand*{\alphahat}{\widehat{\alpha}}
\newcommand*{\betahat}{\widehat{\beta}}
\newcommand*{\Ahat}{\widehat{A}}
\newcommand*{\Bhat}{\widehat{B}}
\newcommand*{\Chat}{\widehat{C}}
\newcommand*{\Jhat}{\widehat{J}}
\newcommand*{\Shat}{\widehat{S}}
\newcommand*{\Yhat}{\widehat{Y}}
\newcommand*{\bhat}{\widehat{b}}
\newcommand*{\shat}{\widehat{s}}
\newcommand*{\what}{\widehat{w}}
\newcommand*{\xhat}{\widehat{x}}
\newcommand*{\yhat}{\widehat{y}}
\newcommand*{\zhat}{\widehat{z}}

\newcommand*{\Btild}{\widetilde{B}}
\newcommand*{\Ctild}{\widetilde{C}}
\newcommand*{\Jtild}{\widetilde{J}}
\newcommand*{\Stild}{\widetilde{S}}
\newcommand*{\Ytild}{\widetilde{Y}}
\newcommand*{\btild}{\widetilde{b}}
\newcommand*{\stild}{\widetilde{s}}
\newcommand*{\wtild}{\widetilde{w}}
\newcommand*{\xtild}{\widetilde{x}}
\newcommand*{\ytild}{\widetilde{y}}
\newcommand*{\ztild}{\widetilde{z}}

\newcommand*{\Fcal}{\mathcal{F}}
\newcommand*{\Tcal}{\mathcal{T}}

\newcommand*{\ebar}{\overline{e}}
\newcommand*{\Xbar}{\overline{X}}
\newcommand*{\xbar}{\overline{x}}
\newcommand*{\Ybar}{\overline{Y}}
\newcommand*{\ybar}{\overline{y}}

% MATH

\newcommand*{\floor}[1]{\left\lfloor #1 \right\rfloor}
\newcommand*{\smallfloor}[1]{\lfloor #1 \rfloor}
\newcommand*{\ceil}[1]{\left\lceil #1 \right\rceil}
\newcommand*{\smallceil}[1]{\lceil #1 \rceil}
\newcommand*{\abs}[1]{\left\lvert #1 \right\rvert}
\newcommand*{\smallabs}[1]{\lvert #1 \rvert}
\newcommand*{\norm}[1]{\left\lVert #1 \right\rVert}
\newcommand*{\smallnorm}[1]{\lVert #1 \rVert}
\newcommand*{\Z}{\mathbb{Z}}
\DeclareMathOperator*{\E}{E}
\DeclareMathOperator*{\Var}{Var}
\DeclareMathOperator*{\argmin}{argmin}
\DeclareMathOperator*{\argmax}{argmax}
\DeclareMathOperator{\poly}{poly}
\DeclareMathOperator{\opt}{opt}
\DeclareMathOperator{\argopt}{argopt}
\DeclareMathOperator{\support}{support}
\newcommand*{\OPT}{\mathrm{OPT}}


\author{Eklavya Sharma}
\date{\empty}

\newenvironment*{tightenum}{\begin{enumerate}[noitemsep]}{\end{enumerate}}
\DeclareMathOperator{\convexHull}{convexHull}
\DeclareMathOperator{\depth}{depth}

\title{Integer Programming: Cutting Plane Method}

\begin{document}

\maketitle
\setlength{\parskip}{0.2em}

\begin{lemma}[Disjunction]
Let $S_1, S_2 \subseteq \mathbb{R}^n$.
If $a^Tx \le \alpha$ for all $x \in S_1$,
and $b^Tx \le \beta$ for all $x \in S_2$,
then $\min(a, b)^Tx \le \max(\alpha, \beta)$ for all $x \in S_1 \cup S_2$.
($\min(a, b)_i \defeq \min(a_i, b_i)$.)
\end{lemma}

\begin{definition}[Inequality dominance]
Let $S \subseteq \mathbb{R}^n$.
The inequality $a^Tx \ge b$ dominates $c^Tx \ge d$ in set $S$ if
$\{x \in S: a^Tx \ge b\} \subseteq \{x \in S: c^Tx \ge d\}$.
\end{definition}

\begin{definition}[Mixing]
Suppose we are given a set $S$ of inequalities
$\{a_i^Tx \ge b_i: i \in I\} \cup \{a_i^Tx = b_i: i \in [m] - I\}$.
Let $w \in \mathbb{R}^m$, where $w_i \ge 0$ for $i \in I$.
Then the inequality $(w^TA)x \ge w^Tb$ is called a \emph{mixing} of the inequalities $S$
(we are essentially doing a linear combination of the inequalities to get a new inequality).
\end{definition}

\begin{theorem}[Valid dominated by mixing]
Let $A \in \mathbb{R}^{m \times n}$, $I \subseteq [m]$, and $J \subseteq [n]$. Let
\\ $P \defeq \{x \in \mathbb{R}^n: (x_j \ge 0, \forall j \in J), ((Ax)_i \ge b_i, \forall i \in I),
((Ax)_i = b_i, \forall i \in [m]-I) \}$ be a non-empty polyhedron.
Let $c^Tx \ge \gamma$ be an inequality satisfied by all points in $P$. Let
\\ $Q \defeq \{w \in \mathbb{R}^m: (y_i \ge 0, \forall i \in I), ((A^Tw)_j \le c_j, \forall j \in J),
((A^Tw)_j = c_j, \forall j \in [n]-J) \}$.
Then $Q \neq \emptyset$. Let $\what \in \argmax_{w \in Q} b^Tw$.
Then $c^Tx \ge \gamma$ is dominated by $(\what^TA)x \ge \what^Tb$
in $S \defeq \{x \in \mathbb{R}^n: x_j \ge 0, \forall j \in J\}$.
\end{theorem}
\begin{proof}
\WLoG, assume $\{x \in P: c^Tx = \gamma\} \neq \emptyset$, since we can increase $\gamma$.
Then $\gamma$ is the optimal objective value of the LP $\min_{x \in P} c^Tx$.
The dual of this LP is $\max_{w \in Q} b^Tw$.
By strong duality, $Q \neq \emptyset$ and $b^T\what = \gamma$.

Let $\xhat \in \{x \in S: \what^TAx \ge \what^Tb\}$. We will show that $c^T\xhat \ge \gamma$,
which will imply that $(\what^TA)x \ge \what^Tb$ dominates $c^Tx \ge \gamma$.
When $j \in J$, then $\xhat_j \ge 0$ and $(A^T\what)_j \le c_j$, so $\xhat_j(c - A^T\what)_j \ge 0$.
When $j \not\in J$, then $(A^T\what)_j = c_j$, so $\xhat_j(c - A^T\what)_j = 0$.
Hence, $\xhat^T(c - A^T\what) \ge 0$, so $c^T\xhat \ge \what^TA\xhat \ge \what^Tb = \gamma$.
\end{proof}

\begin{lemma}[Rounding]
\label{thm:rounding-is-valid}
Let $S \subseteq \mathbb{R}^n$ and $a \in \mathbb{Z}^n$. Then
\[ (\forall x \in S, a^Tx \ge b) \implies (\forall x \in S \cap \mathbb{Z}^n, a^Tx \ge \ceil{b}). \]
\[ (\forall x \in S, a^Tx \le b) \implies (\forall x \in S \cap \mathbb{Z}^n, a^Tx \le \floor{b}). \]
This transformation is called \emph{rounding}.
\end{lemma}
\begin{corollary}
\label{thm:porth-rounding-is-valid}
Let $S \subseteq \mathbb{R}^n_{\ge 0}$. Then
\[ (\forall x \in S, a^Tx \ge b) \implies (\forall x \in S \cap \mathbb{Z}^n, \ceil{a}^Tx \ge \ceil{b}). \]
\[ (\forall x \in S, a^Tx \le b) \implies (\forall x \in S \cap \mathbb{Z}^n, \floor{a}^Tx \le \floor{b}). \]
\end{corollary}

\section{\texorpdfstring{Chv\'atal}{Chvatal}-Gomory Process}

\begin{definition}[Chv\'atal-Gomory process]
Let $P \subseteq \mathbb{R}^n$ be a polyhedron described as a set $S$ of inequalities.
Assign a \emph{depth} of 0 to all inequalities in $S$.
Repeated application of the following operations (for any number of iterations)
is called the Chv\'atal-Gomory (CG) process:
\begin{tightenum}
\item Mix a subset $S'$ of inequalities from $S$ to get a new inequality $c^Tx \ge d$
    such that $c \in \mathbb{Z}^n$.
\item Add the inequality $c^Tx \ge \ceil{d}$ to $S$.
    Assign the depth $1 + \max_{i \in S'} \depth(i)$ to this newly-added inequality.
\end{tightenum}
The output of the process is $S$, the set of (original and newly-added) inequalities.
\end{definition}

\begin{lemma}
Let $Q$ be the output of some CG process on polyhedron $P$.
Let $P_I \defeq \convexHull(P \cap \mathbb{Z}^n)$.
Then $P_I \subseteq Q \subseteq P$.
\end{lemma}

\begin{definition}[CG rank and CG closure]
Let $P \subseteq \mathbb{R}^n$ be a polyhedron, represented as a set $S$ of inequalities.
For any inequality $c^Td \ge d$, where $c \in \mathbb{Z}^n$ and $d \in \mathbb{Z}$,
the CG rank of the inequality is $r$ if it can be obtained as a depth $r$
inequality in some CG process.

The $r\Th$ CG-closure of $P$ is defined as the set of inequalities (and the associated polyhedron)
of CG-rank at most $r$.

The CG-rank of a polyhedron is the smallest number $t$ such that the $t\Th$ CG-closure
of $P$ is $\convexHull(P \cap \mathbb{Z}^n)$ (and $\infty$ if no such integer $t$ exists).
\end{definition}

\begin{lemma}
Let $P \subseteq \mathbb{R}^n$ be a polyhedron,
$P_I \defeq \convexHull(P \cap \mathbb{Z}^n)$,
and $P^{(r)}$ be the $r\Th$ CG closure of $P$.
Then $P_I \subseteq P^{(r)} \subseteq P$.
\end{lemma}

\begin{lemma}
Let $P \defeq \{x \in \mathbb{R}^n: Ax \le b\}$,
where $A \in \mathbb{Z}^{m \times n}$ and $b \in \mathbb{Z}^m$.
Let $P' \defeq \{x: Ax \le b, (w^TA)x \le \floor{w^Tb} \forall w \in [0, 1]^m
\textrm{ such that } w^TA \in \mathbb{Z}^n\}$.
Then $P'$ is the first CG-closure of $P$.
\end{lemma}
\begin{proof}[Proof sketch]
Any rank 1 inequality can be written as $(w^TA)x \le \floor{w^Tb}$ for some $w \ge 0$,
where $w^TA \in \mathbb{Z}^n$. Let $u \defeq w - \floor{w}$.
We can show that $(u^TA)x \le \floor{u^Tb}$ is a rank-1 CG inequality,
and $w^TAx \le \floor{w^Tb}$ is implied by $(u^TA)x \le \floor{u^Tb}$ and $Ax \le b$.
\end{proof}

\begin{theorem}
If $P$ is a rational polyhedron, then the first CG-closure of $P$ is also a rational polyhedron.
\end{theorem}

\begin{theorem}[Chv\'atal-Gomory]
Given a rational polyhedron $P$ described as a set $S$ of inequalities,
we can obtain $\convexHull(P \cap \mathbb{Z}^n)$ using the CG process
for a finite number of iterations
(by appropriately choosing which inequalities to mix in each step).
\end{theorem}
\begin{corollary}
The CG rank of a rational polyhedron is finite.
\end{corollary}

\subsection{Extra Results}

\begin{theorem}
There exist polyhedra whose CG rank is super-polynomial in $mn$.
\end{theorem}

\begin{theorem}
Polyhedra in $[0, 1]^n$ have CG rank at most $n^2(1 + \log n)$.
There exist polyhedra in $[0, 1]^n$ whose CG rank is at least $cn^2$ for some constant $c$.
\end{theorem}

\section{Examples}

\begin{theorem}[Non-bipartite Matching]
Let $G \defeq (V, E)$ be an undirected graph. Let
\[ P \defeq \left\{x \in \mathbb{R}^{|E|}: \left(\sum_{e \in \delta(v)} x_e \le 1, \forall v \in V\right),
    (x_e \ge 0, \forall e \in E)\right\}. \]
Then for any $S \subseteq V$ where $|S|$ is odd, the following inequality has CG rank at most 1:
\[ \sum_{e \in E(S)} x_e \le \frac{|S|-1}{2}. \tag{here $E(S) \defeq \{(u, v) \in E: u \in S, v \in S\}$} \]
\end{theorem}
\begin{proof}
\begin{align*}
& \sum_{v \in S} \frac{1}{2}\left(\sum_{e \in \delta(v)} x_e \le 1\right)
+ \sum_{e \in \delta(S)} \frac{1}{2}\left(-x_{e} \le 0\right)
= \left(\sum_{e \in E(S)} x_e \le \frac{|S|}{2}\right).
\end{align*}
Then round this inequality. $\floor{|S|/2} = (|S|-1)/2$.
\end{proof}

\begin{theorem}[Independent set]
Let $G \defeq (V, E)$ be an undirected graph. Let
\[ P \defeq \left\{x \in \mathbb{R}^{|V|}: (x_u + x_v \le 1, \forall (u, v) \in E),
(0 \le x_v \le 1, \forall v \in V)\right\}. \]
Then for any odd-cycle $C$, the following is a CG rank 1 inequality:
\begin{equation}
\label{eqn:indset:odd-cycle}
\sum_{v \in C} x_v \le \frac{|C|-1}{2}.
\end{equation}
For any clique $S \subseteq V$, the following is an inequality of CG rank at most
$\ceil{\log_2(|S|-1)}$:
\begin{equation}
\label{eqn:indset:clique}
\sum_{v \in S} x_v \le 1.
\end{equation}
\end{theorem}
\begin{proof}
\Cref{eqn:indset:odd-cycle} has rank at most 1 because
\[ \sum_{(u, v) \in C} \frac{1}{2}\left(x_u + x_v \le 1\right)
= \left(\sum_{v \in C} x_v \le \frac{|C|}{2}\right). \]
It has rank exactly 1 since it contains $|C| \ge 3$ terms,
but the original inequalities of $P$ contain at most 2 terms.

WLoG, assume $S \defeq \{1, 2, \ldots, n\}$. Let $m \defeq \ceil{\log_2(n-1)}$.
For $0 \le i \le m$, let $r_i \defeq \min(n, 2^i+1)$. Then $r_{m-1} < n = r_m$.
Let $B(k)$ be the proposition that for any $R \subseteq S$ such that $|R| \le r_k$,
$\sum_{i \in R} x_i \le 1$ is a CG inequality of rank at most $k$.
Then $B(m)$ would imply \cref{eqn:indset:clique}.

We will show $B(k)$ by induction on $k$.
If $|R| \le 2 = r_0$, then $\sum_{i \in R} x_i \le 1$ is a constraint of $P$
(since $S$ is a clique), and so has rank 0.
Now let $r_{k-1} < |R| \le r_k$ for some $k \ge 1$.
By the inductive hypothesis, for any $T \subseteq S$ such that $|T| = r_{k-1}$,
$\sum_{i \in T} x_i \le 1$ is an inequality of CG rank at most $k-1$. Then
\[ \sum_{T \subseteq R: |T| = r_{k-1}} \left(\sum_{i \in T} x_i \le 1\right)
= \left(\binom{|R|-1}{r_{k-1}-1} \sum_{i \in R} x_i \le \binom{|R|}{r_{k-1}} \right). \]
\begin{align*}
\binom{|R|}{r_{k-1}} &= \frac{|R|}{r_{k-1}}\binom{|R|-1}{r_{k-1}-1},
& \frac{|R|}{r_{k-1}} &\le \frac{r_k}{r_{k-1}} = 2 - \frac{1}{r_{k-1}}.
\end{align*}
Hence, $\sum_{i \in R} x_i \le 1$ is an inequality of CG rank at most $k$.
By induction, $B(k)$ $\forall k$.
\end{proof}

\begin{theorem}[Knapsack, \cite{dietrich1994}]
Let $P \defeq \{x \in [0, 1]^n: a^Tx \le b\}$,
where $b \in \mathbb{Z}_{\ge 1}$, and $a \in ([1, b] \cap \mathbb{Z})^n$.
For any $S \subseteq [n]$, let $a(S) \defeq \sum_{i \in S} a_i$.
Let $C \subseteq [n]$ be a \emph{minimal cover},
i.e., $a(C) > b$ and $a(C') \le b$ for all $C' \subsetneq C$.
Then $\sum_{i \in C} x_i \le |C|-1$ is an inequality of CG rank at most 1.
\end{theorem}
\begin{proof}
\begin{align*}
& \sum_{i \in C} \left(1 - \frac{a_i}{b+1}\right)(x_i \le 1)
    + \sum_{i \in [n]-C} \left(\frac{a_i}{b+1}\right)(-x_i \le 0)
    + \frac{(a^Tx \le b)}{b+1}
\\ &= \left(\sum_{i \in C} x_i \le |C| - \frac{a(C)-b}{b+1}\right).
\end{align*}
By minimality of $C$, we get $b < a(C) \le 2b$,
so $(a(C)-b)/(b+1) \in (0, 1)$.
\end{proof}

\bibliographystyle{plainurl}
\bibliography{bibdb}

\end{document}
